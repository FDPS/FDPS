\documentclass[12pt,a4paper]{jarticle}
%
\topmargin=-5mm
\oddsidemargin=-5mm
\evensidemargin=-5mm
\textheight=235mm
\textwidth=165mm
%
\title{FDPSユーザーチュートリアル}
\author{FDPS開発者}
\date{}
%\pagestyle{empty}
\usepackage{graphicx}
\usepackage{wrapfig}
\usepackage{lscape}
\usepackage{amssymb}
\usepackage{amsmath}
\usepackage{bm}
\usepackage{setspace}
\usepackage{listings,jlisting}
\usepackage{color}
\usepackage{ascmac}
\usepackage{here}

\newcommand{\underbold}[1]{\underline{\bf #1}}
\newcommand{\redtext}[1]{\textcolor{red}{#1}}


%\setcounter{secnumdepth}{4}
%%%%%%%%%%%%%%%%%%%%%%%%%%%%%%%%%%
\setcounter{secnumdepth}{5}
\makeatletter
\newcommand{\subsubsubsection}{\@startsection{paragraph}{4}{\z@}%
{1.5\baselineskip \@plus.5\dp0 \@minus.2\dp0}%
{.5\baselineskip \@plus2.3\dp0}%
{\reset@font\normalsize\bfseries}
}
\newcommand{\subsubsubsubsection}{\@startsection{subparagraph}{5}{\z@}%
{1.5\baselineskip \@plus.5\dp0 \@minus.2\dp0}%
{.5\baselineskip \@plus2.3\dp0}%
{\reset@font\normalsize\itshape}
}
\makeatother
\setcounter{tocdepth}{5}
%%%%%%%%%%%%%%%%%%%%%%%%%%%%%%%%%%

%\twocolumn
%\setstretch{1.5}

\lstset{language = C,
numbers = left,
numbersep = 8pt,
breaklines = true,
breakindent = 40pt,
frame = lines,
basicstyle = \ttfamily,
}

\begin{document}
\maketitle
\tableofcontents

\newpage

%%%%%%%%%%%%%%%%%%%%%%%%%%%%%%%%%%%%%%%%%%%%%%%%%%%%%
\section{TODO}

\newpage

%%%%%%%%%%%%%%%%%%%%%%%%%%%%%%%%%%%%%%%%%%%%%%%%%%%%%
\section{変更記録}

\begin{itemize}
\item 2015/01/25 作成
\end{itemize}

\newpage

%%%%%%%%%%%%%%%%%%%%%%%%%%%%%%%%%%%%%%%%%%%%%%%%%%%%%
\section{概要}

本節では、Framework for Developing Particle Simulator (FDPS)の概要につ
いて述べる。FDPSは粒子シミュレーションのコード開発を支援するフレームワー
クである。FDPSが行うのは、計算コストの最も大きな粒子間相互作用の計算と、
粒子間相互作用の計算のコストを負荷分散するための処理である。これらはマ
ルチプロセス、マルチスレッドで並列に処理することができる。比較的計算コ
ストが小さく、並列処理を必要としない処理(粒子の軌道計算など)はユーザー
が行う。

FDPSが対応している座標系は、2次元直交座標系と3次元直交座標系である。
また、境界条件としては、開放境界条件と周期境界条件に対応している。周期
境界条件の場合、$x$、 $y$、 $z$軸方向の任意の組み合わせの周期境界条件を
課すことができる。

ユーザーは粒子間相互作用の形を定義する必要がある。定義できる粒子間相互
作用の形には様々なものがある。粒子間相互作用の形を大きく分けると2種類
あり、1つは長距離力、 もう1つは短距離力である。 この2つの力は、 遠くの
複数の粒子からの作用を1つの超粒子からの作用にまとめるか(長距離力)、 ま
とめないか(短距離力)という基準でもって分類される。

長距離力には、 小分類があり、 無限遠に存在する粒子からの力も計算するカッ
トオフなし長距離力と、 ある距離以上離れた粒子からの力は計算しないカット
オフあり長距離力がある。  前者は開境界条件下における重力やクーロン力に
対して、 後者は周期境界条件下の重力やクーロン力に使うことができる。 後
者のためにはParticle Mesh法などが必要となるが、これはFDPSの拡張機能とし
て用意されている。

短距離力には、小分類が4つ存在する。短距離力の場合、粒子はある距離より離
れた粒子からの作用は受けない。すなわち必ずカットオフが存在する。 このカッ
トオフ長の決め方によって、 小分類がなされる。 すなわち、 全粒子のカット
オフ長が等しいコンスタントカーネル、 カットオフ長が作用を受ける粒子固有
の性質で決まるギャザーカーネル、 カットオフ長が作用を与える粒子固有の性
質で決まるスキャッタカーネル、 カットオフ長が作用を受ける粒子と作用を与
える粒子の両方の性質で決まるシンメトリックカーネルである。 コンスタント
カーネルは分子動力学におけるLJ力に適用でき、 その他のカーネルはSPHなど
に適用できる。

ユーザーは、粒子間相互作用や粒子の軌道積分などを、C++言語を用いて記述す
る。将来的にはFortran言語でも記述できるように検討する。


\newpage

%%%%%%%%%%%%%%%%%%%%%%%%%%%%%%%%%%%%%%%%%%%%%%%%%%%%%
\section{入門}

本節では、FDPSの入門について記述する。FDPSを使用する環境、FDPSに必要な
ソフトウェア、FDPS のインストール方法、サンプルコードの使用方法、の順で
記述する。

%%%%%%%%%%%%%%%%%%%%%%%%%%%%%%%%%%%%%%%%%%%%%%%%%%%%%
\subsection{動作環境}

FDPSが動作する環境は以下の通りである。
\begin{itemize}
\item Linuxのターミナル
\item Mac OS Xのターミナル
\item Windows のCygwin上のターミナル
\end{itemize}

%%%%%%%%%%%%%%%%%%%%%%%%%%%%%%%%%%%%%%%%%%%%%%%%%%%%%
\subsection{必要なソフトウェア}

本節では、FDPSを使用する際に必要となるソフトウェアを記述する。まず標準
機能を用いるのに必要なソフトウェア、次に拡張機能を用いるのに必要なソフ
トウェアを記述する。

\subsubsection{標準機能}

本節では、FDPSの標準機能のみを使用する際に必要なソフトウェアを記述する。
最初に逐次処理機能のみを用いる場合(並列処理機能を用いない場合)に必要
なソフトウェアを記述する。次に並列処理機能を用いる場合に必要なソフトウェ
アを記述する。

\subsubsubsection{逐次処理}

逐次処理の場合に必要なソフトウェアは以下の通りである。
\begin{itemize}
\item make
\item C++03以降のコンパイラ
\end{itemize}

\subsubsubsection{並列処理}

本節では、FDPSの並列処理機能を用いる際に必要なソフトウェアを記述する。
まず、OpenMPを使用する際に必要なソフトウェア、次にMPIを使用する際に必要
なソフトウェア、最後にOpenMPとMPIを同時に使用する際に必要なソフトウェア
を記述する。

\subsubsubsubsection{OpenMP}

OpenMPを使用する際に必要なソフトウェアは以下の通り。
\begin{itemize}
\item make
\item OpenMP対応のC++03以降のコンパイラ
\end{itemize}

\subsubsubsubsection{MPI}

MPIを使用する際に必要なソフトウェアは以下の通り。
\begin{itemize}
\item make
\item MPI対応のC++03以降のコンパイラ
\end{itemize}

\subsubsubsubsection{MPI+OpenMP}

MPIとOpenMPを同時に使用する際に必要なソフトウェアは以下の通り。
\begin{itemize}
\item make
\item MPIとOpenMPに対応のC++03以降のコンパイラ
\end{itemize}

\subsubsection{拡張機能}

本節では、FDPSの拡張機能を使用する際に必要なソフトウェアについて述べる。
FDPSの拡張機能にはParticle Meshがある。以下ではParticle Meshを使用する
際に必要なソフトウェアを述べる。

\subsubsubsection{Particle Mesh}

Particle Meshを使用する際に必要なソフトウェアは以下の通りである。
\begin{itemize}
\item make
\item MPIとOpenMP対応のC++03以降のコンパイラ
\item FFTW 3.3以降
\end{itemize}

%%%%%%%%%%%%%%%%%%%%%%%%%%%%%%%%%%%%%%%%%%%%%%%%%%%%%
\subsection{インストール}

本節では、FDPSのインストールについて述べる。取得方法、ビルド方法、サン
プルコードについて述べる。

\subsubsection{取得方法}

以下の方法のいずれかでFDPSの圧縮ファイル
\redtext{fdps-20xx-xx-xx.tar.gz}を取得できる。
\begin{itemize}
\item ウェブサイトhttps://github.com/FDPS/FDPSからダウンロード
\item \redtext{コマンドラインから}
\end{itemize}

FDPSを展開したいディレクトリに移動し、コマンドライン上で以下を実行する
と、圧縮ファイルを展開できる。
\begin{screen}
\begin{verbatim}
$ tar zxvf fdps-20xx-xx-xx.tar.gz
\end{verbatim}
\end{screen}
展開に成功すると、ディレクトリfdpsができる。

\subsubsection{ビルド方法}

ディレクトリfdpsに移動する。ここでconfigureをする必要はない。

\subsubsection{サンプルコードの使用方法}

本節ではサンプルコードの使用方法について記述する。サンプルコードには重
力$N$体シミュレーションコードと、SPHシミュレーションコードがある。最初
に重力$N$体シミュレーションコード、次にSPHシミュレーションコードの使用
について記述する。

\subsubsubsection{重力$N$体シミュレーションコード}

以下の手順で本コードを使用できる。
\begin{itemize}
\item ディレクトリfdps/sample/nbodyに移動
\item カレントディレクトリにあるMakefileを編集(後述)
\item コマンドライン上でmakeを実行
\item makeされた実行ファイルを実行(後述)
\end{itemize}

Makefileの編集項目は以下の通りである。OpenMPとMPIを使用するかどうかで編
集方法が変ることに注意。
\begin{itemize}
\item OpenMPもMPIも使用しない場合
  \begin{itemize}
  \item マクロCCにC++コンパイラを代入する
  \end{itemize}

\item OpenMPのみ使用の場合
  \begin{itemize}
  \item マクロCCにOpenMP対応のC++コンパイラを代入する
  \item "CFLAGS += -DPARTICLE\_SIMULATOR\_THREAD\_PARALLEL -fopenmp"の
    行のコメントアウトを外す(インテルコンパイラの場合は-fopenmpを外す)
  \end{itemize}

\item MPIのみ使用の場合
  \begin{itemize}
  \item マクロCCにMPIC++コンパイラを代入する
  \item "CFLAGS += -DPARTICLE\_SIMULATOR\_MPI\_PARALLEL"の行のコメント
    アウトを外す
  \end{itemize}

\item OpenMPとMPIの同時使用の場合
  \begin{itemize}
  \item マクロCCにMPI対応のC++コンパイラを代入する
  \item "CFLAGS += -DPARTICLE\_SIMULATOR\_MPI\_PARALLEL"の行のコメント
    アウトを外す
  \item "CFLAGS += -DPARTICLE\_SIMULATOR\_THREAD\_PARALLEL -fopenmp"の
    行のコメントアウトを外す(インテルコンパイラの場合は-fopenmpを外す)
  \end{itemize}

\end{itemize}

実行ファイルの実行方法は以下の通りである。
\begin{itemize}
\item MPIを使用しない場合、コマンドライン上で以下のコマンドを実行する
\begin{screen}
\begin{verbatim}
$ ./nbody
\end{verbatim}
\end{screen}
  
\item MPIを使用する場合、コマンドライン上で以下のコマンドを実行する
\begin{screen}
\begin{verbatim}
$ MPIRUN -np NPROC ./nbody
\end{verbatim}
\end{screen}
ここで、"MPIRUN"にはmpirunやmpiexecなどが、"NPROC"には使用するMPIプロセ
スの数が入る。

以下の手順でxy平面に射影した粒子分布の変化を見ることができる。
\begin{itemize}
\item gnuplotを起動する
\item plot "filename\_time.dat" using 1:2
\end{itemize}
出力ファイルフォーマットは1列目から順に粒子の位置のx, y, z座標、粒子の
x, y, z軸方向の速度、粒子質量である。

\end{itemize}

\subsubsubsection{SPHシミュレーションコード}

以下の手順で本コードを使用できる。
\begin{itemize}
\item ディレクトリfdps/sample/sphに移動
\item カレントディレクトリにあるMakefileを編集(後述)
\item コマンドライン上でmakeを実行
\item makeされた実行ファイルを実行(後述)
\end{itemize}

Makefileの編集項目は以下の通りである。OpenMPとMPIを使用するかどうかで編
集方法が変ることに注意。
\begin{itemize}
\item OpenMPもMPIも使用しない場合
  \begin{itemize}
  \item マクロCCにC++コンパイラを代入する
  \end{itemize}

\item OpenMPのみ使用の場合
  \begin{itemize}
  \item マクロCCにOpenMP対応のC++コンパイラを代入する
  \item "CFLAGS += -DPARTICLE\_SIMULATOR\_THREAD\_PARALLEL -fopenmp"の
    行のコメントアウトを外す(インテルコンパイラの場合は-fopenmpを外す)
  \end{itemize}

\item MPIのみ使用の場合
  \begin{itemize}
  \item マクロCCにMPIC++コンパイラを代入する
  \item "CFLAGS += -DPARTICLE\_SIMULATOR\_MPI\_PARALLEL"の行のコメント
    アウトを外す
  \end{itemize}

\item OpenMPとMPIの同時使用の場合
  \begin{itemize}
  \item マクロCCにMPI対応のC++コンパイラを代入する
  \item "CFLAGS += -DPARTICLE\_SIMULATOR\_MPI\_PARALLEL"の行のコメント
    アウトを外す
  \item "CFLAGS += -DPARTICLE\_SIMULATOR\_THREAD\_PARALLEL -fopenmp"の
    行のコメントアウトを外す(インテルコンパイラの場合は-fopenmpを外す)
  \end{itemize}

\end{itemize}

実行ファイルの実行方法は以下の通りである。
\begin{itemize}
\item MPIを使用しない場合、コマンドライン上で以下のコマンドを実行する
\begin{screen}
\begin{verbatim}
$ ./sph
\end{verbatim}
\end{screen}
  
\item MPIを使用する場合、コマンドライン上で以下のコマンドを実行する
\begin{screen}
\begin{verbatim}
$ MPIRUN -np NPROC ./sph
\end{verbatim}
\end{screen}
ここで、"MPIRUN"にはmpirunやmpiexecなどが、"NPROC"には使用するMPIプロセ
スの数が入る。
\end{itemize}

以下の手順でxy平面に射影した粒子分布の変化を見ることができる。
\begin{itemize}
\item gnuplotを起動する
\item plot "filename\_time.dat" using 1:2
\end{itemize}
出力ファイルフォーマットは1列目から順に粒子の位置のx, y, z座標、粒子の
x, y, z軸方向の速度、粒子質量、密度、内部エネルギーである。


\newpage

%%%%%%%%%%%%%%%%%%%%%%%%%%%%%%%%%%%%%%%%%%%%%%%%%%%%%
\section{使ってみよう}

\subsection{サンプルコードのコンパイルと実行}

\subsection{前提知識}

\subsubsection{Vector型}

\subsection{(穴埋め式で)固定長SPHシミュレーションコードを書く}

\subsubsection{作業ディレクトリ}

\subsubsection{インクルード}

\subsubsection{ユーザー定義クラス}

\subsubsubsection{概要}

\subsubsubsection{FullParticle型}

\subsubsubsection{(EssentialParticleI型)}

\subsubsubsection{(EssentialParticleJ型)}

\subsubsubsection{(SuperParticleJ型)}

\subsubsubsection{Force型}

\subsubsubsection{calcForceEpEp型}

\subsubsubsection{(calcForceEpSp型)}

\subsubsection{プログラム本体}

\subsubsubsection{概要}

\subsubsubsection{全体の初期化}

\subsubsubsection{インスタンスの生成}

\subsubsubsection{(領域クラスの初期化)}

\subsubsubsection{(粒子群クラスの初期化)}

\subsubsubsection{粒子データの入力}

\subsubsubsection{(相互作用ツリークラスの初期化)}

\subsubsubsection{領域分割の実行}

\subsubsubsection{粒子交換の実行}

\subsubsubsection{相互作用計算の実行}

\subsubsubsection{時間積分}

\subsubsubsection{全体の終了}

\subsubsection{コンパイル}

\subsubsection{実行}

\subsubsection{ログファイル}

\subsubsection{可視化(gnuplot)}

\subsection{(穴埋め式で)N体シミュレーションコードを書く}

\subsubsection{作業ディレクトリ}

\subsubsection{ユーザー定義クラス}

\subsubsubsection{概要}

\subsubsubsection{FullParticle型}

\subsubsubsection{(EssentialParticleI型)}

\subsubsubsection{(EssentialParticleJ型)}

\subsubsubsection{(SuperParticleJ型)}

\subsubsubsection{Force型}

\subsubsubsection{calcForceEpEp型}

\subsubsubsection{(calcForceEpSp型)}

\subsubsection{プログラム本体}

\subsubsubsection{概要}

\subsubsubsection{全体の初期化}

\subsubsubsection{インスタンスの生成}

\subsubsubsection{(領域クラスの初期化)}

\subsubsubsection{(粒子群クラスの初期化)}

\subsubsubsection{粒子データの入力}

\subsubsubsection{(相互作用ツリークラスの初期化)}

\subsubsubsection{領域分割の実行}

\subsubsubsection{粒子交換の実行}

\subsubsubsection{相互作用計算の実行}

\subsubsubsection{時間積分}

\subsubsubsection{全体の終了}

\subsubsection{ログファイル}

\subsection{(穴埋め式で)衝突を考慮した$N$体シミュレーションコードを書く}

\subsubsection{作業ディレクトリ}

\subsubsection{ユーザー定義クラス}

\subsubsection{ログファイル}

\newpage

%%%%%%%%%%%%%%%%%%%%%%%%%%%%%%%%%%%%%%%%%%%%%%%%%%%%%
\section{サンプルコード}

\subsection{$N$体シミュレーション}

\subsection{SPHシミュレーション}

\subsection{$N$体+SPHシミュレーション}

\newpage

%%%%%%%%%%%%%%%%%%%%%%%%%%%%%%%%%%%%%%%%%%%%%%%%%%%%%
\section{よくあるエラーメッセージ}

\newpage

%%%%%%%%%%%%%%%%%%%%%%%%%%%%%%%%%%%%%%%%%%%%%%%%%%%%%
\section{ユーザーサポート}

\newpage

%%%%%%%%%%%%%%%%%%%%%%%%%%%%%%%%%%%%%%%%%%%%%%%%%%%%%
\section{ライセンス}

MITライセンスに準ずる。Particle MeshはGreeMのMITライセンスである。

\end{document}
