
本節では、FDPSの入門について記述する。FDPSを使用する環境、FDPSに必要な
ソフトウェア、FDPS のインストール方法、サンプルコードの使用方法、の順で
記述する。

%%%%%%%%%%%%%%%%%%%%%%%%%%%%%%%%%%%%%%%%%%%%%%%%%%%%%
\subsection{動作環境}

FDPSが動作する環境は以下の通りである。
\begin{itemize}
\item Linuxのターミナル
\item Mac OS Xのターミナル
\item Windows のCygwin上のターミナル
\end{itemize}

%%%%%%%%%%%%%%%%%%%%%%%%%%%%%%%%%%%%%%%%%%%%%%%%%%%%%
\subsection{必要なソフトウェア}

本節では、FDPSを使用する際に必要となるソフトウェアを記述する。まず標準
機能を用いるのに必要なソフトウェア、次に拡張機能を用いるのに必要なソフ
トウェアを記述する。

\subsubsection{標準機能}

本節では、FDPSの標準機能のみを使用する際に必要なソフトウェアを記述する。
最初に逐次処理機能のみを用いる場合(並列処理機能を用いない場合)に必要
なソフトウェアを記述する。次に並列処理機能を用いる場合に必要なソフトウェ
アを記述する。

\subsubsubsection{逐次処理}

逐次処理の場合に必要なソフトウェアは以下の通りである。
\begin{itemize}
\item make
\item C++03以降のコンパイラ
\end{itemize}

\subsubsubsection{並列処理}

本節では、FDPSの並列処理機能を用いる際に必要なソフトウェアを記述する。
まず、OpenMPを使用する際に必要なソフトウェア、次にMPIを使用する際に必要
なソフトウェア、最後にOpenMPとMPIを同時に使用する際に必要なソフトウェア
を記述する。

\subsubsubsubsection{OpenMP}

OpenMPを使用する際に必要なソフトウェアは以下の通り。
\begin{itemize}
\item make
\item OpenMP対応のC++03以降のコンパイラ
\end{itemize}

\subsubsubsubsection{MPI}

MPIを使用する際に必要なソフトウェアは以下の通り。
\begin{itemize}
\item make
\item MPI対応のC++03以降のコンパイラ
\end{itemize}

\subsubsubsubsection{MPI+OpenMP}

MPIとOpenMPを同時に使用する際に必要なソフトウェアは以下の通り。
\begin{itemize}
\item make
\item MPIとOpenMPに対応のC++03以降のコンパイラ
\end{itemize}

\subsubsection{拡張機能}

本節では、FDPSの拡張機能を使用する際に必要なソフトウェアについて述べる。
FDPSの拡張機能にはParticle Meshがある。以下ではParticle Meshを使用する
際に必要なソフトウェアを述べる。

\subsubsubsection{Particle Mesh}

Particle Meshを使用する際に必要なソフトウェアは以下の通りである。
\begin{itemize}
\item make
\item MPIとOpenMP対応のC++03以降のコンパイラ
\item FFTW 3.3以降
\end{itemize}

%%%%%%%%%%%%%%%%%%%%%%%%%%%%%%%%%%%%%%%%%%%%%%%%%%%%%
\subsection{インストール}

本節では、FDPSのインストールについて述べる。取得方法、ビルド方法、サン
プルコードについて述べる。

\subsubsection{取得方法}

以下の方法のいずれかでFDPSの圧縮ファイル
\redtext{fdps-20xx-xx-xx.tar.gz}を取得できる。
\begin{itemize}
\item ウェブサイトhttps://github.com/FDPS/FDPSからダウンロード
\item \redtext{コマンドラインから}
\end{itemize}

FDPSを展開したいディレクトリに移動し、コマンドライン上で以下を実行する
と、圧縮ファイルを展開できる。
\begin{screen}
\begin{verbatim}
$ tar zxvf fdps-20xx-xx-xx.tar.gz
\end{verbatim}
\end{screen}
展開に成功すると、ディレクトリfdpsができる。

\subsubsection{ビルド方法}

ディレクトリfdpsに移動する。ここでconfigureをする必要はない。

\subsubsection{サンプルコードの使用方法}

本節ではサンプルコードの使用方法について記述する。サンプルコードには重
力$N$体シミュレーションコードと、SPHシミュレーションコードがある。最初
に重力$N$体シミュレーションコード、次にSPHシミュレーションコードの使用
について記述する。

\subsubsubsection{重力$N$体シミュレーションコード}

以下の手順で本コードを使用できる。
\begin{itemize}
\item ディレクトリfdps/sample/nbodyに移動
\item カレントディレクトリにあるMakefileを編集(後述)
\item コマンドライン上でmakeを実行
\item makeされた実行ファイルを実行(後述)
\end{itemize}

Makefileの編集項目は以下の通りである。OpenMPとMPIを使用するかどうかで編
集方法が変ることに注意。
\begin{itemize}
\item OpenMPもMPIも使用しない場合
  \begin{itemize}
  \item マクロCCにC++コンパイラを代入する
  \end{itemize}

\item OpenMPのみ使用の場合
  \begin{itemize}
  \item マクロCCにOpenMP対応のC++コンパイラを代入する
  \item "CFLAGS += -DPARTICLE\_SIMULATOR\_THREAD\_PARALLEL -fopenmp"の
    行のコメントアウトを外す(インテルコンパイラの場合は-fopenmpを外す)
  \end{itemize}

\item MPIのみ使用の場合
  \begin{itemize}
  \item マクロCCにMPIC++コンパイラを代入する
  \item "CFLAGS += -DPARTICLE\_SIMULATOR\_MPI\_PARALLEL"の行のコメント
    アウトを外す
  \end{itemize}

\item OpenMPとMPIの同時使用の場合
  \begin{itemize}
  \item マクロCCにMPI対応のC++コンパイラを代入する
  \item "CFLAGS += -DPARTICLE\_SIMULATOR\_MPI\_PARALLEL"の行のコメント
    アウトを外す
  \item "CFLAGS += -DPARTICLE\_SIMULATOR\_THREAD\_PARALLEL -fopenmp"の
    行のコメントアウトを外す(インテルコンパイラの場合は-fopenmpを外す)
  \end{itemize}

\end{itemize}

実行ファイルの実行方法は以下の通りである。
\begin{itemize}
\item MPIを使用しない場合、コマンドライン上で以下のコマンドを実行する
\begin{screen}
\begin{verbatim}
$ ./nbody
\end{verbatim}
\end{screen}
  
\item MPIを使用する場合、コマンドライン上で以下のコマンドを実行する
\begin{screen}
\begin{verbatim}
$ MPIRUN -np NPROC ./nbody
\end{verbatim}
\end{screen}
ここで、"MPIRUN"にはmpirunやmpiexecなどが、"NPROC"には使用するMPIプロセ
スの数が入る。

以下の手順でxy平面に射影した粒子分布の変化を見ることができる。
\begin{itemize}
\item gnuplotを起動する
\item plot "filename\_time.dat" using 1:2
\end{itemize}
出力ファイルフォーマットは1列目から順に粒子の位置のx, y, z座標、粒子の
x, y, z軸方向の速度、粒子質量である。

\end{itemize}

\subsubsubsection{SPHシミュレーションコード}

以下の手順で本コードを使用できる。
\begin{itemize}
\item ディレクトリfdps/sample/sphに移動
\item カレントディレクトリにあるMakefileを編集(後述)
\item コマンドライン上でmakeを実行
\item makeされた実行ファイルを実行(後述)
\end{itemize}

Makefileの編集項目は以下の通りである。OpenMPとMPIを使用するかどうかで編
集方法が変ることに注意。
\begin{itemize}
\item OpenMPもMPIも使用しない場合
  \begin{itemize}
  \item マクロCCにC++コンパイラを代入する
  \end{itemize}

\item OpenMPのみ使用の場合
  \begin{itemize}
  \item マクロCCにOpenMP対応のC++コンパイラを代入する
  \item "CFLAGS += -DPARTICLE\_SIMULATOR\_THREAD\_PARALLEL -fopenmp"の
    行のコメントアウトを外す(インテルコンパイラの場合は-fopenmpを外す)
  \end{itemize}

\item MPIのみ使用の場合
  \begin{itemize}
  \item マクロCCにMPIC++コンパイラを代入する
  \item "CFLAGS += -DPARTICLE\_SIMULATOR\_MPI\_PARALLEL"の行のコメント
    アウトを外す
  \end{itemize}

\item OpenMPとMPIの同時使用の場合
  \begin{itemize}
  \item マクロCCにMPI対応のC++コンパイラを代入する
  \item "CFLAGS += -DPARTICLE\_SIMULATOR\_MPI\_PARALLEL"の行のコメント
    アウトを外す
  \item "CFLAGS += -DPARTICLE\_SIMULATOR\_THREAD\_PARALLEL -fopenmp"の
    行のコメントアウトを外す(インテルコンパイラの場合は-fopenmpを外す)
  \end{itemize}

\end{itemize}

実行ファイルの実行方法は以下の通りである。
\begin{itemize}
\item MPIを使用しない場合、コマンドライン上で以下のコマンドを実行する
\begin{screen}
\begin{verbatim}
$ ./sph
\end{verbatim}
\end{screen}
  
\item MPIを使用する場合、コマンドライン上で以下のコマンドを実行する
\begin{screen}
\begin{verbatim}
$ MPIRUN -np NPROC ./sph
\end{verbatim}
\end{screen}
ここで、"MPIRUN"にはmpirunやmpiexecなどが、"NPROC"には使用するMPIプロセ
スの数が入る。
\end{itemize}

以下の手順でxy平面に射影した粒子分布の変化を見ることができる。
\begin{itemize}
\item gnuplotを起動する
\item plot "filename\_time.dat" using 1:2
\end{itemize}
出力ファイルフォーマットは1列目から順に粒子の位置のx, y, z座標、粒子の
x, y, z軸方向の速度、粒子質量、密度、内部エネルギーである。
