\subsubsection{概要}

EssentialParticleIクラスは相互作用の計算に必要なi粒子情報を持つクラス
である。EssentialParticleIクラスはFullParticleクラス
(節\ref{sec:fullparticle})のサブセットであり、FDPSの内部では、このクラ
スがFullParticleクラスから情報を読み取る。情報を読み取るために、このク
ラスはいくつかのメンバ関数を持つ必要がある。以下、この節の前提、常に必
要なメンバ関数と、場合によっては必要なメンバ関数について記述する。

\subsubsection{前提}

この節の中では、名前空間ParticleSimulatorをPSと省略する。このクラスの
クラス名をEssentialParticleIとする。また、粒子すべての情報を持つクラス
のクラス名をFullParticleとする。このFullParticleは
節\ref{sec:fullparticle}のクラスFullParticleと同一のものである。
EssentialParticleI, FullParticleというクラス名は変更可能である。

ParticleSimulatorをPSと省略すること、EssentialParticleIとFullParticle
の宣言は以下の通りである。
\begin{screen}
\begin{verbatim}
namespace PS = ParticleSimulator;
class FullParticle;
class EssentialParticleI;
\end{verbatim}
\end{screen}

\subsubsection{必要なメンバ関数}

\subsubsubsection{概要}

常に必要なメンバ関数はgetPosとCopyfromFPである。getPosは
EssentialParticleIクラスの位置情報をFDPSに読み込ませるための関数で、
copyFromFPはFullParticleクラスの情報をEssentialParticleIクラスに書きこ
む関数である。これらのメンバ関数の記述例と解説を以下に示す。

\subsubsubsection{getPos}

\begin{screen}
\begin{verbatim}
class EssentialParticleI {
public:
    PS::F64vec pos;
    PS::F64vec getPos() const {
        return this->pos;
    }
};
\end{verbatim}
\end{screen}

\begin{itemize}

\item {\bf 前提}
  
  EssentialParticleIのメンバ変数posはある1つの粒子の位置情報。このposのデー
  タ型はPS::F64vec型。
  
\item {\bf 引数}

  なし
  
\item {\bf 返値}

  PS::F64vec型。EssentialParticleIクラスの位置情報を保持したメンバ変数。
  
\item {\bf 機能}

  EssentialParticleIクラスの位置情報を保持したメンバ変数を返す。
  
\item {\bf 備考}

  EssentialParticleIクラスのメンバ変数posの変数名は変更可能。

\end{itemize}

\subsubsubsection{copyFromFP}

\begin{screen}
\begin{verbatim}
class FullParticle {
public:
    PS::S64    identity;
    PS::F64    mass;
    PS::F64vec position;
    PS::F64vec velocity;
    PS::F64vec acceleration;
    PS::F64    potential;
};
class EssentialParticleI {
public:
    PS::S64    id;
    PS::F64vec pos;
    void copyFromFP(const FullParticle & fp) {
        this->id  = fp.identity;
        this->pos = fp.position;
    }
};
\end{verbatim}
\end{screen}

\begin{itemize}

\item {\bf 前提}

  FullParticleクラスのメンバ変数identity, positionとEssentialParticleI
  クラスのメンバ変数id, posはそれぞれ対応する情報を持つ。

\item {\bf 引数}

  fp: 入力。const FullParticle \&型。FullParticleクラスの情報を持つ。
  
\item {\bf 返値}

  なし。
  
\item {\bf 機能}

  FullParticleクラスの持つ1粒子の情報の一部をEssnetialParticleIクラス
  に書き込む。
  
\item {\bf 備考}

  FullParticleクラスのメンバ変数の変数名、EssentialParticleIクラスのメ
  ンバ変数の変数名は変更可能。メンバ関数EssentialParticleI::copyFromFP
  の引数名は変更可能。EssentialParticleIクラスの粒子情報はFullParticle
  クラスの粒子情報のサブセット。対応する情報を持つメンバ変数同士のデー
  タ型が一致している必要はないが、実数型とベクトル型(または整数型とベ
  クトル型)という違いがある場合に正しく動作する保証はない。

\end{itemize}

\subsubsection{場合によっては必要なメンバ関数}

\subsubsubsection{概要}

本節では、場合によっては必要なメンバ関数について記述する。相互作用ツリー
クラスのSEARCH\_MODE型にSEARCH\_MODE\_GATHERまたは
SEARCH\_MODE\_SYMMETRYを用いる場合に必要となるメンバ関数ついて記述
する。

\subsubsubsection{相互作用ツリークラスのSEARCH\_MODE型にSEARCH\_MODE\_GATHERまたはSEARCH\_MODE\_SYMMETRYを用いる場合}

\subsubsubsubsection{getRsearch}

\begin{screen}
\begin{verbatim}
class EssentialParticleI {
public:
    PS::F64 search_radius;
    PS::F64 getRsearch() const {
        return this->search_radius;
    }
};
\end{verbatim}
\end{screen}

\begin{itemize}

\item {\bf 前提}

  EssentialParticleIクラスのメンバ変数search\_radiusはある1つの粒子の
  近傍粒子を探す半径の大きさ。このsearch\_radiusのデータ型はPS::F32型
  またはPS::F64型。
  
\item {\bf 引数}

  なし
  
\item {\bf 返値}

  PS::F32型またはPS::F64型。 EssentialParticleIクラスの近傍粒子を探す
  半径の大きさを保持したメンバ変数。
  
\item {\bf 機能}

  EssentialParticleIクラスの近傍粒子を探す半径の大きさを保持したメンバ
  変数を返す。

\item {\bf 備考}

  EssentialParticleIクラスのメンバ変数search\_radiusの変数名は変更可能。
  
\end{itemize}
