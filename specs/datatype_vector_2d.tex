PS::Vector2はx, yの2要素を持つ。これらに対する様々なAPIや演算子を定義
した。それらの宣言を以下に記述する。この節ではこれらについて詳しく記述
する。
\begin{lstlisting}[caption=Vector2]
namespace ParticleSimulator{
    template <typename T>
    class Vector2{
    public:
        //メンバ変数2要素
        T x, y;

        //コンストラクタ
        Vector2();
        Vector2(const T _x, const T _y) : x(_x), y(_y) {}
        Vector2(const T s) : x(s), y(s) {}
        Vector2(const Vector2 & src) : x(src.x), y(src.y) {}

        //代入演算子
        const Vector2 & operator = (const Vector2 & rhs);

        //加減算
        Vector2 operator + (const Vector2 & rhs) const;
        const Vector2 & operator += (const Vector2 & rhs);
        Vector2 operator - (const Vector2 & rhs) const;
        const Vector2 & operator -= (const Vector2 & rhs);

        //ベクトルスカラ積
        Vector2 operator * (const T s) const;
        const Vector2 & operator *= (const T s);
        friend Vector2 operator * (const T s, const Vector2 & v);
        Vector2 operator / (const T s) const;
        const Vector2 & operator /= (const T s);

        //内積
        T operator * (const Vector2 & rhs) const;

        //外積(返り値はスカラ!!)
        T operator ^ (const Vector2 & rhs) const;

        //Vector2<U>への型変換
        template <typename U>
        operator Vector2<U> () const;
    };
}
namespace PS = ParticleSimulator;
\end{lstlisting}

\subsubsubsection{コンストラクタ}

\begin{screen}
\begin{verbatim}
template<typename T>
PS::Vector2<T>()
\end{verbatim}
\end{screen}

\begin{itemize}

\item{{\bf 引数}}

なし。

\item{{\bf 機能}}

デフォルトコンストラクタ。メンバx,yは0で初期化される。

\end{itemize}

\begin{screen}
\begin{verbatim}
template<typename T>
PS::Vector2<T>(const T _x, const T _y)
\end{verbatim}
\end{screen}

\begin{itemize}

\item{{\bf 引数}}

{\tt \_x}: 入力。{\tt const T}型。

{\tt \_y}: 入力。{\tt const T}型。

\item{{\bf 機能}}

メンバ{\tt x}、{\tt y}をそれぞれ{\tt \_x}、{\tt \_y}で初期化する。

\end{itemize}

\begin{screen}
\begin{verbatim}
template<typename T>
PS::Vector2<T>(const T s);
\end{verbatim}
\end{screen}

\begin{itemize}

\item{{\bf 引数}}

{\tt s}: 入力。{\tt const T}型。

\item{{\bf 機能}}

メンバ{\tt x}、{\tt y}を両方とも{\tt s}の値で初期化する。

\end{itemize}

\begin{screen}
\begin{verbatim}
template<typename T>
PS::Vector2<T>(const PS::Vector2<T> & src)
\end{verbatim}
\end{screen}

\begin{itemize}

\item{{\bf 引数}}

{\tt src}: 入力。{\tt const PS::Vector2<T> \&}型。

\item{{\bf 機能}}

コピーコンストラクタ。{\tt src}で初期化する。

\end{itemize}

\subsubsubsection{代入演算子}

\begin{screen}
\begin{verbatim}
template<typename T>
const PS::Vector2<T> & PS::Vector2<T>::operator = 
                       (const PS::Vector2<T> & rhs);
\end{verbatim}
\end{screen}

\begin{itemize}

\item{{\bf 引数}}

{\tt rhs}: 入力。{\tt const PS::Vector2<T> \&}型。

\item{{\bf 返り値}}

{\tt const PS::Vector2<T> \&}型。{\tt rhs}のx,yの値を自身のメンバx,yに
代入し自身の参照を返す。代入演算子。

\end{itemize}

\subsubsubsection{加減算}

\begin{screen}
\begin{verbatim}
template<typename T>
PS::Vector2<T> PS::Vector2<T>::operator + 
               (const PS::Vector2<T> & rhs) const;
\end{verbatim}
\end{screen}

\begin{itemize}

\item{{\bf 引数}}

{\tt rhs}: 入力。{\tt const PS::Vector2<T> \&}型。

\item{{\bf 返り値}}

{\tt PS::Vector2<T> }型。{\tt rhs}のx,yの値と自身のメンバx,yの値の和を
取った値を返す。

\end{itemize}

\begin{screen}
\begin{verbatim}
template<typename T>
const PS::Vector2<T> & PS::Vector2<T>::operator += 
                       (const PS::Vector2<T> & rhs);
\end{verbatim}
\end{screen}

\begin{itemize}

\item{{\bf 引数}}

{\tt rhs}: 入力。{\tt const PS::Vector2<T> \&}型。

\item{{\bf 返り値}}

{\tt const PS::Vector2<T> \&}型。{\tt rhs}のx,yの値を自身のメンバx,yに足し、自
身を返す。

\end{itemize}

\begin{screen}
\begin{verbatim}
template<typename T>
PS::Vector2<T> PS::Vector2<T>::operator - 
               (const PS::Vector2<T> & rhs) const;
\end{verbatim}
\end{screen}

\begin{itemize}

\item{{\bf 引数}}

{\tt rhs}: 入力。{\tt const PS::Vector2<T> \&}型。

\item{{\bf 返り値}}

{\tt PS::Vector2<T> }型。{\tt rhs}のx,yの値と自身のメンバx,yの値の差を
取った値を返す。

\end{itemize}

\begin{screen}
\begin{verbatim}
template<typename T>
const PS::Vector2<T> & PS::Vector2<T>::operator -= 
                       (const PS::Vector2<T> & rhs);
\end{verbatim}
\end{screen}

\begin{itemize}

\item{{\bf 引数}}

{\tt rhs}: 入力。{\tt const PS::Vector2<T> \&}型。

\item{{\bf 返り値}}

{\tt const PS::Vector2<T> \&}型。自身のメンバx,yから{\tt rhs}のx,yを引
き自身を返す。

\end{itemize}

\subsubsubsection{ベクトルスカラ積}

\begin{screen}
\begin{verbatim}
template<typename T>
PS::Vector2<T> PS::Vector2<T>::operator * (const T s) const;
\end{verbatim}
\end{screen}

\begin{itemize}

\item{{\bf 引数}}

{\tt s}: 入力。{\tt const T}型。

\item{{\bf 返り値}}

{\tt PS::Vector2<T>}型。自身のメンバx,yそれぞれに{\tt s}をかけた値を返
す。

\end{itemize}

\begin{screen}
\begin{verbatim}
template<typename T>
const PS::Vector2<T> & PS::Vector2<T>::operator *= (const T s);
\end{verbatim}
\end{screen}

\begin{itemize}

\item{{\bf 引数}}

{\tt rhs}: 入力。{\tt const T}型。

\item{{\bf 返り値}}

{\tt const PS::Vector2<T> \&}型。自身のメンバx,yそれぞれに{\tt s}をかけ
自身を返す。

\end{itemize}

\begin{screen}
\begin{verbatim}
template<typename T>
PS::Vector2<T> PS::Vector2<T>::operator / (const T s) const;
\end{verbatim}
\end{screen}

\begin{itemize}

\item{{\bf 引数}}

{\tt s}: 入力。{\tt const T}型。

\item{{\bf 返り値}}

{\tt PS::Vector2<T>}型。自身のメンバx,yそれぞれを{\tt s}で割った値を返
す。

\end{itemize}

\begin{screen}
\begin{verbatim}
template<typename T>
const PS::Vector2<T> & PS::Vector2<T>::operator /= (const T s);
\end{verbatim}
\end{screen}

\begin{itemize}

\item{{\bf 引数}}

{\tt rhs}: 入力。{\tt const T}型。

\item{{\bf 返り値}}

{\tt const PS::Vector2<T> \&}型。自身のメンバx,yそれぞれを{\tt s}で割り
自身を返す。

\end{itemize}

\subsubsubsection{内積、外積}

\begin{screen}
\begin{verbatim}
template<typename T>
T PS::Vector2<T>::operator * (const PS::Vector2<T> & rhs) const;
\end{verbatim}
\end{screen}

\begin{itemize}

\item{{\bf 引数}}

{\tt rhs}: 入力。{\tt const PS::Vector2<T> \&}型。

\item{{\bf 返り値}}

{\tt T}型。自身と{\tt rhs}の内積を取った値を返す。

\end{itemize}

\begin{screen}
\begin{verbatim}
template<typename T>
T PS::Vector2<T>::operator ^ (const PS::Vector2<T> & rhs) const;
\end{verbatim}
\end{screen}

\begin{itemize}

\item{{\bf 引数}}

{\tt rhs}: 入力。{\tt const PS::Vector2<T> \&}型。

\item{{\bf 返り値}}

{\tt T}型。自身と{\tt rhs}の外積を取った値を返す。

\end{itemize}

\subsubsubsection{{\tt Vector2<U>}への型変換}

\begin{screen}
\begin{verbatim}
template<typename T>
template <typename U>
PS::Vector2<T>::operator PS::Vector2<U> () const;
\end{verbatim}
\end{screen}

\begin{itemize}

\item{{\bf 引数}}

  なし。

\item{{\bf 返り値}}

  {\tt const PS::Vector2<U>}型。

\item{{\bf 機能}}

  {\tt const PS::Vector2<T>}型を{\tt const PS::Vector2<U>}型にキャ
  ストする。

\end{itemize}


