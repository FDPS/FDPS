\subsection{概要}

本節では、ユーザーが定義するクラスとファンクタについて記述する。ユーザー
定義クラスとなるのは、粒子の情報すべてを持つFullParticleクラス、ある相
互作用を計算する際にi粒子に必要な情報を持つEssentialParticleIクラス、
ある相互作用を計算する際にj粒子に必要な情報を持つEssentialParticleJク
ラス、ある相互作用(SEARCH\_MODE型がSEARCH\_MODE\_LONGまたは
SEARCH\_MODE\_LONG\_CUTOFFの場合に限る)を計算する際に超粒子に必要な情
報を持つSuperParticleJクラス、ツリーセルのモーメント情報を持つMomentク
ラス、相互作用の結果の情報を持つForceクラス、入出力ファイルのヘッダ情
報を持つヘッダクラスである。また、ユーザー定義ファンクタには、j粒子か
らi粒子への作用を計算するcalcForceEpEpファンクタ、超粒子からi粒子への
作用を計算するcalcForceSpEpファンクタがある。

この節で記述するのは、これらのクラスに関する規定である。ユーザーはこれ
らのクラスの間でのデータのやりとりや、ファンクタ内でのデータの加工につ
いてコードに書く必要がある。これらは上に挙げたクラスとファンクタのメン
バ関数内で行われる。以下、必要なメンバ関数とその規定について記述する。

\subsection{FullParticleクラス}
\label{sec:fullparticle}

\subsubsection{概要}

FullParticleクラスは粒子情報すべてを持つクラスである。FDPSはこのクラス
からいくつかの情報を読み取る。FDPSが情報を読み取るために、このクラスは
いくつかのメンバ関数を持つ必要がある。以下、この節の前提、常に必要なメ
ンバ関数と、場合によっては必要なメンバ関数について記述する。

\subsubsection{前提}

この節の中では、以下のように、名前空間ParticleSimulatorをPSと省略し、
FullParticleというクラスを例とする。FullParticleという名前は自由に変え
ることができる。
\begin{screen}
\begin{verbatim}
namespace PS = ParticleSimulator;
class FullParticle;
\end{verbatim}
\end{screen}

\subsubsection{必要なメンバ関数}

\subsubsubsection{概要}

常に必要なメンバ関数はgetPosとcopyFromForceである。getPosは
FullParticleの位置情報をFDPSに読み込ませるための関数で、copyFromForce
は計算された相互作用の結果をFullParticleに書き戻す関数である。これらの
メンバ関数の記述例と解説を以下に示す。

\subsubsubsection{getPos}

\begin{screen}
\begin{verbatim}
class FullParticle {
public:
    PS::F64vec pos;
    PS::F64vec getPos() const {
        return this->pos;
    }
};
\end{verbatim}
\end{screen}

\begin{itemize}

\item {\bf 前提}
  
  FullParticleのメンバ変数posはある1つの粒子の位置情報。このposのデー
  タ型はPS::F32vec型またはPS::F64vec型。
  
\item {\bf 引数}

  なし
  
\item {\bf 返値}

  PS::F32vec型またはPS::F64vec型。FullParticleクラスの位置情報を保持し
  たメンバ変数。
  
\item {\bf 機能}

  FullParticleクラスの位置情報を保持したメンバ変数を返す。
  
\item {\bf 備考}

  FullParticleクラスのメンバ変数posの変数名は変更可能。ただしこのposの
  データ型とメンバ関数FullParticle::getPosの返値のデータ型が一致してい
  ない場合の動作は保証しない。

\end{itemize}

\subsubsubsection{copyFromForce}

\begin{screen}
\begin{verbatim}
class Force {
public:
    PS::F64vec acc;
    PS::F64    pot;
};
class FullParticle {
public:
    PS::F64vec acceleration;
    PS::F64    potential;
    void copyFromForce(const Force & force) {
        this->acceleration = force.acc;
        this->potential    = force.pot;
    }
};
\end{verbatim}
\end{screen}

\begin{itemize}

\item {\bf 前提}

  Forceクラスは粒子の相互作用の計算結果を保持するクラス。

\item {\bf 引数}

  force: 入力。const Force \&型。粒子の相互作用の計算結果を保持。
  
\item {\bf 返値}

  なし。
  
\item {\bf 機能}

  粒子の相互作用の計算結果をFullParticleクラスへ書き戻す。Forceクラス
  のメンバ変数acc, potがそれぞれFullParticleクラスのメンバ変数
  acceleration, potentialに対応。
  
\item {\bf 備考}

  Forceクラスというクラス名とそのメンバ変数名は変更可能。FullParticle
  のメンバ変数名は変更可能。メンバ関数FullParticle::copyFromForceの引
  数名は変更可能。

\end{itemize}

\subsubsection{場合によっては必要なメンバ関数}

\subsubsubsection{概要}

本節では、場合によっては必要なメンバ関数について記述する。相互作用ツリー
クラスのSEARCH\_MODE型にSEARCH\_MODE\_LONG以外を用いる場合、粒子群クラ
スのファイル入出力APIを用いる場合、粒子群クラスのAPIである
ParticleSystem::adjustPositionIntoRootDomainを用いる場合、拡張機能の
Particle Meshクラスを用いる場合について必要となるメンバ関数を記述する。

\subsubsubsection{相互作用ツリークラスのSEARCH\_MODE型にSEARCH\_MODE\_LONG以外を用いる場合}

\subsubsubsubsection{getRsearch}

\begin{screen}
\begin{verbatim}
class FullParticle {
public:
    PS::F64 search_radius;
    PS::F64 getRsearch() const {
        return this->search_radius;
    }
};
\end{verbatim}
\end{screen}

\begin{itemize}

\item {\bf 前提}

  FullParticleクラスのメンバ変数search\_radiusはある1つの粒子の近傍粒
  子を探す半径の大きさ。このsearch\_radiusのデータ型はPS::F32型または
  PS::F64型。
  
\item {\bf 引数}

  なし
  
\item {\bf 返値}

  PS::F32型またはPS::F64型。 FullParticleクラスの近傍粒子を探す半径の
  大きさを保持したメンバ変数。
  
\item {\bf 機能}

  FullParticleクラスの近傍粒子を探す半径の大きさを保持したメンバ変数を
  返す。

\item {\bf 備考}

  FullParticleクラスのメンバ変数search\_radiusの変数名は変更可能。
  
\end{itemize}

\subsubsubsection{粒子群クラスのファイル入出力APIを用いる場合}

粒子群クラスのファイル入出力APIであるreadParticleAscii,
readParticleBinary, writeParticleAscii, writeParticleBinaryを使用する
ときにそれぞれreadAscii, readBinary, writeAscii, writeBinaryというメン
バ関数が必要となる。以下、readAsciiとreadBinaryの規定は同じであり,
writeAsciiとwriteBinaryの規定も同じである。以下、それぞれの規定につい
て記述する。

\subsubsubsubsection{readAscii, readBinary}

\begin{screen}
\begin{verbatim}
class FullParticle {
public:
    PS::S32 id;
    PS::F64 mass;
    PS::F64vec pos;
    void readAscii(FILE *fp) {
        fscanf(fp, "%d%lf%lf%lf%lf", &this->id, &this->mass,
               &this->pos[0], &this->pos[1], &this->pos[2]);
    }
    void readBinary(FILE *fp) {
        fscanf(fp, "%d%lf%lf%lf%lf", &this->id, &this->mass,
               &this->pos[0], &this->pos[1], &this->pos[2]);
    }
};
\end{verbatim}
\end{screen}

\begin{itemize}

\item {\bf 前提}

  粒子データの入力ファイルの1列目にはFullParticleクラスのメンバ変数id
  を表すデータが、2列目にはメンバ変数massを表すデータが、3、4、5列
  めにはメンバ変数posの第1、2、3要素が、それ以降の列にはデータがな
  いとする。ファイルの形式はアスキー形式(readAsciiの場合)、バイナリー
  形式(readBinaryの場合)とする。3次元直交座標系を選択したとする。

\item {\bf 引数}

  fp: FILE *型。粒子データの入力ファイルを指すファイルポインタ。
  
\item {\bf 返値}

  なし。
  
\item {\bf 機能}

  粒子データの入力ファイルからFullParticleクラスのid、mass、posの情報
  を読み取る。
  
\item {\bf 備考}

  なし。
  
\end{itemize}

\subsubsubsubsection{writeAscii, writeBinary}

\begin{screen}
\begin{verbatim}
class FullParticle {
public:
    PS::S32 id;
    PS::F64 mass;
    PS::F64vec pos;
    void writeAscii(FILE *fp) {
        fscanf(fp, "%d %lf %lf %lf %lf", this->id, this->mass,
               this->pos[0], this->pos[1], this->pos[2]);
    }
    void writeBinary(FILE *fp) {
        fscanf(fp, "%d %lf %lf %lf %lf", this->id, this->mass,
               this->pos[0], this->pos[1], this->pos[2]);
    }
};
\end{verbatim}
\end{screen}

\begin{itemize}

\item {\bf 前提}

  粒子データの出力ファイルの1列目にはFullParticleクラスのメンバ変数id
  を表すデータが、2列目にはメンバ変数massを表すデータが、3、4、5列
  めにはメンバ変数posの第1、2、3要素が、それ以降の列にはデータがな
  いとする。ファイルの形式はアスキー形式(writeAsciiの場合)、バイナリー
  形式(writeBinaryの場合)とする。3次元直交座標系を選択したとする。

\item {\bf 引数}

  fp: FILE *型。粒子データの出力ファイルを指すファイルポインタ。
  
\item {\bf 返値}

  なし。
  
\item {\bf 機能}

  粒子データの出力ファイルへFullParticleクラスのメンバ変数id、mass、
  posの情報を書き出す。
  
\item {\bf 備考}

  なし。
  
\end{itemize}

\subsubsubsection{粒子群クラスのadjustPositionIntoRootDomainを用いる場合}

\subsubsubsubsection{setPos}

\begin{screen}
\begin{verbatim}
class FullParticle {
public:
    PS::F64vec pos;
    void setPos(const PS::F64vec pos_new) {
        this->pos = pos_new;
    }
};
\end{verbatim}
\end{screen}

\begin{itemize}

\item {\bf 前提}

  FullParticleクラスのメンバ変数posは1つの粒子の位置情報。このposのデー
  タ型はPS::F32vecまたはPS::F64vec。

\item {\bf 引数}

  pos\_new: 入力。const PS::F32vecまたはconst PS::F64vec型。FDPS側で修
  正した粒子の位置情報。

\item {\bf 返値}

  なし。
  
\item {\bf 機能}

  FDPSが修正した粒子の位置情報をFullParticleクラスの位置情報に書き込む。

\item {\bf 備考}

  FullParticleクラスのメンバ変数posの変数名は変更可能。メンバ関数
  FullParticle::setPosの引数名pos\_newは変更可能。posとpos\_newのデー
  タ型が異なる場合の動作は保証しない。

\end{itemize}

\subsubsubsection{Particle Meshクラスを用いる場合}

Particle Meshクラスを用いる場合には、メンバ関数getChargeParticleMeshと
copyFromForceParticleMeshを用意する必要がある。以下にそれぞれの規定を
記述する。

\subsubsubsubsection{getChargeParticleMesh}

\begin{screen}
\begin{verbatim}
class FullParticle {
public:
    PS::F64 mass;
    PS::F64 getChargeParticleMesh() const {
        return this->mass;
    }
};
\end{verbatim}
\end{screen}

\begin{itemize}

\item {\bf 前提}

  FullParticleクラスのメンバ変数massは1つの粒子の質量または電荷の情報
  を持つ変数。データ型はPS::F32またはPS::F64型。

\item {\bf 引数}

  なし。

\item {\bf 返値}

  PS::F32型またはPS::F64型。1つの粒子の質量または電荷の変数を返す。
  
\item {\bf 機能}

  1つの粒子の質量または電荷の変数を返す。

\item {\bf 備考}

  FullParticleクラスのメンバ変数massの変数名は変更可能。

\end{itemize}

\subsubsubsubsection{copyFromForceParticleMesh}

\begin{screen}
\begin{verbatim}
class FullParticle {
public:
    PS::F64vec accelerationFromPM;
    void copyFromForceParticleMesh(const PS::F32vec & acc_pm) {
        this->accelerationFromPM = acc_pm;
    }
};
\end{verbatim}
\end{screen}

\begin{itemize}

\item {\bf 前提}

  FullParticleクラスのメンバ変数accelerationFromPM\_pmは1つの粒子の
  Particle Meshによる力の情報を保持する変数。この
  accelerationFromPM\_pmのデータ型はPS::F32vecまたはPS::F64vec。

\item {\bf 引数}

  acc\_pm: const PS::F32vec型またはconst PS::F64vec型。1つの粒子の
  Particle Meshによる力の計算結果。

\item {\bf 返値}

  なし。
  
\item {\bf 機能}

  1つの粒子のParticle Meshによる力の計算結果をこの粒子のメンバ変数に
  書き込む。
  
\item {\bf 備考}

  FullParticleクラスのメンバ変数acc\_pmの変数名は変更可能。メンバ関数
  FullParticle::copyFromForceParticleMeshの引数acc\_pmの引数名は変更可
  能。

\end{itemize}


\subsection{EssentialParticleIクラス}
\label{sec:essentialparticlei}

\subsubsection{概要}

EssentialParticleIクラスは相互作用の計算に必要なi粒子情報を持つクラス
である。EssentialParticleIクラスはFullParticleクラス
(節\ref{sec:fullparticle})のサブセットであり、FDPSの内部では、このクラ
スがFullParticleクラスから情報を読み取る。情報を読み取るために、このク
ラスはいくつかのメンバ関数を持つ必要がある。以下、この節の前提、常に必
要なメンバ関数と、場合によっては必要なメンバ関数について記述する。

\subsubsection{前提}

この節の中では、名前空間ParticleSimulatorをPSと省略する。このクラスの
クラス名をEssentialParticleIとする。また、粒子すべての情報を持つクラス
のクラス名をFullParticleとする。このFullParticleは
節\ref{sec:fullparticle}のクラスFullParticleと同一のものである。
EssentialParticleI, FullParticleというクラス名は変更可能である。

ParticleSimulatorをPSと省略すること、EssentialParticleIとFullParticle
の宣言は以下の通りである。
\begin{screen}
\begin{verbatim}
namespace PS = ParticleSimulator;
class FullParticle;
class EssentialParticleI;
\end{verbatim}
\end{screen}

\subsubsection{必要なメンバ関数}

\subsubsubsection{概要}

常に必要なメンバ関数はgetPosとCopyfromFPである。getPosは
EssentialParticleIクラスの位置情報をFDPSに読み込ませるための関数で、
copyFromFPはFullParticleクラスの情報をEssentialParticleIクラスに書きこ
む関数である。これらのメンバ関数の記述例と解説を以下に示す。

\subsubsubsection{getPos}

\begin{screen}
\begin{verbatim}
class EssentialParticleI {
public:
    PS::F64vec pos;
    PS::F64vec getPos() const {
        return this->pos;
    }
};
\end{verbatim}
\end{screen}

\begin{itemize}

\item {\bf 前提}
  
  EssentialParticleIのメンバ変数posはある1つの粒子の位置情報。このposのデー
  タ型はPS::F64vec型。
  
\item {\bf 引数}

  なし
  
\item {\bf 返値}

  PS::F64vec型。EssentialParticleIクラスの位置情報を保持したメンバ変数。
  
\item {\bf 機能}

  EssentialParticleIクラスの位置情報を保持したメンバ変数を返す。
  
\item {\bf 備考}

  EssentialParticleIクラスのメンバ変数posの変数名は変更可能。

\end{itemize}

\subsubsubsection{copyFromFP}

\begin{screen}
\begin{verbatim}
class FullParticle {
public:
    PS::S64    identity;
    PS::F64    mass;
    PS::F64vec position;
    PS::F64vec velocity;
    PS::F64vec acceleration;
    PS::F64    potential;
};
class EssentialParticleI {
public:
    PS::S64    id;
    PS::F64vec pos;
    void copyFromFP(const FullParticle & fp) {
        this->id  = fp.identity;
        this->pos = fp.position;
    }
};
\end{verbatim}
\end{screen}

\begin{itemize}

\item {\bf 前提}

  FullParticleクラスのメンバ変数identity, positionとEssentialParticleI
  クラスのメンバ変数id, posはそれぞれ対応する情報を持つ。

\item {\bf 引数}

  fp: 入力。const FullParticle \&型。FullParticleクラスの情報を持つ。
  
\item {\bf 返値}

  なし。
  
\item {\bf 機能}

  FullParticleクラスの持つ1粒子の情報の一部をEssnetialParticleIクラス
  に書き込む。
  
\item {\bf 備考}

  FullParticleクラスのメンバ変数の変数名、EssentialParticleIクラスのメ
  ンバ変数の変数名は変更可能。メンバ関数EssentialParticleI::copyFromFP
  の引数名は変更可能。EssentialParticleIクラスの粒子情報はFullParticle
  クラスの粒子情報のサブセット。対応する情報を持つメンバ変数同士のデー
  タ型が一致している必要はないが、実数型とベクトル型(または整数型とベ
  クトル型)という違いがある場合に正しく動作する保証はない。

\end{itemize}

\subsubsection{場合によっては必要なメンバ関数}

\subsubsubsection{概要}

本節では、場合によっては必要なメンバ関数について記述する。相互作用ツリー
クラスのSEARCH\_MODE型にSEARCH\_MODE\_GATHERまたは
SEARCH\_MODE\_SYMMETRYを用いる場合に必要となるメンバ関数ついて記述
する。

\subsubsubsection{相互作用ツリークラスのSEARCH\_MODE型にSEARCH\_MODE\_GATHERまたはSEARCH\_MODE\_SYMMETRYを用いる場合}

\subsubsubsubsection{getRsearch}

\begin{screen}
\begin{verbatim}
class EssentialParticleI {
public:
    PS::F64 search_radius;
    PS::F64 getRsearch() const {
        return this->search_radius;
    }
};
\end{verbatim}
\end{screen}

\begin{itemize}

\item {\bf 前提}

  EssentialParticleIクラスのメンバ変数search\_radiusはある1つの粒子の
  近傍粒子を探す半径の大きさ。このsearch\_radiusのデータ型はPS::F32型
  またはPS::F64型。
  
\item {\bf 引数}

  なし
  
\item {\bf 返値}

  PS::F32型またはPS::F64型。 EssentialParticleIクラスの近傍粒子を探す
  半径の大きさを保持したメンバ変数。
  
\item {\bf 機能}

  EssentialParticleIクラスの近傍粒子を探す半径の大きさを保持したメンバ
  変数を返す。

\item {\bf 備考}

  EssentialParticleIクラスのメンバ変数search\_radiusの変数名は変更可能。
  
\end{itemize}

\subsection{EssentialParticleJクラス}
\label{sec:essentialparticlej}

\subsubsection{概要}

EssentialParticleJクラスは相互作用の計算に必要なj粒子情報を持つクラス
である。FDPSの内部では、このクラスがFullParticleクラスから情報を読み取
る。情報を読み取るために、このクラスはいくつかのメンバ関数を持つ必要が
ある。以下、この節の前提、常に必要なメンバ関数と、場合によっては必要な
メンバ関数について記述する。

\subsubsection{前提}

この節の中では、名前空間ParticleSimulatorをPSと省略する。このクラスの
クラス名をEssentialParticleJとする。また、粒子すべての情報を持つクラス
のクラス名をFullParticleとする。このFullParticleは
節\ref{sec:fullparticle}のクラスFullParticleと同一のものである。
EssentialParticleJ, FullParticleというクラス名は変更可能である。

ParticleSimulatorをPSと省略すること、EssentialParticleJとFullParticle
の宣言は以下の通りである。
\begin{screen}
\begin{verbatim}
namespace PS = ParticleSimulator;
class FullParticle;
class EssentialParticleJ;
\end{verbatim}
\end{screen}

\subsubsection{必要なメンバ関数}

\subsubsubsection{概要}

常に必要なメンバ関数はgetPosとCopyfromFPである。getPosは
EssentialParticleJクラスの位置情報をFDPSに読み込ませるための関数で、
copyFromFPはFullParticleクラスの情報をEssentialParticleJクラスに書きこ
む関数である。これらのメンバ関数の記述例と解説を以下に示す。

\subsubsubsection{getPos}

\begin{screen}
\begin{verbatim}
class EssentialParticleJ {
public:
    PS::F64vec pos;
    PS::F64vec getPos() const {
        return this->pos;
    }
};
\end{verbatim}
\end{screen}

\begin{itemize}

\item {\bf 前提}
  
  EssentialParticleJのメンバ変数posはある1つの粒子の位置情報。このposのデー
  タ型はPS::F64vec型。
  
\item {\bf 引数}

  なし
  
\item {\bf 返値}

  PS::F64vec型。EssentialParticleJクラスの位置情報を保持したメンバ変数。
  
\item {\bf 機能}

  EssentialParticleJクラスの位置情報を保持したメンバ変数を返す。
  
\item {\bf 備考}

  EssentialParticleJクラスのメンバ変数posの変数名は変更可能。

\end{itemize}

\subsubsubsection{copyFromFP}

\begin{screen}
\begin{verbatim}
class FullParticle {
public:
    PS::S64    identity;
    PS::F64    mass;
    PS::F64vec position;
    PS::F64vec velocity;
    PS::F64vec acceleration;
    PS::F64    potential;
};
class EssentialParticleJ {
public:
    PS::S64    id;
    PS::F64    m;
    PS::F64vec pos;
    void copyFromFP(const FullParticle & fp) {
        this->id  = fp.identity;
        this->m   = fp.mass;
        this->pos = fp.position;
    }
};
\end{verbatim}
\end{screen}

\begin{itemize}

\item {\bf 前提}

  FullParticleクラスのメンバ変数identity, mass, positionと
  EssentialParticleJクラスのメンバ変数id, m, posはそれぞれ対応する情報
  を持つ。

\item {\bf 引数}

  fp: 入力。const FullParticle \&型。FullParticleクラスの情報を持つ。
  
\item {\bf 返値}

  なし。
  
\item {\bf 機能}

  FullParticleクラスの持つ1粒子の情報の一部をEssnetialParticleJクラス
  に書き込む。
  
\item {\bf 備考}

  Fullparticeクラスのメンバ変数の変数名、EssentialParticleJクラスのメ
  ンバ変数の変数名は変更可能。メンバ関数EssentialParticleJ::copyFromFP
  の引数名は変更可能。EssentialParticleJクラスの粒子情報はFullParticle
  クラスの粒子情報のサブセット。対応する情報を持つメンバ変数同士のデー
  タ型が一致している必要はないが、実数型とベクトル型(または整数型とベ
  クトル型)という違いがある場合に正しく動作する保証はない。

\end{itemize}

\subsubsection{場合によっては必要なメンバ関数}

\subsubsubsection{概要}

本節では、場合によっては必要なメンバ関数について記述する。相互作用ツリー
クラスのSEARCH\_MODE型にSEARCH\_MODE\_LONG以外を用いる場合に必要なメン
バ関数、列挙型のBOUNDARY\_CONDITION型にPS::BOUNDARY\_CONDITION\_OPEN以
外を選んだ場合に必要となるメンバ関数について記述する。

\subsubsubsection{相互作用ツリークラスのSEARCH\_MODE型にSEARCH\_MODE\_LONG以外を用いる場合}

\subsubsubsubsection{getRsearch}

\begin{screen}
\begin{verbatim}
class EssentialParticleJ {
public:
    PS::F64 search_radius;
    PS::F64 getRsearch() const {
        return this->search_radius;
    }
};
\end{verbatim}
\end{screen}

\begin{itemize}

\item {\bf 前提}

  EssentialParticleJクラスのメンバ変数search\_radiusはある1つの粒子の
  近傍粒子を探す半径の大きさ。このsearch\_radiusのデータ型はPS::F32型
  またはPS::F64型。
  
\item {\bf 引数}

  なし
  
\item {\bf 返値}

  PS::F32型またはPS::F64型。 EssentialParticleJクラスの近傍粒子を探す
  半径の大きさを保持したメンバ変数。
  
\item {\bf 機能}

  EssentialParticleJクラスの近傍粒子を探す半径の大きさを保持したメンバ
  変数を返す。

\item {\bf 備考}

  EssentialParticleJクラスのメンバ変数search\_radiusの変数名は変更可能。
  
\end{itemize}

\subsubsubsection{BOUNDARY\_CONDITION型にPS::BOUNDARY\_CONDITION\_OPEN以外を用いる場合}

\subsubsubsubsection{setPos}

\begin{screen}
\begin{verbatim}
class EssentialParticleJ {
public:
    PS::F64vec pos;
    void setPos(const PS::F64vec pos_new) {
        this->pos = pos_new;
    }
};
\end{verbatim}
\end{screen}

\begin{itemize}

\item {\bf 前提}
  
  EssentialParticleJクラスのメンバ変数posは1つの粒子の位置情報。この
  posのデータ型はPS::F32vecまたはPS::F64vec。EssentialParticleJクラス
  のメンバ変数posの元データとなっているのはFullParticleクラスのメンバ
  変数position。このデータ型はPS::F32vecまたはPS::F64vec。

\item {\bf 引数}

  pos\_new: 入力。const PS::F32vecまたはconst PS::F64vec型。FDPS側で修
  正した粒子の位置情報。

\item {\bf 返値}

  なし。
  
\item {\bf 機能}

  FDPSが修正した粒子の位置情報をEssentialParticleJクラスの位置情報に書き込む。

\item {\bf 備考}

  EssentialParticleJクラスのメンバ変数posの変数名は変更可能。メンバ関
  数EssentialParticleJ::setPosの引数名pos\_newは変更可能。posと
  pos\_newのデータ型が異なる場合の動作は保証しない。

\end{itemize}

\subsection{SuperParticleJクラス}
\label{sec:superparticlej}

\subsubsection{概要}

\subsubsection{既存のクラス}

\subsubsection{必要なメンバ関数}

\subsubsection{場合によっては必要なメンバ関数}

\subsubsubsection{getPos}

\subsubsubsection{setPos}

longcutoff

\subsubsubsection{copyFromMoment}

\subsubsubsection{convertToMoment}

\subsubsubsection{clear}


\subsection{Momentクラス}
\label{sec:moment}

\subsubsection{概要}

\subsubsection{既存のクラス}

\subsubsection{必要なメンバ関数}

\subsubsubsection{init}

\subsubsubsection{getPos}

\subsubsubsection{set}

\subsubsubsection{accumulateAtLeaf}

\subsubsubsection{accumulate}

\subsubsection{場合によっては必要なメンバ関数}

\subsubsubsection{accumulateAtLeaf2}

\subsubsubsection{accumulate2}


\subsection{Forceクラス}
\label{sec:force}

\subsubsection{概要}

\subsubsection{必要なメンバ関数}

\subsubsubsection{clear}


\subsection{ヘッダクラス}

\subsubsection{概要}

\subsubsection{必要なメンバ関数}

\subsubsubsection{概要}

\subsubsection{場合によっては必要なメンバ関数}

\subsubsubsection{概要}

\subsubsubsection{readParticleAscii}

\subsubsubsection{readParticleBinary}

\subsubsubsection{writeParticleAscii}

\subsubsubsection{writeParticleBinary}

\subsection{calcForceEpEpファンクタ}

\subsection{calcForceSpEpファンクタ}
