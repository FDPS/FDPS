ベクトル型のラッパーの定義を以下に示す。
\begin{lstlisting}[caption=vectorwrapper]
namespace ParticleSimulator{
    typedef Vector2<F32> F32vec2;
    typedef Vector3<F32> F32vec3;
    typedef Vector2<F64> F64vec2;
    typedef Vector3<F64> F64vec3;
#ifdef PARTICLE_SIMULATOR_TOW_DIMENSION
    typedef F32vec2 F32vec;
    typedef F64vec2 F64vec;
#else
    typedef F32vec3 F32vec;
    typedef F64vec3 F64vec;
#endif
}
namespace PS = ParticleSimulator;
\end{lstlisting}

すなわちPS::F32vec2, PS::F32vec3, PS::F64vec2, PS::F64vec3はそれぞれ単
精度2次元ベクトル、倍精度2次元ベクトル、単精度3次元ベクトル、倍精度3次
元ベクトルである。FDPSで扱う空間座標系を2次元とした場合、PS::F32vecと
PS::F64vecはそれぞれ単精度2次元ベクトル、倍精度2次元ベクトルとなる。一
方、FDPSで扱う空間座標系を3次元とした場合、PS::F32vecとPS::F64vecはそ
れぞれ単精度3次元ベクトル、倍精度3次元ベクトルとなる。

