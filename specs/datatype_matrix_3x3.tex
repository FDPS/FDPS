PS::MatrixSym3はxx, yy, zz, xy, xz, yzの6要素を持つ。これらに対する様々
なAPIや演算子を定義した。それらの宣言を以下に記述する。この節ではこれ
らについて詳しく記述する。
\begin{lstlisting}[caption=MatrixSym3]
namespace ParticleSimulator{
    template<class T>
    class MatrixSym3{
    public:
        // メンバ変数6要素
        T xx, yy, zz, xy, xz, yz;

        // コンストラクタ
        MatrixSym3() : xx(T(0)), yy(T(0)), zz(T(0)),
                       xy(T(0)), xz(T(0)), yz(T(0)) {}
        MatrixSym3(const T _xx, const T _yy, const T _zz,
                   const T _xy, const T _xz, const T _yz )
                       : xx(_xx), yy(_yy), zz(_zz),
                       xy(_xy), xz(_xz), yz(_yz) {}
        MatrixSym3(const T s) : xx(s), yy(s), zz(s),
                                xy(s), xz(s), yz(s) {}
        MatrixSym3(const MatrixSym3 & src) :
            xx(src.xx), yy(src.yy), zz(src.zz),
            xy(src.xy), xz(src.xz), yz(src.yz) {}

        // 代入演算子
        const MatrixSym3 & operator = (const MatrixSym3 & rhs);

        // 加減算
        MatrixSym3 operator + (const MatrixSym3 & rhs) const;
        const MatrixSym3 & operator += (const MatrixSym3 & rhs) const;
        MatrixSym3 operator - (const MatrixSym3 & rhs) const;
        const MatrixSym3 & operator -= (const MatrixSym3 & rhs) const;

        // トレースを取る
        T getTrace() const;

        // MatrixSym3<U>への型変換
        template <typename U>
        operator MatrixSym3<U> () const;
    }
}
namespace PS = ParticleSimulator;
\end{lstlisting}

%%%%%%%%%%%%%%%%%%%%%%%%%%%%%%%%%%%%%%%%%%%%%%%%%%%%%
\subsubsubsection{コンストラクタ}

\begin{screen}
\begin{verbatim}
template<typename T>
PS::MatrixSym3<T>();
\end{verbatim}
\end{screen}

\begin{itemize}

\item{{\bf 引数}}

なし。

\item{{\bf 機能}}

デフォルトコンストラクタ。6要素は0で初期化される。

\end{itemize}

\begin{screen}
\begin{verbatim}
template<typename T>
PS::MatrixSym3<T>(const T _xx,
                  const T _yy,
                  const T _zz,
                  const T _xy,
                  const T _xz,
                  const T _yz);
\end{verbatim}
\end{screen}

\begin{itemize}

\item{{\bf 引数}}

{\tt \_xx}: 入力。{\tt const T}型。

{\tt \_yy}: 入力。{\tt const T}型。

{\tt \_zz}: 入力。{\tt const T}型。

{\tt \_xy}: 入力。{\tt const T}型。

{\tt \_xz}: 入力。{\tt const T}型。

{\tt \_yz}: 入力。{\tt const T}型。

\item{{\bf 機能}}

メンバ{\tt xx}、{\tt yy}、{\tt zz}、{\tt xy}、{\tt xz}、{\tt yz}をそれ
ぞれ{\tt \_xx}、{\tt \_yy}、{\tt \_zz}、{\tt \_xy}、{\tt \_xz}、{\tt
  \_yz}で初期化する。

\end{itemize}

\begin{screen}
\begin{verbatim}
template<typename T>
PS::MatrixSym3<T>(const T s);
\end{verbatim}
\end{screen}

\begin{itemize}

\item{{\bf 引数}}

{\tt s}: 入力。{\tt const T}型。

\item{{\bf 機能}}

6要素すべてを{\tt s}の値で初期化する。

\end{itemize}

\begin{screen}
\begin{verbatim}
template<typename T>
PS::MatrixSym3<T>(const PS::MatrixSym3<T> & src)
\end{verbatim}
\end{screen}

\begin{itemize}

\item{{\bf 引数}}

{\tt src}: 入力。{\tt const PS::MatrixSym3<T> \&}型。

\item{{\bf 機能}}

コピーコンストラクタ。{\tt src}で初期化する。

\end{itemize}

%%%%%%%%%%%%%%%%%%%%%%%%%%%%%%%%%%%%%%%%%%%%%%%%%%%%%
\subsubsubsection{代入演算子}

\begin{screen}
\begin{verbatim}
template<typename T>
const PS::MatrixSym3<T> & PS::MatrixSym3<T>::operator = 
                       (const PS::MatrixSym3<T> & rhs);
\end{verbatim}
\end{screen}

\begin{itemize}

\item{{\bf 引数}}

{\tt rhs}: 入力。{\tt const PS::MatrixSym3<T> \&}型。

\item{{\bf 返り値}}

{\tt const PS::MatrixSym3<T> \&}型。{\tt rhs}の6要素それぞれの値を自
身の6要素それぞれに代入し自身の参照を返す。代入演算子。

\end{itemize}

%%%%%%%%%%%%%%%%%%%%%%%%%%%%%%%%%%%%%%%%%%%%%%%%%%%%%
\subsubsubsection{加減算}

\begin{screen}
\begin{verbatim}
template<typename T>
PS::MatrixSym3<T> PS::MatrixSym3<T>::operator + 
               (const PS::MatrixSym3<T> & rhs) const;
\end{verbatim}
\end{screen}

\begin{itemize}

\item{{\bf 引数}}

{\tt rhs}: 入力。{\tt const PS::MatrixSym3<T> \&}型。

\item{{\bf 返り値}}

{\tt PS::MatrixSym3<T> }型。{\tt rhs}の6要素それぞれの値と自身の6要
素の値の和を取った値を返す。

\end{itemize}

\begin{screen}
\begin{verbatim}
template<typename T>
const PS::MatrixSym3<T> & PS::MatrixSym3<T>::operator += 
                       (const PS::MatrixSym3<T> & rhs);
\end{verbatim}
\end{screen}

\begin{itemize}

\item{{\bf 引数}}

{\tt rhs}: 入力。{\tt const PS::MatrixSym3<T> \&}型。

\item{{\bf 返り値}}

{\tt const PS::MatrixSym3<T> \&}型。{\tt rhs}の6要素それぞれの値を自
身の6要素それぞれに足し、自身を返す。

\end{itemize}

\begin{screen}
\begin{verbatim}
template<typename T>
PS::MatrixSym3<T> PS::MatrixSym3<T>::operator - 
               (const PS::MatrixSym3<T> & rhs) const;
\end{verbatim}
\end{screen}

\begin{itemize}

\item{{\bf 引数}}

{\tt rhs}: 入力。{\tt const PS::MatrixSym3<T> \&}型。

\item{{\bf 返り値}}

{\tt PS::MatrixSym3<T> }型。{\tt rhs}の6要素それぞれの値と自身の6要
素それぞれの値の差を取った値を返す。

\end{itemize}

\begin{screen}
\begin{verbatim}
template<typename T>
const PS::MatrixSym3<T> & PS::MatrixSym3<T>::operator -= 
                       (const PS::MatrixSym3<T> & rhs);
\end{verbatim}
\end{screen}

\begin{itemize}

\item{{\bf 引数}}

{\tt rhs}: 入力。{\tt const PS::MatrixSym3<T> \&}型。

\item{{\bf 返り値}}

{\tt const PS::MatrixSym3<T> \&}型。自身の6要素それぞれから{\tt rhs}
の6要素それぞれを引き自身を返す。

\end{itemize}

%%%%%%%%%%%%%%%%%%%%%%%%%%%%%%%%%%%%%%%%%%%%%%%%%%%%%
\subsubsubsection{トレースの計算}

\begin{screen}
\begin{verbatim}
template<typename T>
T PS::MatrixSym3<T>::getTrace() const;
\end{verbatim}
\end{screen}

\begin{itemize}

\item{{\bf 引数}}

なし

\item{{\bf 返り値}}

{\tt T}型。

\item{{\bf 機能}}

  トレースを計算し、その結果を返す。

\end{itemize}

%%%%%%%%%%%%%%%%%%%%%%%%%%%%%%%%%%%%%%%%%%%%%%%%%%%%%
\subsubsubsection{{\tt MatrixSym3<U>}への型変換}

\begin{screen}
\begin{verbatim}
template<typename T>
template<typename U>
PS::MatrixSym3<T>::operator PS::MatrixSym3<U> () const;
\end{verbatim}
\end{screen}

\begin{itemize}

\item{{\bf 引数}}

  なし。

\item{{\bf 返り値}}

{\tt const PS::MatrixSym3<U>}型。

\item{{\bf 機能}}

\redtext{{\tt const PS::MatrixSym3<T>}型を{\tt const
    PS::MatrixSym3<U>}型にキャストする}

\end{itemize}

