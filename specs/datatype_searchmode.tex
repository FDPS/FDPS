\subsubsection{概要}

本節では、SEARCH\_MODE型について記述する。SEARCH\_MODE型は相互作用ツリー
クラスのテンプレート引数としてのみ使用されるものである。この型によって、
相互作用ツリークラスで計算する相互作用のモードを決定する。SEARCH\_MODE
型にはSEARCH\_MODE\_LONG, SEARCH\_MODE\_LONG\_CUTOFF,
SEARCH\_MODE\_GATHER, SEARCH\_MODE\_SCATTER, SEARCH\_MODE\_SYMMETRYが
ある。以下に、それぞれが対応する相互作用のモードについて記述する。

\subsubsection{SEARCH\_MODE\_LONG}

この型を使用するのは、遠くの粒子からの寄与を複数の粒子にまとめた超粒子
からの寄与として計算する場合である。開放境界条件における重力やクーロン
力に適用できる。

\subsubsection{SEARCH\_MODE\_LONG\_CUTOFF}

この型を使用するのは、遠くの粒子からの寄与を複数の粒子にまとめた超粒子
からの寄与として計算し、かつ有限の距離までの寄与しか計算しない場合であ
る。周期境界条件における重力やクーロン力(Particle Mesh法を並用)などに
適用できる。

\subsubsection{SEARCH\_MODE\_GATHER}

この型を使用するのは、相互作用の到達距離が有限でかつ、その到達距離がi
粒子の大きさで決まる場合である。

\subsubsection{SEARCH\_MODE\_SCATTER}

この型を使用するのは、相互作用の到達距離が有限でかつ、その到達距離がj
粒子の大きさで決まる場合である。

\subsubsection{SEARCH\_MODE\_SYMMETRY}

この型を使用するのは、相互作用の到達距離が有限でかつ、その到達距離がi,
j粒子両方の大きさで決まる場合である。

