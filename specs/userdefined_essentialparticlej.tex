\subsubsection{概要}

EssentialParticleJクラスは相互作用の計算に必要なj粒子情報を持つクラス
である。FDPSの内部では、このクラスがFullParticleクラスから情報を読み取
る。情報を読み取るために、このクラスはいくつかのメンバ関数を持つ必要が
ある。以下、この節の前提、常に必要なメンバ関数と、場合によっては必要な
メンバ関数について記述する。

\subsubsection{前提}

この節の中では、名前空間ParticleSimulatorをPSと省略する。このクラスの
クラス名をEssentialParticleJとする。また、粒子すべての情報を持つクラス
のクラス名をFullParticleとする。このFullParticleは
節\ref{sec:fullparticle}のクラスFullParticleと同一のものである。
EssentialParticleJ, FullParticleというクラス名は変更可能である。

ParticleSimulatorをPSと省略すること、EssentialParticleJとFullParticle
の宣言は以下の通りである。
\begin{screen}
\begin{verbatim}
namespace PS = ParticleSimulator;
class FullParticle;
class EssentialParticleJ;
\end{verbatim}
\end{screen}

\subsubsection{必要なメンバ関数}

\subsubsubsection{概要}

常に必要なメンバ関数はgetPosとCopyfromFPである。getPosは
EssentialParticleJクラスの位置情報をFDPSに読み込ませるための関数で、
copyFromFPはFullParticleクラスの情報をEssentialParticleJクラスに書きこ
む関数である。これらのメンバ関数の記述例と解説を以下に示す。

\subsubsubsection{getPos}

\begin{screen}
\begin{verbatim}
class EssentialParticleJ {
public:
    PS::F64vec pos;
    PS::F64vec getPos() const {
        return this->pos;
    }
};
\end{verbatim}
\end{screen}

\begin{itemize}

\item {\bf 前提}
  
  EssentialParticleJのメンバ変数posはある1つの粒子の位置情報。このposのデー
  タ型はPS::F64vec型。
  
\item {\bf 引数}

  なし
  
\item {\bf 返値}

  PS::F64vec型。EssentialParticleJクラスの位置情報を保持したメンバ変数。
  
\item {\bf 機能}

  EssentialParticleJクラスの位置情報を保持したメンバ変数を返す。
  
\item {\bf 備考}

  EssentialParticleJクラスのメンバ変数posの変数名は変更可能。

\end{itemize}

\subsubsubsection{copyFromFP}

\begin{screen}
\begin{verbatim}
class FullParticle {
public:
    PS::S64    identity;
    PS::F64    mass;
    PS::F64vec position;
    PS::F64vec velocity;
    PS::F64vec acceleration;
    PS::F64    potential;
};
class EssentialParticleJ {
public:
    PS::S64    id;
    PS::F64    m;
    PS::F64vec pos;
    void copyFromFP(const FullParticle & fp) {
        this->id  = fp.identity;
        this->m   = fp.mass;
        this->pos = fp.position;
    }
};
\end{verbatim}
\end{screen}

\begin{itemize}

\item {\bf 前提}

  FullParticleクラスのメンバ変数identity, mass, positionと
  EssentialParticleJクラスのメンバ変数id, m, posはそれぞれ対応する情報
  を持つ。

\item {\bf 引数}

  fp: 入力。const FullParticle \&型。FullParticleクラスの情報を持つ。
  
\item {\bf 返値}

  なし。
  
\item {\bf 機能}

  FullParticleクラスの持つ1粒子の情報の一部をEssnetialParticleJクラス
  に書き込む。
  
\item {\bf 備考}

  Fullparticeクラスのメンバ変数の変数名、EssentialParticleJクラスのメ
  ンバ変数の変数名は変更可能。メンバ関数EssentialParticleJ::copyFromFP
  の引数名は変更可能。EssentialParticleJクラスの粒子情報はFullParticle
  クラスの粒子情報のサブセット。対応する情報を持つメンバ変数同士のデー
  タ型が一致している必要はないが、実数型とベクトル型(または整数型とベ
  クトル型)という違いがある場合に正しく動作する保証はない。

\end{itemize}

\subsubsection{場合によっては必要なメンバ関数}

\subsubsubsection{概要}

本節では、場合によっては必要なメンバ関数について記述する。相互作用ツリー
クラスのSEARCH\_MODE型にSEARCH\_MODE\_LONG以外を用いる場合に必要なメン
バ関数、列挙型のBOUNDARY\_CONDITION型にPS::BOUNDARY\_CONDITION\_OPEN以
外を選んだ場合に必要となるメンバ関数について記述する。

\subsubsubsection{相互作用ツリークラスのSEARCH\_MODE型にSEARCH\_MODE\_LONG以外を用いる場合}

\subsubsubsubsection{getRsearch}

\begin{screen}
\begin{verbatim}
class EssentialParticleJ {
public:
    PS::F64 search_radius;
    PS::F64 getRsearch() const {
        return this->search_radius;
    }
};
\end{verbatim}
\end{screen}

\begin{itemize}

\item {\bf 前提}

  EssentialParticleJクラスのメンバ変数search\_radiusはある1つの粒子の
  近傍粒子を探す半径の大きさ。このsearch\_radiusのデータ型はPS::F32型
  またはPS::F64型。
  
\item {\bf 引数}

  なし
  
\item {\bf 返値}

  PS::F32型またはPS::F64型。 EssentialParticleJクラスの近傍粒子を探す
  半径の大きさを保持したメンバ変数。
  
\item {\bf 機能}

  EssentialParticleJクラスの近傍粒子を探す半径の大きさを保持したメンバ
  変数を返す。

\item {\bf 備考}

  EssentialParticleJクラスのメンバ変数search\_radiusの変数名は変更可能。
  
\end{itemize}

\subsubsubsection{BOUNDARY\_CONDITION型にPS::BOUNDARY\_CONDITION\_OPEN以外を用いる場合}

\subsubsubsubsection{setPos}

\begin{screen}
\begin{verbatim}
class EssentialParticleJ {
public:
    PS::F64vec pos;
    void setPos(const PS::F64vec pos_new) {
        this->pos = pos_new;
    }
};
\end{verbatim}
\end{screen}

\begin{itemize}

\item {\bf 前提}
  
  EssentialParticleJクラスのメンバ変数posは1つの粒子の位置情報。この
  posのデータ型はPS::F32vecまたはPS::F64vec。EssentialParticleJクラス
  のメンバ変数posの元データとなっているのはFullParticleクラスのメンバ
  変数position。このデータ型はPS::F32vecまたはPS::F64vec。

\item {\bf 引数}

  pos\_new: 入力。const PS::F32vecまたはconst PS::F64vec型。FDPS側で修
  正した粒子の位置情報。

\item {\bf 返値}

  なし。
  
\item {\bf 機能}

  FDPSが修正した粒子の位置情報をEssentialParticleJクラスの位置情報に書き込む。

\item {\bf 備考}

  EssentialParticleJクラスのメンバ変数posの変数名は変更可能。メンバ関
  数EssentialParticleJ::setPosの引数名pos\_newは変更可能。posと
  pos\_newのデータ型が異なる場合の動作は保証しない。

\end{itemize}
