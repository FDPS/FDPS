\subsubsection{概要}

本節ではFDPSで定義されている列挙型について記述する。列挙型には
BOUNDARY\_CONDITION型が存在する。以下、各列挙型について記述する。

\subsubsection{BOUNDARY\_CONDITION型}
\label{sec:type_enum}

\subsubsubsection{概要}

BOUNDARY\_CONDITION型は境界条件を指定するためのデータ型である。これは
以下のように定義されている。
\begin{lstlisting}[caption=boundarycondition]
namespace ParticleSimulator{
    enum BOUNDARY_CONDITION{
        BOUNDARY_CONDITION_OPEN,
        BOUNDARY_CONDITION_PERIODIC_X,
        BOUNDARY_CONDITION_PERIODIC_Y,
        BOUNDARY_CONDITION_PERIODIC_Z,
        BOUNDARY_CONDITION_PERIODIC_XY,
        BOUNDARY_CONDITION_PERIODIC_XZ,
        BOUNDARY_CONDITION_PERIODIC_YZ,
        BOUNDARY_CONDITION_PERIODIC_XYZ,
        BOUNDARY_CONDITION_SHEARING_BOX,
        BOUNDARY_CONDITION_USER_DEFINED,
    };
}
namespace PS = ParticleSimulator;
\end{lstlisting}

以下にどの変数がどの境界条件に対応するかを記述する。

\subsubsubsection{PS::BOUNDARY\_CONDITION\_OPEN}

開放境界となる。

\subsubsubsection{PS::BOUNDARY\_CONDITION\_PERIODIC\_X}

x軸方向のみ周期境界、その他の軸方向は開放境界となる。周期の境界の下限
は閉境界、上限は開境界となっている。この境界の規定はすべての軸方向にあ
てはまる。

\subsubsubsection{PS::BOUNDARY\_CONDITION\_PERIODIC\_Y}

y軸方向のみ周期境界、その他の軸方向は開放境界となる。

\subsubsubsection{PS::BOUNDARY\_CONDITION\_PERIODIC\_Z}

z軸方向のみ周期境界、その他の軸方向は開放境界となる。

\subsubsubsection{PS::BOUNDARY\_CONDITION\_PERIODIC\_XY}

x, y軸方向のみ周期境界、その他の軸方向は開放境界となる。

\subsubsubsection{PS::BOUNDARY\_CONDITION\_PERIODIC\_XZ}

x, z軸方向のみ周期境界、その他の軸方向は開放境界となる。

\subsubsubsection{PS::BOUNDARY\_CONDITION\_PERIODIC\_YZ}

y, z軸方向のみ周期境界、その他の軸方向は開放境界となる。

\subsubsubsection{PS::BOUNDARY\_CONDITION\_PERIODIC\_XYZ}

x, y, z軸方向すべてが周期境界となる。

\subsubsubsection{PS::BOUNDARY\_CONDITION\_SHEARING\_BOX}

未実装。

\subsubsubsection{PS::BOUNDARY\_CONDITION\_USER\_DEFINED}

未実装。

