本節では、領域クラスについて記述する。このクラスは領域情報の保持や領域
の分割を行うモジュールである。まずオブジェクトの生成方法を記述し、その
後APIを記述する。

%%%%%%%%%%%%%%%%%%%%%%%%%%%%%%%%%%%%%%%%%%%%%%%%%%%%%%%%%%%%%%%%%
\subsubsubsection{オブジェクトの生成}

領域クラスは以下のように宣言されている。
\begin{lstlisting}[caption=DomainInfo0]
namespace ParticleSimulator {
    class DomainInfo;
}
namespace PS = ParticleSimulator;
\end{lstlisting}

領域クラスのオブジェクトの生成は以下のように行う。ここではdinfoという
オブジェクトを生成している。
\begin{screen}
\begin{verbatim}
PS::DomainInfo dinfo;
\end{verbatim}
\end{screen}

%%%%%%%%%%%%%%%%%%%%%%%%%%%%%%%%%%%%%%%%%%%%%%%%%%%%%%%%%%%%%%%%%
\subsubsubsection{API}

領域クラスには初期設定関連のAPI、領域分割関連のAPIがある。以下、各節に
分けて記述する。

%%%%%%%%%%%%%%%%%%%%%%%%%%%%%%%%%%%%%%%%%%%%%%%%%%%%%%
\subsubsubsubsection{初期設定}

初期設定関連のAPIの宣言は以下のようになっている。このあと各APIについて
記述する。
\begin{lstlisting}[caption=DomainInfo1]
namespace ParticleSimulator {
    class DomainInfo{
    public
        DomainInfo();
        void initialize(const F32 coef_ema=1.0);
        void setNumberOfDomainMultiDimension(const S32 nx,
                                             const S32 ny,
                                             const S32 nz=1);
        void setBoundaryCondition(enum BOUNDARY_CONDITION bc);
        void setPosRootDomain(const F32vec & low,
                              const F32vec & high);
    };
}
namespace PS = ParticleSimulator;
\end{lstlisting}

%%%%%%%%%%%%%%%%%%%%%%%%%%%%%%%%%%%%%%%%%%%
\subsubsubsubsubsection{コンストラクタ}

\begin{screen}
\begin{verbatim}
void PS::DomainInfo::DomainInfo();
\end{verbatim}
\end{screen}

\begin{itemize}

\item {\bf 引数}

なし

\item {\bf 返値}

なし

\item {\bf 機能}

領域クラスのオブジェクトを生成する。

\end{itemize}

%%%%%%%%%%%%%%%%%%%%%%%%%%%%%%%%%%%%%%%%%%%
\subsubsubsubsubsection{initialize}

\begin{screen}
\begin{verbatim}
void PS::DomainInfo::initialize(const PS::F32 coef_ema=1.0);
\end{verbatim}
\end{screen}

\begin{itemize}

\item {\bf 引数}

coef\_ema: 入力。 const PS::F32型。指数移動平均の平滑化係数。デフォルト1.0

\item {\bf 返値}

なし

\item {\bf 機能}

領域クラスのオブジェクトを初期化する。

指数移動平均の平滑化係数を設定する。この係数の許される値は0から1である。
大きくなるほど、最新の粒子分布の情報が領域分割に反映されやすい。1の場
合、最新の粒子分布の情報のみ反映される。1度は呼ぶ必要があるが、2度呼
ぶと例外が送出される。

\end{itemize}

%%%%%%%%%%%%%%%%%%%%%%%%%%%%%%%%%%%%%%%%%%%
\subsubsubsubsubsection{setNumberOfDomainMultiDimension}

\begin{screen}
\begin{verbatim}
void PS::DomainInfo::setNumberOfDomainMultiDimension
                 (const PS::S32 nx,
                  const PS::S32 ny,
                  const PS::S32 nz=1);
\end{verbatim}
\end{screen}

\begin{itemize}

\item {\bf 引数}

nx: 入力。 const PS::S32型。x軸方向のルートドメインの分割数。

ny: 入力。 const PS::S32型。y軸方向のルートドメインの分割数。

nz: 入力。 const PS::S32型。z軸方向のルートドメインの分割数。デフォル
ト1。

\item {\bf 返値}

なし

\item {\bf 機能}

ルートドメインの分割する方法を設定する。nx, ny, nzはそれぞれx軸、y軸、
z軸方向のルートドメインの分割数である。呼ばなければ自動的にnx, ny, nz
が決まる。呼んだ場合に入力するnx, ny, nzの総積がMPIプロセス数と等しく
なければ、例外が送出される。

\end{itemize}

%%%%%%%%%%%%%%%%%%%%%%%%%%%%%%%%%%%%%%%%%%%
\subsubsubsubsubsection{setBoundaryCondition}

\begin{screen}
\begin{verbatim}
void PS::DomainInfo::setBoundaryCondition
                 (enum BOUNDARY_CONDITION bc);
\end{verbatim}
\end{screen}

\begin{itemize}

\item {\bf 引数}

bc: 入力。 列挙型。境界条件。

\item {\bf 返値}

なし

\item {\bf 機能}

境界条件の設定をする。許される入力は、\ref{sec:type_enum}で挙げた列挙
型のみ(ただしBOUNDARY\_CONDITION\_SHEARING\_BOX,
BOUNDARY\_CONDITION\_USER\_DEFINEDは未実装)。呼ばない場合は、開放境界
となる。

\end{itemize}

%%%%%%%%%%%%%%%%%%%%%%%%%%%%%%%%%%%%%%%%%%%
\subsubsubsubsubsection{setPosRootDomain}

\begin{screen}
\begin{verbatim}
void PS::DomainInfo::setPosRootDomain
                 (const PS::F32vec & low,
                  const PS::F32vec & high);
\end{verbatim}
\end{screen}

\begin{itemize}

\item {\bf 引数}

low: 入力。 PS::F32vec型。ルートドメインの下限(閉境界)。

high: 入力。 PS::F32vec型。ルートドメインの上限(解境界)。

\item {\bf 返値}

なし

\item {\bf 機能}

ルートドメインの下限と上限を設定する。開放境界条件の場合は呼ぶ必要はな
い。それ以外の境界条件の場合は、呼ばなくても動作するが、その結果が正し
いことは保証できない。

\end{itemize}

%%%%%%%%%%%%%%%%%%%%%%%%%%%%%%%%%%%%%%%%%%%%%%%%%%%%%%
\subsubsubsubsection{領域分割}

領域分割関連のAPIの宣言は以下のようになっている。このあと各APIについて
記述する。
\begin{lstlisting}[caption=DomainInfo2]
namespace ParticleSimulator {
    class DomainInfo{
    public:
        template<class Tpsys>
        void collectSampleParticle(Tpsys & psys,
                                   const F32 weight=1.0,
                                   const bool clear=true);
        void decomposeDomain();
        template<class Tpsys>
        void decomposeDomainAll(Tpsys & psys,
                                const F32 wgh=1.0);
    };
}
namespace PS = ParticleSimulator;
\end{lstlisting}

%%%%%%%%%%%%%%%%%%%%%%%%%%%%%%%%%%%%%%%%%%%
\subsubsubsubsubsection{collectSampleParticle}

\begin{screen}
\begin{verbatim}
template<class Tpsys>
void PS::DomainInfo::collectSampleParticle
                 (Tpsys & psys,
                  const PS::F32 weight=1.0,
                  const bool clear=true);
\end{verbatim}
\end{screen}

\begin{itemize}

\item {\bf 引数}

psys: 入力。 Tpsys型。領域分割のためのサンプル粒子を提供する粒子群クラ
ス。

weight: 入力。 const PS::F32型。領域分割のためのサンプル粒子数を決める
ためのウェイト。デフォルト1.0。

clear: 入力。 bool型。前にサンプルされた粒子情報をクリアするかどうかを
決定するフラグ。trueでクリアする。デフォルトtrue。

\item {\bf 返値}

なし

\item {\bf 機能}

粒子群クラスのオブジェクトpsysから粒子をサンプルする。weightによってそ
のMPIプロセスからサンプルする粒子の量を調整する(weightが大きいほどサン
プル粒子数が多い)。clearによってこれより前にサンプルした粒子の情報を消
すかどうか決める。

\end{itemize}

%%%%%%%%%%%%%%%%%%%%%%%%%%%%%%%%%%%%%%%%%%%
\subsubsubsubsubsection{decomposeDomain}

\begin{screen}
\begin{verbatim}
template<class Tpsys>
void PS::DomainInfo::decomposeDomain();
\end{verbatim}
\end{screen}

\begin{itemize}

\item {\bf 引数}

なし

\item {\bf 返値}

なし

\item {\bf 機能}

ルートドメインの分割を行う。

\end{itemize}

%%%%%%%%%%%%%%%%%%%%%%%%%%%%%%%%%%%%%%%%%%%
\subsubsubsubsubsection{decomposeDomainAll}

\begin{screen}
\begin{verbatim}
template<class Tpsys>
void PS::DomainInfo::decomposeDomainAll
                 (Tpsys & psys,
                  const PS::F32 weight=1.0);
\end{verbatim}
\end{screen}

\begin{itemize}

\item {\bf 引数}

psys: 入力。 Tpsys型。領域分割のためのサンプル粒子を提供する粒子群クラ
ス。

weight: 入力。 const PS::F32型。領域分割のためのサンプル粒子数を決める
ためのウェイト。デフォルト1.0。

\item {\bf 返値}

なし

\item {\bf 機能}

粒子群クラスのオブジェクトpsysから粒子をサンプルし、続けてルートドメイ
ンの分割を行う。PS::DomainInfo::collectSampleParticleと
PS::DomainInfo::decomposeDomainで行われていることが一度に行われる。
weightの意味はPS::DomainInfo::collectSampleParticleと同じ。

\end{itemize}

%%\subsubsubsubsubsubsection{getNumberofDomainOneAxis}

%%\subsubsubsubsubsubsection{getDomain}
