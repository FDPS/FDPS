\documentclass[12pt,a4paper]{jarticle}
%
\topmargin=-5mm
\oddsidemargin=-5mm
\evensidemargin=-5mm
\textheight=235mm
\textwidth=165mm
%
\title{FDPS仕様書}
\author{FDPS開発者}
\date{}
%\pagestyle{empty}
\usepackage{graphicx}
\usepackage{wrapfig}
\usepackage{lscape}
\usepackage{amssymb}
\usepackage{amsmath}
\usepackage{bm}
\usepackage{setspace}
\usepackage{listings,jlisting}
\usepackage{color}
\usepackage{ascmac}
\usepackage{here}

\newcommand{\underbold}[1]{\underline{\bf #1}}
\newcommand{\redtext}[1]{\textcolor{red}{#1}}


%\setcounter{secnumdepth}{4}
%%%%%%%%%%%%%%%%%%%%%%%%%%%%%%%%%%
\setcounter{secnumdepth}{6}
\makeatletter
\newcounter{subsubparagraph}[subparagraph]
\renewcommand\thesubsubparagraph{\thesubparagraph.\@arabic\c@subsubparagraph}
\newcommand\subsubparagraph{\@startsection{subsubparagraph}{6}{\parindent}%
                                       {3.25ex \@plus1ex \@minus .2ex}%
                                       {-1em}%
                              {\normalfont\normalsize\bfseries}}
\newcommand*\l@subsubparagraph{\@dottedtocline{6}{10em}{5em}}
\newcommand{\subsubsubsection}{\@startsection{paragraph}{4}{\z@}%
{1.5\baselineskip \@plus.5\dp0 \@minus.2\dp0}%
{.5\baselineskip \@plus2.3\dp0}%
{\reset@font\normalsize\bfseries}
}
\newcommand{\subsubsubsubsection}{\@startsection{subparagraph}{5}{\z@}%
{1.5\baselineskip \@plus.5\dp0 \@minus.2\dp0}%
{.5\baselineskip \@plus2.3\dp0}%
{\reset@font\normalsize\itshape}
}
\newcommand{\subsubsubsubsubsection}{\@startsection{subsubparagraph}{6}{\z@}%
{1.5\baselineskip \@plus.5\dp0 \@minus.2\dp0}%
{.5\baselineskip \@plus2.3\dp0}%
{\reset@font\normalsize\itshape}
}
\setcounter{tocdepth}{6}
%%%%%%%%%%%%%%%%%%%%%%%%%%%%%%%%%%

%\twocolumn
%\setstretch{1.5}

\lstset{language = C,
numbers = left,
numbersep = 8pt,
breaklines = true,
breakindent = 40pt,
frame = lines,
basicstyle = \ttfamily,
}

\begin{document}
\maketitle
\tableofcontents

\newpage

%%%%%%%%%%%%%%%%%%%%%%%%%%%%%%%%%%%%%%%%%%%%%%%%%%%%%
\section{ファイル構成}

\subsection{概要}

\subsection{標準機能関係ファイル}

\subsection{拡張機能関係ファイル}

\subsection{テストコード}

\subsection{サンプルコード}

\newpage

%%%%%%%%%%%%%%%%%%%%%%%%%%%%%%%%%%%%%%%%%%%%%%%%%%%%%
\section{コンパイル時のマクロによる選択}

\subsection{概要}

\subsection{空間次元数}

\subsubsection{概要}

\subsubsection{直交座標系2次元}

\subsubsection{直交座標系3次元}

\subsection{並列処理}

\subsubsection{概要}

\subsubsection{OpenMPの使用}

\subsubsection{MPIの使用}

\newpage

%%%%%%%%%%%%%%%%%%%%%%%%%%%%%%%%%%%%%%%%%%%%%%%%%%%%%
\section{名前空間}

\subsection{概要}

\subsection{ParticleSimulator}

\subsubsection{ParticleMesh}

\newpage

%%%%%%%%%%%%%%%%%%%%%%%%%%%%%%%%%%%%%%%%%%%%%%%%%%%%%
\section{データ型}

\subsection{概要}

FDPSでは独自の整数型、実数型、ベクトル型、行列型、SEARCH\_MODE型、列挙
型が定義されている。整数型、実数型、ベクトル型、行列型に関しては必ずし
もここに挙げるものを用いる必要はないが、これらを用いることを推奨する。
SEARCH\_MODE型、列挙型は必ず用いる必要がある。以下、整数型、実数型、ベ
クトル型、行列型、SEARCH\_MODE型、列挙型の順に記述する。

\subsection{整数型}

\subsubsection{概要}

整数型にはPS::S32, PS::S64, PS::U32, PS::U64がある。以下、順にこれらを
記述する。

\subsubsection{PS::S32}

PS::S32は以下のように定義されている。すなわち32bitの符号付き整数である。
\begin{lstlisting}[caption=S32]
namespace ParticleSimulator {
    typedef int S32;
}
namespace PS = ParticleSimulator;
\end{lstlisting}

ただし、GCCコンパイラとKコンパイラでのみ32bitであることが保証されている。

\subsubsection{PS::S64}

PS::S64は以下のように定義されている。すなわち64bitの符号付き整数である。
\begin{lstlisting}[caption=S64]
namespace ParticleSimulator {
    typedef long S64;
}
namespace PS = ParticleSimulator;
\end{lstlisting}

ただし、GCCコンパイラとKコンパイラでのみ64bitであることが保証されている。

\subsubsection{PS::U32}

PS::U32は以下のように定義されている。すなわち32bitの符号なし整数である。
\begin{lstlisting}[caption=U32]
namespace ParticleSimulator {
    typedef unsinged U32;
}
namespace PS = ParticleSimulator;
\end{lstlisting}

ただし、GCCコンパイラとKコンパイラでのみ32bitであることが保証されている。

\subsubsection{PS::U64}

PS::U64は以下のように定義されている。すなわち64bitの符号なし整数である。
\begin{lstlisting}[caption=U64]
namespace ParticleSimulator {
    typedef unsinged U64;
}
namespace PS = ParticleSimulator;
\end{lstlisting}

ただし、GCCコンパイラとKコンパイラでのみ64bitであることが保証されている。



\subsection{実数型}

\subsubsection{概要}

実数型にはPS::F32, PS::F64がある。以下、順にこれらを記述する。

\subsubsection{PS::F32}

PS::F32は以下のように定義されている。すなわち32bitの浮動小数点数である。
\begin{lstlisting}[caption=F32]
namespace ParticleSimulator {
    typedef float F32;
}
namespace PS = ParticleSimulator;
\end{lstlisting}

\subsubsection{PS::F64}

PS::F64は以下のように定義されている。すなわち64bitの浮動小数点数である。
\begin{lstlisting}[caption=F64]
namespace ParticleSimulator {
    typedef double F64;
}
namespace PS = ParticleSimulator;
\end{lstlisting}


\subsection{ベクトル型}

\subsubsection{概要}

ベクトル型には2次元ベクトル型PS::Vector2と3次元ベクトル型PS::Vector3が
ある。まずこれら2つを記述する。最後にこれらベクトル型のラッパーについ
て記述する。

\subsubsection{PS::Vector2}

PS::Vector2はx, yの2要素を持つ。これらに対する様々なAPIや演算子を定義
した。それらの宣言を以下に記述する。この節ではこれらについて詳しく記述
する。
\begin{lstlisting}[caption=Vector2]
namespace ParticleSimulator{
    template <typename T>
    class Vector2{
    public:
        //メンバ変数2要素
        T x, y;

        //コンストラクタ
        Vector2();
        Vector2(const T _x, const T _y) : x(_x), y(_y) {}
        Vector2(const T s) : x(s), y(s) {}
        Vector2(const Vector2 & src) : x(src.x), y(src.y) {}

        //代入演算子
        const Vector2 & operator = (const Vector2 & rhs);

        //加減算
        Vector2 operator + (const Vector2 & rhs) const;
        const Vector2 & operator += (const Vector2 & rhs);
        Vector2 operator - (const Vector2 & rhs) const;
        const Vector2 & operator -= (const Vector2 & rhs);

        //ベクトルスカラ積
        Vector2 operator * (const T s) const;
        const Vector2 & operator *= (const T s);
        friend Vector2 operator * (const T s, const Vector2 & v);
        Vector2 operator / (const T s) const;
        const Vector2 & operator /= (const T s);

        //内積
        T operator * (const Vector2 & rhs) const;

        //外積(返り値はスカラ!!)
        T operator ^ (const Vector2 & rhs) const;

        //Vector2<U>への型変換
        template <typename U>
        operator Vector2<U> () const;
    };
}
namespace PS = ParticleSimulator;
\end{lstlisting}

\subsubsubsection{コンストラクタ}

\begin{screen}
\begin{verbatim}
template<typename T>
PS::Vector2<T>()
\end{verbatim}
\end{screen}

\begin{itemize}

\item{{\bf 引数}}

なし。

\item{{\bf 機能}}

デフォルトコンストラクタ。メンバx,yは0で初期化される。

\end{itemize}

\begin{screen}
\begin{verbatim}
template<typename T>
PS::Vector2<T>(const T _x, const T _y)
\end{verbatim}
\end{screen}

\begin{itemize}

\item{{\bf 引数}}

{\tt \_x}: 入力。{\tt const T}型。

{\tt \_y}: 入力。{\tt const T}型。

\item{{\bf 機能}}

メンバ{\tt x}、{\tt y}をそれぞれ{\tt \_x}、{\tt \_y}で初期化する。

\end{itemize}

\begin{screen}
\begin{verbatim}
template<typename T>
PS::Vector2<T>(const T s);
\end{verbatim}
\end{screen}

\begin{itemize}

\item{{\bf 引数}}

{\tt s}: 入力。{\tt const T}型。

\item{{\bf 機能}}

メンバ{\tt x}、{\tt y}を両方とも{\tt s}の値で初期化する。

\end{itemize}

\begin{screen}
\begin{verbatim}
template<typename T>
PS::Vector2<T>(const PS::Vector2<T> & src)
\end{verbatim}
\end{screen}

\begin{itemize}

\item{{\bf 引数}}

{\tt src}: 入力。{\tt const PS::Vector2<T> \&}型。

\item{{\bf 機能}}

コピーコンストラクタ。{\tt src}で初期化する。

\end{itemize}

\subsubsubsection{代入演算子}

\begin{screen}
\begin{verbatim}
template<typename T>
const PS::Vector2<T> & PS::Vector2<T>::operator = 
                       (const PS::Vector2<T> & rhs);
\end{verbatim}
\end{screen}

\begin{itemize}

\item{{\bf 引数}}

{\tt rhs}: 入力。{\tt const PS::Vector2<T> \&}型。

\item{{\bf 返り値}}

{\tt const PS::Vector2<T> \&}型。{\tt rhs}のx,yの値を自身のメンバx,yに
代入し自身の参照を返す。代入演算子。

\end{itemize}

\subsubsubsection{加減算}

\begin{screen}
\begin{verbatim}
template<typename T>
PS::Vector2<T> PS::Vector2<T>::operator + 
               (const PS::Vector2<T> & rhs) const;
\end{verbatim}
\end{screen}

\begin{itemize}

\item{{\bf 引数}}

{\tt rhs}: 入力。{\tt const PS::Vector2<T> \&}型。

\item{{\bf 返り値}}

{\tt PS::Vector2<T> }型。{\tt rhs}のx,yの値と自身のメンバx,yの値の和を
取った値を返す。

\end{itemize}

\begin{screen}
\begin{verbatim}
template<typename T>
const PS::Vector2<T> & PS::Vector2<T>::operator += 
                       (const PS::Vector2<T> & rhs);
\end{verbatim}
\end{screen}

\begin{itemize}

\item{{\bf 引数}}

{\tt rhs}: 入力。{\tt const PS::Vector2<T> \&}型。

\item{{\bf 返り値}}

{\tt const PS::Vector2<T> \&}型。{\tt rhs}のx,yの値を自身のメンバx,yに足し、自
身を返す。

\end{itemize}

\begin{screen}
\begin{verbatim}
template<typename T>
PS::Vector2<T> PS::Vector2<T>::operator - 
               (const PS::Vector2<T> & rhs) const;
\end{verbatim}
\end{screen}

\begin{itemize}

\item{{\bf 引数}}

{\tt rhs}: 入力。{\tt const PS::Vector2<T> \&}型。

\item{{\bf 返り値}}

{\tt PS::Vector2<T> }型。{\tt rhs}のx,yの値と自身のメンバx,yの値の差を
取った値を返す。

\end{itemize}

\begin{screen}
\begin{verbatim}
template<typename T>
const PS::Vector2<T> & PS::Vector2<T>::operator -= 
                       (const PS::Vector2<T> & rhs);
\end{verbatim}
\end{screen}

\begin{itemize}

\item{{\bf 引数}}

{\tt rhs}: 入力。{\tt const PS::Vector2<T> \&}型。

\item{{\bf 返り値}}

{\tt const PS::Vector2<T> \&}型。自身のメンバx,yから{\tt rhs}のx,yを引
き自身を返す。

\end{itemize}

\subsubsubsection{ベクトルスカラ積}

\begin{screen}
\begin{verbatim}
template<typename T>
PS::Vector2<T> PS::Vector2<T>::operator * (const T s) const;
\end{verbatim}
\end{screen}

\begin{itemize}

\item{{\bf 引数}}

{\tt s}: 入力。{\tt const T}型。

\item{{\bf 返り値}}

{\tt PS::Vector2<T>}型。自身のメンバx,yそれぞれに{\tt s}をかけた値を返
す。

\end{itemize}

\begin{screen}
\begin{verbatim}
template<typename T>
const PS::Vector2<T> & PS::Vector2<T>::operator *= (const T s);
\end{verbatim}
\end{screen}

\begin{itemize}

\item{{\bf 引数}}

{\tt rhs}: 入力。{\tt const T}型。

\item{{\bf 返り値}}

{\tt const PS::Vector2<T> \&}型。自身のメンバx,yそれぞれに{\tt s}をかけ
自身を返す。

\end{itemize}

\begin{screen}
\begin{verbatim}
template<typename T>
PS::Vector2<T> PS::Vector2<T>::operator / (const T s) const;
\end{verbatim}
\end{screen}

\begin{itemize}

\item{{\bf 引数}}

{\tt s}: 入力。{\tt const T}型。

\item{{\bf 返り値}}

{\tt PS::Vector2<T>}型。自身のメンバx,yそれぞれを{\tt s}で割った値を返
す。

\end{itemize}

\begin{screen}
\begin{verbatim}
template<typename T>
const PS::Vector2<T> & PS::Vector2<T>::operator /= (const T s);
\end{verbatim}
\end{screen}

\begin{itemize}

\item{{\bf 引数}}

{\tt rhs}: 入力。{\tt const T}型。

\item{{\bf 返り値}}

{\tt const PS::Vector2<T> \&}型。自身のメンバx,yそれぞれを{\tt s}で割り
自身を返す。

\end{itemize}

\subsubsubsection{内積、外積}

\begin{screen}
\begin{verbatim}
template<typename T>
T PS::Vector2<T>::operator * (const PS::Vector2<T> & rhs) const;
\end{verbatim}
\end{screen}

\begin{itemize}

\item{{\bf 引数}}

{\tt rhs}: 入力。{\tt const PS::Vector2<T> \&}型。

\item{{\bf 返り値}}

{\tt T}型。自身と{\tt rhs}の内積を取った値を返す。

\end{itemize}

\begin{screen}
\begin{verbatim}
template<typename T>
T PS::Vector2<T>::operator ^ (const PS::Vector2<T> & rhs) const;
\end{verbatim}
\end{screen}

\begin{itemize}

\item{{\bf 引数}}

{\tt rhs}: 入力。{\tt const PS::Vector2<T> \&}型。

\item{{\bf 返り値}}

{\tt T}型。自身と{\tt rhs}の外積を取った値を返す。

\end{itemize}

\subsubsubsection{{\tt Vector2<U>}への型変換}

\begin{screen}
\begin{verbatim}
template<typename T>
template <typename U>
PS::Vector2<T>::operator PS::Vector2<U> () const;
\end{verbatim}
\end{screen}

\begin{itemize}

\item{{\bf 引数}}

  なし。

\item{{\bf 返り値}}

  {\tt const PS::Vector2<U>}型。

\item{{\bf 機能}}

  {\tt const PS::Vector2<T>}型を{\tt const PS::Vector2<U>}型にキャ
  ストする。

\end{itemize}




\subsubsection{PS::Vector3}

PS::Vecotr3はx, y, zの2要素を持つ。これらに対する様々なAPIや演算子を定
義した。それらの宣言を以下に記述する。この節ではこれらについて詳しく記
述する。
\begin{lstlisting}[caption=Vector3]
namespace ParticleSimulator{
    template <typename T>
    class Vector3{
    public:
        //メンバ変数は以下の二つのみ。
        T x, y, z;

        //コンストラクタ
        Vector3() : x(T(0)), y(T(0)), z(T(0)) {}
        Vector3(const T _x, const T _y, const T _z) : x(_x), y(_y), z(_z) {}
        Vector3(const T s) : x(s), y(s), z(s) {}
        Vector3(const Vector3 & src) : x(src.x), y(src.y), z(src.z) {}

        //代入演算子
        const Vector3 & operator = (const Vector3 & rhs);

        //加減算
        Vector3 operator + (const Vector3 & rhs) const;
        const Vector3 & operator += (const Vector3 & rhs);
        Vector3 operator - (const Vector3 & rhs) const;
        const Vector3 & operator -= (const Vector3 & rhs);

        //ベクトルスカラ積
        Vector3 operator * (const T s) const;
        const Vector3 & operator *= (const T s);
        friend Vector3 operator * (const T s, const Vector3 & v);
        Vector3 operator / (const T s) const;
        const Vector3 & operator /= (const T s);

        //内積
        T operator * (const Vector3 & rhs) const;

        //外積(返り値はスカラ!!)
        T operator ^ (const Vector3 & rhs) const;

        //Vector3<U>への型変換
        template <typename U>
        operator Vector3<U> () const;
    };
}
\end{lstlisting}
%%%%%%%%%%%%%%%%%%%%%%%%%%%%%
\subsubsubsection{コンストラクタ}
\mbox{}
%%%%%%%%%%%%%%%%%%%%%%%%%%%%%
%%%%%%%%%%%%%%%%%%%%%%%%%%%%%
\begin{screen}
\begin{verbatim}
template<typename T>
PS::Vector3<T>()
\end{verbatim}
\end{screen}

\begin{itemize}

\item{{\bf 引数}}

なし。

\item{{\bf 機能}}

デフォルトコンストラクタ。メンバx,yは0で初期化される。

\end{itemize}

%%%%%%%%%%%%%%%%%%%%%%%%%%%%%
\begin{screen}
\begin{verbatim}
template<typename T>
PS::Vector3<T>(const T _x, const T _y)
\end{verbatim}
\end{screen}

\begin{itemize}

\item{{\bf 引数}}

{\tt \_x}: 入力。{\tt const T}型。

{\tt \_y}: 入力。{\tt const T}型。

\item{{\bf 機能}}

メンバ{\tt x}、{\tt y}をそれぞれ{\tt \_x}、{\tt \_y}で初期化する。

\end{itemize}

%%%%%%%%%%%%%%%%%%%%%%%%%%%%%
\begin{screen}
\begin{verbatim}
template<typename T>
PS::Vector3<T>(const T s);
\end{verbatim}
\end{screen}

\begin{itemize}

\item{{\bf 引数}}

{\tt s}: 入力。{\tt const T}型。

\item{{\bf 機能}}

メンバ{\tt x}、{\tt y}を両方とも{\tt s}の値で初期化する。

\end{itemize}

%%%%%%%%%%%%%%%%%%%%%%%%%%%%%
\begin{screen}
\begin{verbatim}
template<typename T>
PS::Vector3<T>(const PS::Vector3<T> & src)
\end{verbatim}
\end{screen}

\begin{itemize}

\item{{\bf 引数}}

{\tt src}: 入力。{\tt const PS::Vector3<T> \&}型。

\item{{\bf 機能}}

コピーコンストラクタ。{\tt src}で初期化する。

\end{itemize}

%%%%%%%%%%%%%%%%%%%%%%%%%%%%%
\subsubsubsection{代入演算子}
\mbox{}
%%%%%%%%%%%%%%%%%%%%%%%%%%%%%

%%%%%%%%%%%%%%%%%%%%%%%%%%%%%
\begin{screen}
\begin{verbatim}
template<typename T>
const PS::Vector3<T> & PS::Vector3<T>::operator = 
                       (const PS::Vector3<T> & rhs);
\end{verbatim}
\end{screen}

\begin{itemize}

\item{{\bf 引数}}

{\tt rhs}: 入力。{\tt const PS::Vector3<T> \&}型。

\item{{\bf 返り値}}

{\tt const PS::Vector3<T> \&}型。{\tt rhs}のx,yの値を自身のメンバx,yに
代入し自身の参照を返す。代入演算子。

\end{itemize}


%%%%%%%%%%%%%%%%%%%%%%%%%%%%%
\subsubsubsection{加減算}
\mbox{}
%%%%%%%%%%%%%%%%%%%%%%%%%%%%%

%%%%%%%%%%%%%%%%%%%%%%%%%%%%%
\begin{screen}
\begin{verbatim}
template<typename T>
PS::Vector3<T> PS::Vector3<T>::operator + 
               (const PS::Vector3<T> & rhs) const;
\end{verbatim}
\end{screen}

\begin{itemize}

\item{{\bf 引数}}

{\tt rhs}: 入力。{\tt const PS::Vector3<T> \&}型。

\item{{\bf 返り値}}

{\tt PS::Vector3<T> }型。{\tt rhs}のx,yの値と自身のメンバx,yの値の和を
取った値を返す。

\end{itemize}


%%%%%%%%%%%%%%%%%%%%%%%%%%%%%
\begin{screen}
\begin{verbatim}
template<typename T>
const PS::Vector3<T> & PS::Vector3<T>::operator += 
                       (const PS::Vector3<T> & rhs);
\end{verbatim}
\end{screen}

\begin{itemize}

\item{{\bf 引数}}

{\tt rhs}: 入力。{\tt const PS::Vector3<T> \&}型。

\item{{\bf 返り値}}

{\tt const PS::Vector3<T> \&}型。{\tt rhs}のx,yの値を自身のメンバx,yに足し、自
身を返す。

\end{itemize}


%%%%%%%%%%%%%%%%%%%%%%%%%%%%%
\begin{screen}
\begin{verbatim}
template<typename T>
PS::Vector3<T> PS::Vector3<T>::operator - 
               (const PS::Vector3<T> & rhs) const;
\end{verbatim}
\end{screen}

\begin{itemize}

\item{{\bf 引数}}

{\tt rhs}: 入力。{\tt const PS::Vector3<T> \&}型。

\item{{\bf 返り値}}

{\tt PS::Vector3<T> }型。{\tt rhs}のx,yの値と自身のメンバx,yの値の差を
取った値を返す。

\end{itemize}


%%%%%%%%%%%%%%%%%%%%%%%%%%%%%
\begin{screen}
\begin{verbatim}
template<typename T>
const PS::Vector3<T> & PS::Vector3<T>::operator -= 
                       (const PS::Vector3<T> & rhs);
\end{verbatim}
\end{screen}

\begin{itemize}

\item{{\bf 引数}}

{\tt rhs}: 入力。{\tt const PS::Vector3<T> \&}型。

\item{{\bf 返り値}}

{\tt const PS::Vector3<T> \&}型。自身のメンバx,yから{\tt rhs}のx,yを引
き自身を返す。

\end{itemize}

%%%%%%%%%%%%%%%%%%%%%%%%%%%%%
\subsubsubsection{ベクトルスカラ積}
\mbox{}
%%%%%%%%%%%%%%%%%%%%%%%%%%%%%

%%%%%%%%%%%%%%%%%%%%%%%%%%%%%
\begin{screen}
\begin{verbatim}
template<typename T>
PS::Vector3<T> PS::Vector3<T>::operator * (const T s) const;
\end{verbatim}
\end{screen}

\begin{itemize}

\item{{\bf 引数}}

{\tt s}: 入力。{\tt const T}型。

\item{{\bf 返り値}}

{\tt PS::Vector3<T>}型。自身のメンバx,yそれぞれに{\tt s}をかけた値を返
す。

\end{itemize}


%%%%%%%%%%%%%%%%%%%%%%%%%%%%%
\begin{screen}
\begin{verbatim}
template<typename T>
const PS::Vector3<T> & PS::Vector3<T>::operator *= (const T s);
\end{verbatim}
\end{screen}

\begin{itemize}

\item{{\bf 引数}}

{\tt rhs}: 入力。{\tt const T}型。

\item{{\bf 返り値}}

{\tt const PS::Vector3<T> \&}型。自身のメンバx,yそれぞれに{\tt s}をかけ
自身を返す。

\end{itemize}


%%%%%%%%%%%%%%%%%%%%%%%%%%%%%
\begin{screen}
\begin{verbatim}
template<typename T>
PS::Vector3<T> PS::Vector3<T>::operator / (const T s) const;
\end{verbatim}
\end{screen}

\begin{itemize}

\item{{\bf 引数}}

{\tt s}: 入力。{\tt const T}型。

\item{{\bf 返り値}}

{\tt PS::Vector3<T>}型。自身のメンバx,yそれぞれを{\tt s}で割った値を返
す。

\end{itemize}


%%%%%%%%%%%%%%%%%%%%%%%%%%%%%
\begin{screen}
\begin{verbatim}
template<typename T>
const PS::Vector3<T> & PS::Vector3<T>::operator /= (const T s);
\end{verbatim}
\end{screen}

\begin{itemize}

\item{{\bf 引数}}

{\tt rhs}: 入力。{\tt const T}型。

\item{{\bf 返り値}}

{\tt const PS::Vector3<T> \&}型。自身のメンバx,yそれぞれを{\tt s}で割り
自身を返す。

\end{itemize}


%%%%%%%%%%%%%%%%%%%%%%%%%%%%%
\subsubsubsection{内積、外積}
\mbox{}
%%%%%%%%%%%%%%%%%%%%%%%%%%%%%

%%%%%%%%%%%%%%%%%%%%%%%%%%%%%
\begin{screen}
\begin{verbatim}
template<typename T>
T PS::Vector3<T>::operator * (const PS::Vector3<T> & rhs) const;
\end{verbatim}
\end{screen}

\begin{itemize}

\item{{\bf 引数}}

{\tt rhs}: 入力。{\tt const PS::Vector3<T> \&}型。

\item{{\bf 返り値}}

{\tt T}型。自身と{\tt rhs}の内積を取った値を返す。

\end{itemize}

%%%%%%%%%%%%%%%%%%%%%%%%%%%%%
\begin{screen}
\begin{verbatim}
template<typename T>
T PS::Vector3<T>::operator ^ (const PS::Vector3<T> & rhs) const;
\end{verbatim}
\end{screen}

\begin{itemize}

\item{{\bf 引数}}

{\tt rhs}: 入力。{\tt const PS::Vector3<T> \&}型。

\item{{\bf 返り値}}

{\tt T}型。自身と{\tt rhs}の外積を取った値を返す。

\end{itemize}


%%%%%%%%%%%%%%%%%%%%%%%%%%%%%
\subsubsubsection{{\tt Vector3<U>}への型変換}
\mbox{}
%%%%%%%%%%%%%%%%%%%%%%%%%%%%%

%%%%%%%%%%%%%%%%%%%%%%%%%%%%%
\begin{screen}
\begin{verbatim}
template<typename T>
template <typename U>
PS::Vector3<T>::operator PS::Vector3<U> () const;
\end{verbatim}
\end{screen}

\begin{itemize}

\item{{\bf 引数}}

  なし

\item{{\bf 返り値}}

  {\tt const PS::Vector3<U>}型。

\item{{\bf 機能}}

  {\tt const PS::Vector3<T>}型を{\tt const PS::Vector3<U>}型にキャ
  ストする。

\end{itemize}


\subsubsection{ベクトル型のラッパー}

ベクトル型のラッパーの定義を以下に示す。
\begin{lstlisting}[caption=vectorwrapper]
namespace ParticleSimulator{
    typedef Vector2<F32> F32vec2;
    typedef Vector3<F32> F32vec3;
    typedef Vector2<F64> F64vec2;
    typedef Vector3<F64> F64vec3;
#ifdef PARTICLE_SIMULATOR_TOW_DIMENSION
    typedef F32vec2 F32vec;
    typedef F64vec2 F64vec;
#else
    typedef F32vec3 F32vec;
    typedef F64vec3 F64vec;
#endif
}
namespace PS = ParticleSimulator;
\end{lstlisting}

すなわちPS::F32vec2, PS::F32vec3, PS::F64vec2, PS::F64vec3はそれぞれ単
精度2次元ベクトル、倍精度2次元ベクトル、単精度3次元ベクトル、倍精度3次
元ベクトルである。FDPSで扱う空間座標系を2次元とした場合、PS::F32vecと
PS::F64vecはそれぞれ単精度2次元ベクトル、倍精度2次元ベクトルとなる。一
方、FDPSで扱う空間座標系を3次元とした場合、PS::F32vecとPS::F64vecはそ
れぞれ単精度3次元ベクトル、倍精度3次元ベクトルとなる。





\subsection{対称行列型}

\subsubsection{概要}

対称行列型には2x2対称行列型PS::MatrixSym2と3x3対称行列型PS::MatrixSym3
がある。まずこれら2つを記述する。最後にこれら対称行列型のラッパーにつ
いて記述する。

\subsubsection{PS::MatrixSym2}

PS::MatrixSym2はxx, yy, xyの3要素を持つ。これらに対する様々なAPIや演算
子を定義した。それらの宣言を以下に記述する。この節ではこれらについて詳
しく記述する。
\begin{lstlisting}[caption=MatrixSym2]
namespace ParticleSimulator{
    template<class T>
    class MatrixSym2{
    public:
        // メンバ変数3要素
        T xx, yy, xy;

        // コンストラクタ
        MatrixSym2() : xx(T(0)), yy(T(0)), xy(T(0)) {}
        MatrixSym2(const T _xx, const T _yy, const T _xy)
            : xx(_xx), yy(_yy), xy(_xy) {}
        MatrixSym2(const T s) : xx(s), yy(s), xy(s){}
        MatrixSym2(const MatrixSym2 & src) : xx(src.xx), yy(src.yy), xy(src.xy) {}

        // 代入演算子
        const MatrixSym2 & operator = (const MatrixSym2 & rhs);

        // 加減算
        MatrixSym2 operator + (const MatrixSym2 & rhs) const;
        const MatrixSym2 & operator += (const MatrixSym2 & rhs) const;
        MatrixSym2 operator - (const MatrixSym2 & rhs) const;
        const MatrixSym2 & operator -= (const MatrixSym2 & rhs) const;

        // トレースの計算
        T getTrace() const;

        // MatrixSym2<U>への型変換
        template <typename U>
        operator MatrixSym2<U> () const;
    }
}
namespace PS = ParticleSimulator;
\end{lstlisting}

%%%%%%%%%%%%%%%%%%%%%%%%%%%%%%%%%%%%%%%%%%%%%%%%%%%%%
\subsubsubsection{コンストラクタ}

\begin{screen}
\begin{verbatim}
template<typename T>
PS::MatrixSym2<T>();
\end{verbatim}
\end{screen}

\begin{itemize}

\item{{\bf 引数}}

なし。

\item{{\bf 機能}}

デフォルトコンストラクタ。メンバxx,yy,xyは0で初期化される。

\end{itemize}

\begin{screen}
\begin{verbatim}
template<typename T>
PS::MatrixSym2<T>(const T _xx,
                  const T _yy,
                  const T _xy);
\end{verbatim}
\end{screen}

\begin{itemize}

\item{{\bf 引数}}

{\tt \_xx}: 入力。{\tt const T}型。

{\tt \_yy}: 入力。{\tt const T}型。

{\tt \_xy}: 入力。{\tt const T}型。

\item{{\bf 機能}}

メンバ{\tt xx}、{\tt yy}、{\tt xy}をそれぞれ{\tt \_xx}、{\tt \_yy}、
{\tt \_xy}で初期化する。

\end{itemize}

\begin{screen}
\begin{verbatim}
template<typename T>
PS::MatrixSym2<T>(const T s);
\end{verbatim}
\end{screen}

\begin{itemize}

\item{{\bf 引数}}

{\tt s}: 入力。{\tt const T}型。

\item{{\bf 機能}}

メンバ{\tt xx}、{\tt yy}、{\tt xy}すべてを{\tt s}の値で初期化する。

\end{itemize}

\begin{screen}
\begin{verbatim}
template<typename T>
PS::MatrixSym2<T>(const PS::MatrixSym2<T> & src)
\end{verbatim}
\end{screen}

\begin{itemize}

\item{{\bf 引数}}

{\tt src}: 入力。{\tt const PS::MatrixSym2<T> \&}型。

\item{{\bf 機能}}

コピーコンストラクタ。{\tt src}で初期化する。

\end{itemize}

%%%%%%%%%%%%%%%%%%%%%%%%%%%%%%%%%%%%%%%%%%%%%%%%%%%%%
\subsubsubsection{代入演算子}

\begin{screen}
\begin{verbatim}
template<typename T>
const PS::MatrixSym2<T> & PS::MatrixSym2<T>::operator = 
                       (const PS::MatrixSym2<T> & rhs);
\end{verbatim}
\end{screen}

\begin{itemize}

\item{{\bf 引数}}

{\tt rhs}: 入力。{\tt const PS::MatrixSym2<T> \&}型。

\item{{\bf 返り値}}

{\tt const PS::MatrixSym2<T> \&}型。{\tt rhs}のxx,yy,xyの値を自身のメ
ンバxx,yy,xyに代入し自身の参照を返す。代入演算子。

\end{itemize}

%%%%%%%%%%%%%%%%%%%%%%%%%%%%%%%%%%%%%%%%%%%%%%%%%%%%%
\subsubsubsection{加減算}

\begin{screen}
\begin{verbatim}
template<typename T>
PS::MatrixSym2<T> PS::MatrixSym2<T>::operator + 
               (const PS::MatrixSym2<T> & rhs) const;
\end{verbatim}
\end{screen}

\begin{itemize}

\item{{\bf 引数}}

{\tt rhs}: 入力。{\tt const PS::MatrixSym2<T> \&}型。

\item{{\bf 返り値}}

{\tt PS::MatrixSym2<T> }型。{\tt rhs}のxx,yy,xyの値と自身のメンバ
xx,yy,xyの値の和を取った値を返す。

\end{itemize}

\begin{screen}
\begin{verbatim}
template<typename T>
const PS::MatrixSym2<T> & PS::MatrixSym2<T>::operator += 
                       (const PS::MatrixSym2<T> & rhs);
\end{verbatim}
\end{screen}

\begin{itemize}

\item{{\bf 引数}}

{\tt rhs}: 入力。{\tt const PS::MatrixSym2<T> \&}型。

\item{{\bf 返り値}}

{\tt const PS::MatrixSym2<T> \&}型。{\tt rhs}のxx,yy,xyの値を自身のメ
ンバxx,yy,xyに足し、自身を返す。

\end{itemize}

\begin{screen}
\begin{verbatim}
template<typename T>
PS::MatrixSym2<T> PS::MatrixSym2<T>::operator - 
               (const PS::MatrixSym2<T> & rhs) const;
\end{verbatim}
\end{screen}

\begin{itemize}

\item{{\bf 引数}}

{\tt rhs}: 入力。{\tt const PS::MatrixSym2<T> \&}型。

\item{{\bf 返り値}}

{\tt PS::MatrixSym2<T> }型。{\tt rhs}のxx,yy,xyの値と自身のメンバ
xx,yy,xyの値の差を取った値を返す。

\end{itemize}

\begin{screen}
\begin{verbatim}
template<typename T>
const PS::MatrixSym2<T> & PS::MatrixSym2<T>::operator -= 
                       (const PS::MatrixSym2<T> & rhs);
\end{verbatim}
\end{screen}

\begin{itemize}

\item{{\bf 引数}}

{\tt rhs}: 入力。{\tt const PS::MatrixSym2<T> \&}型。

\item{{\bf 返り値}}

{\tt const PS::MatrixSym2<T> \&}型。自身のメンバxx,yy,xyから{\tt rhs}
のxx,yy,xyを引き自身を返す。

\end{itemize}

%%%%%%%%%%%%%%%%%%%%%%%%%%%%%%%%%%%%%%%%%%%%%%%%%%%%%
\subsubsubsection{トレースの計算}

\begin{screen}
\begin{verbatim}
template<typename T>
T PS::MatrixSym2<T>::getTrace() const;
\end{verbatim}
\end{screen}

\begin{itemize}

\item{{\bf 引数}}

なし

\item{{\bf 返り値}}

{\tt T}型。

\item{{\bf 機能}}

  トレースを計算し、その結果を返す。

\end{itemize}

%%%%%%%%%%%%%%%%%%%%%%%%%%%%%%%%%%%%%%%%%%%%%%%%%%%%%
\subsubsubsection{{\tt MatrixSym2<U>}への型変換}

\begin{screen}
\begin{verbatim}
template<typename T>
template<typename U>
PS::MatrixSym2<T>::operator PS::MatrixSym2<U> () const;
\end{verbatim}
\end{screen}

\begin{itemize}

\item{{\bf 引数}}

  なし。

\item{{\bf 返り値}}

{\tt const PS::MatrixSym2<U>}型。

\item{{\bf 機能}}

\redtext{{\tt const PS::MatrixSym2<T>}型を{\tt const
    PS::MatrixSym2<U>}型にキャストする}

\end{itemize}



\subsubsection{PS::MatrixSym3}

PS::MatrixSym3はxx, yy, zz, xy, xz, yzの6要素を持つ。これらに対する様々
なAPIや演算子を定義した。それらの宣言を以下に記述する。この節ではこれ
らについて詳しく記述する。
\begin{lstlisting}[caption=MatrixSym3]
namespace ParticleSimulator{
    template<class T>
    class MatrixSym3{
    public:
        // メンバ変数6要素
        T xx, yy, zz, xy, xz, yz;

        // コンストラクタ
        MatrixSym3() : xx(T(0)), yy(T(0)), zz(T(0)),
                       xy(T(0)), xz(T(0)), yz(T(0)) {}
        MatrixSym3(const T _xx, const T _yy, const T _zz,
                   const T _xy, const T _xz, const T _yz )
                       : xx(_xx), yy(_yy), zz(_zz),
                       xy(_xy), xz(_xz), yz(_yz) {}
        MatrixSym3(const T s) : xx(s), yy(s), zz(s),
                                xy(s), xz(s), yz(s) {}
        MatrixSym3(const MatrixSym3 & src) :
            xx(src.xx), yy(src.yy), zz(src.zz),
            xy(src.xy), xz(src.xz), yz(src.yz) {}

        // 代入演算子
        const MatrixSym3 & operator = (const MatrixSym3 & rhs);

        // 加減算
        MatrixSym3 operator + (const MatrixSym3 & rhs) const;
        const MatrixSym3 & operator += (const MatrixSym3 & rhs) const;
        MatrixSym3 operator - (const MatrixSym3 & rhs) const;
        const MatrixSym3 & operator -= (const MatrixSym3 & rhs) const;

        // トレースを取る
        T getTrace() const;

        // MatrixSym3<U>への型変換
        template <typename U>
        operator MatrixSym3<U> () const;
    }
}
namespace PS = ParticleSimulator;
\end{lstlisting}

%%%%%%%%%%%%%%%%%%%%%%%%%%%%%%%%%%%%%%%%%%%%%%%%%%%%%
\subsubsubsection{コンストラクタ}

\begin{screen}
\begin{verbatim}
template<typename T>
PS::MatrixSym3<T>();
\end{verbatim}
\end{screen}

\begin{itemize}

\item{{\bf 引数}}

なし。

\item{{\bf 機能}}

デフォルトコンストラクタ。6要素は0で初期化される。

\end{itemize}

\begin{screen}
\begin{verbatim}
template<typename T>
PS::MatrixSym3<T>(const T _xx,
                  const T _yy,
                  const T _zz,
                  const T _xy,
                  const T _xz,
                  const T _yz);
\end{verbatim}
\end{screen}

\begin{itemize}

\item{{\bf 引数}}

{\tt \_xx}: 入力。{\tt const T}型。

{\tt \_yy}: 入力。{\tt const T}型。

{\tt \_zz}: 入力。{\tt const T}型。

{\tt \_xy}: 入力。{\tt const T}型。

{\tt \_xz}: 入力。{\tt const T}型。

{\tt \_yz}: 入力。{\tt const T}型。

\item{{\bf 機能}}

メンバ{\tt xx}、{\tt yy}、{\tt zz}、{\tt xy}、{\tt xz}、{\tt yz}をそれ
ぞれ{\tt \_xx}、{\tt \_yy}、{\tt \_zz}、{\tt \_xy}、{\tt \_xz}、{\tt
  \_yz}で初期化する。

\end{itemize}

\begin{screen}
\begin{verbatim}
template<typename T>
PS::MatrixSym3<T>(const T s);
\end{verbatim}
\end{screen}

\begin{itemize}

\item{{\bf 引数}}

{\tt s}: 入力。{\tt const T}型。

\item{{\bf 機能}}

6要素すべてを{\tt s}の値で初期化する。

\end{itemize}

\begin{screen}
\begin{verbatim}
template<typename T>
PS::MatrixSym3<T>(const PS::MatrixSym3<T> & src)
\end{verbatim}
\end{screen}

\begin{itemize}

\item{{\bf 引数}}

{\tt src}: 入力。{\tt const PS::MatrixSym3<T> \&}型。

\item{{\bf 機能}}

コピーコンストラクタ。{\tt src}で初期化する。

\end{itemize}

%%%%%%%%%%%%%%%%%%%%%%%%%%%%%%%%%%%%%%%%%%%%%%%%%%%%%
\subsubsubsection{代入演算子}

\begin{screen}
\begin{verbatim}
template<typename T>
const PS::MatrixSym3<T> & PS::MatrixSym3<T>::operator = 
                       (const PS::MatrixSym3<T> & rhs);
\end{verbatim}
\end{screen}

\begin{itemize}

\item{{\bf 引数}}

{\tt rhs}: 入力。{\tt const PS::MatrixSym3<T> \&}型。

\item{{\bf 返り値}}

{\tt const PS::MatrixSym3<T> \&}型。{\tt rhs}の6要素それぞれの値を自
身の6要素それぞれに代入し自身の参照を返す。代入演算子。

\end{itemize}

%%%%%%%%%%%%%%%%%%%%%%%%%%%%%%%%%%%%%%%%%%%%%%%%%%%%%
\subsubsubsection{加減算}

\begin{screen}
\begin{verbatim}
template<typename T>
PS::MatrixSym3<T> PS::MatrixSym3<T>::operator + 
               (const PS::MatrixSym3<T> & rhs) const;
\end{verbatim}
\end{screen}

\begin{itemize}

\item{{\bf 引数}}

{\tt rhs}: 入力。{\tt const PS::MatrixSym3<T> \&}型。

\item{{\bf 返り値}}

{\tt PS::MatrixSym3<T> }型。{\tt rhs}の6要素それぞれの値と自身の6要
素の値の和を取った値を返す。

\end{itemize}

\begin{screen}
\begin{verbatim}
template<typename T>
const PS::MatrixSym3<T> & PS::MatrixSym3<T>::operator += 
                       (const PS::MatrixSym3<T> & rhs);
\end{verbatim}
\end{screen}

\begin{itemize}

\item{{\bf 引数}}

{\tt rhs}: 入力。{\tt const PS::MatrixSym3<T> \&}型。

\item{{\bf 返り値}}

{\tt const PS::MatrixSym3<T> \&}型。{\tt rhs}の6要素それぞれの値を自
身の6要素それぞれに足し、自身を返す。

\end{itemize}

\begin{screen}
\begin{verbatim}
template<typename T>
PS::MatrixSym3<T> PS::MatrixSym3<T>::operator - 
               (const PS::MatrixSym3<T> & rhs) const;
\end{verbatim}
\end{screen}

\begin{itemize}

\item{{\bf 引数}}

{\tt rhs}: 入力。{\tt const PS::MatrixSym3<T> \&}型。

\item{{\bf 返り値}}

{\tt PS::MatrixSym3<T> }型。{\tt rhs}の6要素それぞれの値と自身の6要
素それぞれの値の差を取った値を返す。

\end{itemize}

\begin{screen}
\begin{verbatim}
template<typename T>
const PS::MatrixSym3<T> & PS::MatrixSym3<T>::operator -= 
                       (const PS::MatrixSym3<T> & rhs);
\end{verbatim}
\end{screen}

\begin{itemize}

\item{{\bf 引数}}

{\tt rhs}: 入力。{\tt const PS::MatrixSym3<T> \&}型。

\item{{\bf 返り値}}

{\tt const PS::MatrixSym3<T> \&}型。自身の6要素それぞれから{\tt rhs}
の6要素それぞれを引き自身を返す。

\end{itemize}

%%%%%%%%%%%%%%%%%%%%%%%%%%%%%%%%%%%%%%%%%%%%%%%%%%%%%
\subsubsubsection{トレースの計算}

\begin{screen}
\begin{verbatim}
template<typename T>
T PS::MatrixSym3<T>::getTrace() const;
\end{verbatim}
\end{screen}

\begin{itemize}

\item{{\bf 引数}}

なし

\item{{\bf 返り値}}

{\tt T}型。

\item{{\bf 機能}}

  トレースを計算し、その結果を返す。

\end{itemize}

%%%%%%%%%%%%%%%%%%%%%%%%%%%%%%%%%%%%%%%%%%%%%%%%%%%%%
\subsubsubsection{{\tt MatrixSym3<U>}への型変換}

\begin{screen}
\begin{verbatim}
template<typename T>
template<typename U>
PS::MatrixSym3<T>::operator PS::MatrixSym3<U> () const;
\end{verbatim}
\end{screen}

\begin{itemize}

\item{{\bf 引数}}

  なし。

\item{{\bf 返り値}}

{\tt const PS::MatrixSym3<U>}型。

\item{{\bf 機能}}

\redtext{{\tt const PS::MatrixSym3<T>}型を{\tt const
    PS::MatrixSym3<U>}型にキャストする}

\end{itemize}



\subsubsection{行列型のラッパー}

対称行列型のラッパーの定義を以下に示す。
\begin{lstlisting}[caption=matrixsymwrapper]
namespace ParticleSimulator{
    typedef MatrixSym2<F32> F32mat2;
    typedef MatrixSym3<F32> F32mat3;
    typedef MatrixSym2<F64> F64mat2;
    typedef MatrixSym3<F64> F64mat3;
#ifdef PARTICLE_SIMULATOR_TOW_DIMENSION
    typedef F32mat2 F32mat;
    typedef F64mat2 F64mat;
#else
    typedef F32mat3 F32mat;
    typedef F64mat3 F64mat;
#endif
}
namespace PS = ParticleSimulator;
\end{lstlisting}

すなわちPS::F32mat2, PS::F32mat3, PS::F64mat2, PS::F64mat3はそれぞれ単
精度2x2対称行列、倍精度2x2対称行列、単精度3x3対称行列、倍精度3x3対称行
列である。FDPSで扱う空間座標系を2次元とした場合、PS::F32matと
PS::F64matはそれぞれ単精度2x2対称行列、倍精度2x2対称行列となる。一方、
FDPSで扱う空間座標系を3次元とした場合、PS::F32matとPS::F64matはそれぞ
れ単精度3x3対称行列、倍精度3x3対称行列となる。





\subsection{SEARCH\_MODE型}

\subsubsection{概要}

本節では、SEARCH\_MODE型について記述する。SEARCH\_MODE型は相互作用ツリー
クラスのテンプレート引数としてのみ使用されるものである。この型によって、
相互作用ツリークラスで計算する相互作用のモードを決定する。SEARCH\_MODE
型にはSEARCH\_MODE\_LONG, SEARCH\_MODE\_LONG\_CUTOFF,
SEARCH\_MODE\_GATHER, SEARCH\_MODE\_SCATTER, SEARCH\_MODE\_SYMMETRYが
ある。以下に、それぞれが対応する相互作用のモードについて記述する。

\subsubsection{SEARCH\_MODE\_LONG}

この型を使用するのは、遠くの粒子からの寄与を複数の粒子にまとめた超粒子
からの寄与として計算する場合である。開放境界条件における重力やクーロン
力に適用できる。

\subsubsection{SEARCH\_MODE\_LONG\_CUTOFF}

この型を使用するのは、遠くの粒子からの寄与を複数の粒子にまとめた超粒子
からの寄与として計算し、かつ有限の距離までの寄与しか計算しない場合であ
る。周期境界条件における重力やクーロン力(Particle Mesh法を並用)などに
適用できる。

\subsubsection{SEARCH\_MODE\_GATHER}

この型を使用するのは、相互作用の到達距離が有限でかつ、その到達距離がi
粒子の大きさで決まる場合である。

\subsubsection{SEARCH\_MODE\_SCATTER}

この型を使用するのは、相互作用の到達距離が有限でかつ、その到達距離がj
粒子の大きさで決まる場合である。

\subsubsection{SEARCH\_MODE\_SYMMETRY}

この型を使用するのは、相互作用の到達距離が有限でかつ、その到達距離がi,
j粒子両方の大きさで決まる場合である。



\subsection{列挙型}

\subsubsection{概要}

本節ではFDPSで定義されている列挙型について記述する。列挙型には
BOUNDARY\_CONDITION型が存在する。以下、各列挙型について記述する。

\subsubsection{BOUNDARY\_CONDITION型}
\label{sec:type_enum}

\subsubsubsection{概要}

BOUNDARY\_CONDITION型は境界条件を指定するためのデータ型である。これは
以下のように定義されている。
\begin{lstlisting}[caption=boundarycondition]
namespace ParticleSimulator{
    enum BOUNDARY_CONDITION{
        BOUNDARY_CONDITION_OPEN,
        BOUNDARY_CONDITION_PERIODIC_X,
        BOUNDARY_CONDITION_PERIODIC_Y,
        BOUNDARY_CONDITION_PERIODIC_Z,
        BOUNDARY_CONDITION_PERIODIC_XY,
        BOUNDARY_CONDITION_PERIODIC_XZ,
        BOUNDARY_CONDITION_PERIODIC_YZ,
        BOUNDARY_CONDITION_PERIODIC_XYZ,
        BOUNDARY_CONDITION_SHEARING_BOX,
        BOUNDARY_CONDITION_USER_DEFINED,
    };
}
namespace PS = ParticleSimulator;
\end{lstlisting}

以下にどの変数がどの境界条件に対応するかを記述する。

\subsubsubsection{PS::BOUNDARY\_CONDITION\_OPEN}

開放境界となる。

\subsubsubsection{PS::BOUNDARY\_CONDITION\_PERIODIC\_X}

x軸方向のみ周期境界、その他の軸方向は開放境界となる。周期の境界の下限
は閉境界、上限は開境界となっている。この境界の規定はすべての軸方向にあ
てはまる。

\subsubsubsection{PS::BOUNDARY\_CONDITION\_PERIODIC\_Y}

y軸方向のみ周期境界、その他の軸方向は開放境界となる。

\subsubsubsection{PS::BOUNDARY\_CONDITION\_PERIODIC\_Z}

z軸方向のみ周期境界、その他の軸方向は開放境界となる。

\subsubsubsection{PS::BOUNDARY\_CONDITION\_PERIODIC\_XY}

x, y軸方向のみ周期境界、その他の軸方向は開放境界となる。

\subsubsubsection{PS::BOUNDARY\_CONDITION\_PERIODIC\_XZ}

x, z軸方向のみ周期境界、その他の軸方向は開放境界となる。

\subsubsubsection{PS::BOUNDARY\_CONDITION\_PERIODIC\_YZ}

y, z軸方向のみ周期境界、その他の軸方向は開放境界となる。

\subsubsubsection{PS::BOUNDARY\_CONDITION\_PERIODIC\_XYZ}

x, y, z軸方向すべてが周期境界となる。

\subsubsubsection{PS::BOUNDARY\_CONDITION\_SHEARING\_BOX}

未実装。

\subsubsubsection{PS::BOUNDARY\_CONDITION\_USER\_DEFINED}

未実装。



%%\subsection{MPIデータ型}

%%\subsubsection{概要}

本節ではFDPSで定義されているMPIデータ型について記述する。

\subsubsection{PS::GetDataType$<$S32$>$()}

PS::S32に対応するMPIデータ型である。

\subsubsection{PS::GetDataType$<$S64$>$()}

PS::S64に対応するMPIデータ型である。

\subsubsection{PS::GetDataType$<$U32$>$()}

PS::U32に対応するMPIデータ型である。

\subsubsection{PS::GetDataType$<$U64$>$()}

PS::U64に対応するMPIデータ型である。

\subsubsection{PS::GetDataType$<$F32$>$()}

PS::F32に対応するMPIデータ型である。

\subsubsection{PS::GetDataType$<$F64$>$()}

PS::F64に対応するMPIデータ型である。

\subsubsection{PS::MPI\_F32VEC}

PS::F32vecに対応するMPIデータ型である。PS::F32vecは、2次元直交座標系を
扱っている場合には2次元ベクトル、3次元直交座標系を扱っている場合には3
次元ベクトルである。

\subsubsection{PS::MPI\_F64VEC}

PS::F64vecに対応するMPIデータ型である。PS::F64vecは、2次元直交座標系を
扱っている場合には2次元ベクトル、3次元直交座標系を扱っている場合には3
次元ベクトルである。




\newpage

%%%%%%%%%%%%%%%%%%%%%%%%%%%%%%%%%%%%%%%%%%%%%%%%%%%%%
\section{ユーザー定義クラス、ファンクタ}

\subsection{概要}

本節では、ユーザーが定義するクラスとファンクタについて記述する。ユーザー
定義クラスとなるのは、粒子の情報すべてを持つFullParticleクラス、ある相
互作用を計算する際にi粒子に必要な情報を持つEssentialParticleIクラス、
ある相互作用を計算する際にj粒子に必要な情報を持つEssentialParticleJク
ラス、ある相互作用(SEARCH\_MODE型がSEARCH\_MODE\_LONGまたは
SEARCH\_MODE\_LONG\_CUTOFFの場合に限る)を計算する際に超粒子に必要な情
報を持つSuperParticleJクラス、ツリーセルのモーメント情報を持つMomentク
ラス、相互作用の結果の情報を持つForceクラス、入出力ファイルのヘッダ情
報を持つヘッダクラスである。また、ユーザー定義ファンクタには、j粒子か
らi粒子への作用を計算するcalcForceEpEpファンクタ、超粒子からi粒子への
作用を計算するcalcForceSpEpファンクタがある。

この節で記述するのは、これらのクラスに関する規定である。ユーザーはこれ
らのクラスの間でのデータのやりとりや、ファンクタ内でのデータの加工につ
いてコードに書く必要がある。これらは上に挙げたクラスとファンクタのメン
バ関数内で行われる。以下、必要なメンバ関数とその規定について記述する。

\subsection{FullParticleクラス}
\label{sec:fullparticle}

\subsubsection{概要}

FullParticleクラスは粒子情報すべてを持つクラスである。FDPSはこのクラス
からいくつかの情報を読み取る。FDPSが情報を読み取るために、このクラスは
いくつかのメンバ関数を持つ必要がある。以下、この節の前提、常に必要なメ
ンバ関数と、場合によっては必要なメンバ関数について記述する。

\subsubsection{前提}

この節の中では、以下のように、名前空間ParticleSimulatorをPSと省略し、
FullParticleというクラスを例とする。FullParticleという名前は自由に変え
ることができる。
\begin{screen}
\begin{verbatim}
namespace PS = ParticleSimulator;
class FullParticle;
\end{verbatim}
\end{screen}

\subsubsection{必要なメンバ関数}

\subsubsubsection{概要}

常に必要なメンバ関数はgetPosとcopyFromForceである。getPosは
FullParticleの位置情報をFDPSに読み込ませるための関数で、copyFromForce
は計算された相互作用の結果をFullParticleに書き戻す関数である。これらの
メンバ関数の記述例と解説を以下に示す。

\subsubsubsection{getPos}

\begin{screen}
\begin{verbatim}
class FullParticle {
public:
    PS::F64vec pos;
    PS::F64vec getPos() const {
        return this->pos;
    }
};
\end{verbatim}
\end{screen}

\begin{itemize}

\item {\bf 前提}
  
  FullParticleのメンバ変数posはある1つの粒子の位置情報。このposのデー
  タ型はPS::F32vec型またはPS::F64vec型。
  
\item {\bf 引数}

  なし
  
\item {\bf 返値}

  PS::F32vec型またはPS::F64vec型。FullParticleクラスの位置情報を保持し
  たメンバ変数。
  
\item {\bf 機能}

  FullParticleクラスの位置情報を保持したメンバ変数を返す。
  
\item {\bf 備考}

  FullParticleクラスのメンバ変数posの変数名は変更可能。ただしこのposの
  データ型とメンバ関数FullParticle::getPosの返値のデータ型が一致してい
  ない場合の動作は保証しない。

\end{itemize}

\subsubsubsection{copyFromForce}

\begin{screen}
\begin{verbatim}
class Force {
public:
    PS::F64vec acc;
    PS::F64    pot;
};
class FullParticle {
public:
    PS::F64vec acceleration;
    PS::F64    potential;
    void copyFromForce(const Force & force) {
        this->acceleration = force.acc;
        this->potential    = force.pot;
    }
};
\end{verbatim}
\end{screen}

\begin{itemize}

\item {\bf 前提}

  Forceクラスは粒子の相互作用の計算結果を保持するクラス。

\item {\bf 引数}

  force: 入力。const Force \&型。粒子の相互作用の計算結果を保持。
  
\item {\bf 返値}

  なし。
  
\item {\bf 機能}

  粒子の相互作用の計算結果をFullParticleクラスへ書き戻す。Forceクラス
  のメンバ変数acc, potがそれぞれFullParticleクラスのメンバ変数
  acceleration, potentialに対応。
  
\item {\bf 備考}

  Forceクラスというクラス名とそのメンバ変数名は変更可能。FullParticle
  のメンバ変数名は変更可能。メンバ関数FullParticle::copyFromForceの引
  数名は変更可能。

\end{itemize}

\subsubsection{場合によっては必要なメンバ関数}

\subsubsubsection{概要}

本節では、場合によっては必要なメンバ関数について記述する。相互作用ツリー
クラスのSEARCH\_MODE型にSEARCH\_MODE\_LONG以外を用いる場合、粒子群クラ
スのファイル入出力APIを用いる場合、粒子群クラスのAPIである
ParticleSystem::adjustPositionIntoRootDomainを用いる場合、拡張機能の
Particle Meshクラスを用いる場合について必要となるメンバ関数を記述する。

\subsubsubsection{相互作用ツリークラスのSEARCH\_MODE型にSEARCH\_MODE\_LONG以外を用いる場合}

\subsubsubsubsection{getRsearch}

\begin{screen}
\begin{verbatim}
class FullParticle {
public:
    PS::F64 search_radius;
    PS::F64 getRsearch() const {
        return this->search_radius;
    }
};
\end{verbatim}
\end{screen}

\begin{itemize}

\item {\bf 前提}

  FullParticleクラスのメンバ変数search\_radiusはある1つの粒子の近傍粒
  子を探す半径の大きさ。このsearch\_radiusのデータ型はPS::F32型または
  PS::F64型。
  
\item {\bf 引数}

  なし
  
\item {\bf 返値}

  PS::F32型またはPS::F64型。 FullParticleクラスの近傍粒子を探す半径の
  大きさを保持したメンバ変数。
  
\item {\bf 機能}

  FullParticleクラスの近傍粒子を探す半径の大きさを保持したメンバ変数を
  返す。

\item {\bf 備考}

  FullParticleクラスのメンバ変数search\_radiusの変数名は変更可能。
  
\end{itemize}

\subsubsubsection{粒子群クラスのファイル入出力APIを用いる場合}

粒子群クラスのファイル入出力APIであるreadParticleAscii,
readParticleBinary, writeParticleAscii, writeParticleBinaryを使用する
ときにそれぞれreadAscii, readBinary, writeAscii, writeBinaryというメン
バ関数が必要となる。以下、readAsciiとreadBinaryの規定は同じであり,
writeAsciiとwriteBinaryの規定も同じである。以下、それぞれの規定につい
て記述する。

\subsubsubsubsection{readAscii, readBinary}

\begin{screen}
\begin{verbatim}
class FullParticle {
public:
    PS::S32 id;
    PS::F64 mass;
    PS::F64vec pos;
    void readAscii(FILE *fp) {
        fscanf(fp, "%d%lf%lf%lf%lf", &this->id, &this->mass,
               &this->pos[0], &this->pos[1], &this->pos[2]);
    }
    void readBinary(FILE *fp) {
        fscanf(fp, "%d%lf%lf%lf%lf", &this->id, &this->mass,
               &this->pos[0], &this->pos[1], &this->pos[2]);
    }
};
\end{verbatim}
\end{screen}

\begin{itemize}

\item {\bf 前提}

  粒子データの入力ファイルの1列目にはFullParticleクラスのメンバ変数id
  を表すデータが、2列目にはメンバ変数massを表すデータが、3、4、5列
  めにはメンバ変数posの第1、2、3要素が、それ以降の列にはデータがな
  いとする。ファイルの形式はアスキー形式(readAsciiの場合)、バイナリー
  形式(readBinaryの場合)とする。3次元直交座標系を選択したとする。

\item {\bf 引数}

  fp: FILE *型。粒子データの入力ファイルを指すファイルポインタ。
  
\item {\bf 返値}

  なし。
  
\item {\bf 機能}

  粒子データの入力ファイルからFullParticleクラスのid、mass、posの情報
  を読み取る。
  
\item {\bf 備考}

  なし。
  
\end{itemize}

\subsubsubsubsection{writeAscii, writeBinary}

\begin{screen}
\begin{verbatim}
class FullParticle {
public:
    PS::S32 id;
    PS::F64 mass;
    PS::F64vec pos;
    void writeAscii(FILE *fp) {
        fscanf(fp, "%d %lf %lf %lf %lf", this->id, this->mass,
               this->pos[0], this->pos[1], this->pos[2]);
    }
    void writeBinary(FILE *fp) {
        fscanf(fp, "%d %lf %lf %lf %lf", this->id, this->mass,
               this->pos[0], this->pos[1], this->pos[2]);
    }
};
\end{verbatim}
\end{screen}

\begin{itemize}

\item {\bf 前提}

  粒子データの出力ファイルの1列目にはFullParticleクラスのメンバ変数id
  を表すデータが、2列目にはメンバ変数massを表すデータが、3、4、5列
  めにはメンバ変数posの第1、2、3要素が、それ以降の列にはデータがな
  いとする。ファイルの形式はアスキー形式(writeAsciiの場合)、バイナリー
  形式(writeBinaryの場合)とする。3次元直交座標系を選択したとする。

\item {\bf 引数}

  fp: FILE *型。粒子データの出力ファイルを指すファイルポインタ。
  
\item {\bf 返値}

  なし。
  
\item {\bf 機能}

  粒子データの出力ファイルへFullParticleクラスのメンバ変数id、mass、
  posの情報を書き出す。
  
\item {\bf 備考}

  なし。
  
\end{itemize}

\subsubsubsection{粒子群クラスのadjustPositionIntoRootDomainを用いる場合}

\subsubsubsubsection{setPos}

\begin{screen}
\begin{verbatim}
class FullParticle {
public:
    PS::F64vec pos;
    void setPos(const PS::F64vec pos_new) {
        this->pos = pos_new;
    }
};
\end{verbatim}
\end{screen}

\begin{itemize}

\item {\bf 前提}

  FullParticleクラスのメンバ変数posは1つの粒子の位置情報。このposのデー
  タ型はPS::F32vecまたはPS::F64vec。

\item {\bf 引数}

  pos\_new: 入力。const PS::F32vecまたはconst PS::F64vec型。FDPS側で修
  正した粒子の位置情報。

\item {\bf 返値}

  なし。
  
\item {\bf 機能}

  FDPSが修正した粒子の位置情報をFullParticleクラスの位置情報に書き込む。

\item {\bf 備考}

  FullParticleクラスのメンバ変数posの変数名は変更可能。メンバ関数
  FullParticle::setPosの引数名pos\_newは変更可能。posとpos\_newのデー
  タ型が異なる場合の動作は保証しない。

\end{itemize}

\subsubsubsection{Particle Meshクラスを用いる場合}

Particle Meshクラスを用いる場合には、メンバ関数getChargeParticleMeshと
copyFromForceParticleMeshを用意する必要がある。以下にそれぞれの規定を
記述する。

\subsubsubsubsection{getChargeParticleMesh}

\begin{screen}
\begin{verbatim}
class FullParticle {
public:
    PS::F64 mass;
    PS::F64 getChargeParticleMesh() const {
        return this->mass;
    }
};
\end{verbatim}
\end{screen}

\begin{itemize}

\item {\bf 前提}

  FullParticleクラスのメンバ変数massは1つの粒子の質量または電荷の情報
  を持つ変数。データ型はPS::F32またはPS::F64型。

\item {\bf 引数}

  なし。

\item {\bf 返値}

  PS::F32型またはPS::F64型。1つの粒子の質量または電荷の変数を返す。
  
\item {\bf 機能}

  1つの粒子の質量または電荷の変数を返す。

\item {\bf 備考}

  FullParticleクラスのメンバ変数massの変数名は変更可能。

\end{itemize}

\subsubsubsubsection{copyFromForceParticleMesh}

\begin{screen}
\begin{verbatim}
class FullParticle {
public:
    PS::F64vec accelerationFromPM;
    void copyFromForceParticleMesh(const PS::F32vec & acc_pm) {
        this->accelerationFromPM = acc_pm;
    }
};
\end{verbatim}
\end{screen}

\begin{itemize}

\item {\bf 前提}

  FullParticleクラスのメンバ変数accelerationFromPM\_pmは1つの粒子の
  Particle Meshによる力の情報を保持する変数。この
  accelerationFromPM\_pmのデータ型はPS::F32vecまたはPS::F64vec。

\item {\bf 引数}

  acc\_pm: const PS::F32vec型またはconst PS::F64vec型。1つの粒子の
  Particle Meshによる力の計算結果。

\item {\bf 返値}

  なし。
  
\item {\bf 機能}

  1つの粒子のParticle Meshによる力の計算結果をこの粒子のメンバ変数に
  書き込む。
  
\item {\bf 備考}

  FullParticleクラスのメンバ変数acc\_pmの変数名は変更可能。メンバ関数
  FullParticle::copyFromForceParticleMeshの引数acc\_pmの引数名は変更可
  能。

\end{itemize}


\subsection{EssentialParticleIクラス}
\label{sec:essentialparticlei}

\subsubsection{概要}

EssentialParticleIクラスは相互作用の計算に必要なi粒子情報を持つクラス
である。EssentialParticleIクラスはFullParticleクラス
(節\ref{sec:fullparticle})のサブセットであり、FDPSの内部では、このクラ
スがFullParticleクラスから情報を読み取る。情報を読み取るために、このク
ラスはいくつかのメンバ関数を持つ必要がある。以下、この節の前提、常に必
要なメンバ関数と、場合によっては必要なメンバ関数について記述する。

\subsubsection{前提}

この節の中では、名前空間ParticleSimulatorをPSと省略する。このクラスの
クラス名をEssentialParticleIとする。また、粒子すべての情報を持つクラス
のクラス名をFullParticleとする。このFullParticleは
節\ref{sec:fullparticle}のクラスFullParticleと同一のものである。
EssentialParticleI, FullParticleというクラス名は変更可能である。

ParticleSimulatorをPSと省略すること、EssentialParticleIとFullParticle
の宣言は以下の通りである。
\begin{screen}
\begin{verbatim}
namespace PS = ParticleSimulator;
class FullParticle;
class EssentialParticleI;
\end{verbatim}
\end{screen}

\subsubsection{必要なメンバ関数}

\subsubsubsection{概要}

常に必要なメンバ関数はgetPosとCopyfromFPである。getPosは
EssentialParticleIクラスの位置情報をFDPSに読み込ませるための関数で、
copyFromFPはFullParticleクラスの情報をEssentialParticleIクラスに書きこ
む関数である。これらのメンバ関数の記述例と解説を以下に示す。

\subsubsubsection{getPos}

\begin{screen}
\begin{verbatim}
class EssentialParticleI {
public:
    PS::F64vec pos;
    PS::F64vec getPos() const {
        return this->pos;
    }
};
\end{verbatim}
\end{screen}

\begin{itemize}

\item {\bf 前提}
  
  EssentialParticleIのメンバ変数posはある1つの粒子の位置情報。このposのデー
  タ型はPS::F64vec型。
  
\item {\bf 引数}

  なし
  
\item {\bf 返値}

  PS::F64vec型。EssentialParticleIクラスの位置情報を保持したメンバ変数。
  
\item {\bf 機能}

  EssentialParticleIクラスの位置情報を保持したメンバ変数を返す。
  
\item {\bf 備考}

  EssentialParticleIクラスのメンバ変数posの変数名は変更可能。

\end{itemize}

\subsubsubsection{copyFromFP}

\begin{screen}
\begin{verbatim}
class FullParticle {
public:
    PS::S64    identity;
    PS::F64    mass;
    PS::F64vec position;
    PS::F64vec velocity;
    PS::F64vec acceleration;
    PS::F64    potential;
};
class EssentialParticleI {
public:
    PS::S64    id;
    PS::F64vec pos;
    void copyFromFP(const FullParticle & fp) {
        this->id  = fp.identity;
        this->pos = fp.position;
    }
};
\end{verbatim}
\end{screen}

\begin{itemize}

\item {\bf 前提}

  FullParticleクラスのメンバ変数identity, positionとEssentialParticleI
  クラスのメンバ変数id, posはそれぞれ対応する情報を持つ。

\item {\bf 引数}

  fp: 入力。const FullParticle \&型。FullParticleクラスの情報を持つ。
  
\item {\bf 返値}

  なし。
  
\item {\bf 機能}

  FullParticleクラスの持つ1粒子の情報の一部をEssnetialParticleIクラス
  に書き込む。
  
\item {\bf 備考}

  FullParticleクラスのメンバ変数の変数名、EssentialParticleIクラスのメ
  ンバ変数の変数名は変更可能。メンバ関数EssentialParticleI::copyFromFP
  の引数名は変更可能。EssentialParticleIクラスの粒子情報はFullParticle
  クラスの粒子情報のサブセット。対応する情報を持つメンバ変数同士のデー
  タ型が一致している必要はないが、実数型とベクトル型(または整数型とベ
  クトル型)という違いがある場合に正しく動作する保証はない。

\end{itemize}

\subsubsection{場合によっては必要なメンバ関数}

\subsubsubsection{概要}

本節では、場合によっては必要なメンバ関数について記述する。相互作用ツリー
クラスのSEARCH\_MODE型にSEARCH\_MODE\_GATHERまたは
SEARCH\_MODE\_SYMMETRYを用いる場合に必要となるメンバ関数ついて記述
する。

\subsubsubsection{相互作用ツリークラスのSEARCH\_MODE型にSEARCH\_MODE\_GATHERまたはSEARCH\_MODE\_SYMMETRYを用いる場合}

\subsubsubsubsection{getRsearch}

\begin{screen}
\begin{verbatim}
class EssentialParticleI {
public:
    PS::F64 search_radius;
    PS::F64 getRsearch() const {
        return this->search_radius;
    }
};
\end{verbatim}
\end{screen}

\begin{itemize}

\item {\bf 前提}

  EssentialParticleIクラスのメンバ変数search\_radiusはある1つの粒子の
  近傍粒子を探す半径の大きさ。このsearch\_radiusのデータ型はPS::F32型
  またはPS::F64型。
  
\item {\bf 引数}

  なし
  
\item {\bf 返値}

  PS::F32型またはPS::F64型。 EssentialParticleIクラスの近傍粒子を探す
  半径の大きさを保持したメンバ変数。
  
\item {\bf 機能}

  EssentialParticleIクラスの近傍粒子を探す半径の大きさを保持したメンバ
  変数を返す。

\item {\bf 備考}

  EssentialParticleIクラスのメンバ変数search\_radiusの変数名は変更可能。
  
\end{itemize}

\subsection{EssentialParticleJクラス}
\label{sec:essentialparticlej}

\subsubsection{概要}

EssentialParticleJクラスは相互作用の計算に必要なj粒子情報を持つクラス
である。FDPSの内部では、このクラスがFullParticleクラスから情報を読み取
る。情報を読み取るために、このクラスはいくつかのメンバ関数を持つ必要が
ある。以下、この節の前提、常に必要なメンバ関数と、場合によっては必要な
メンバ関数について記述する。

\subsubsection{前提}

この節の中では、名前空間ParticleSimulatorをPSと省略する。このクラスの
クラス名をEssentialParticleJとする。また、粒子すべての情報を持つクラス
のクラス名をFullParticleとする。このFullParticleは
節\ref{sec:fullparticle}のクラスFullParticleと同一のものである。
EssentialParticleJ, FullParticleというクラス名は変更可能である。

ParticleSimulatorをPSと省略すること、EssentialParticleJとFullParticle
の宣言は以下の通りである。
\begin{screen}
\begin{verbatim}
namespace PS = ParticleSimulator;
class FullParticle;
class EssentialParticleJ;
\end{verbatim}
\end{screen}

\subsubsection{必要なメンバ関数}

\subsubsubsection{概要}

常に必要なメンバ関数はgetPosとCopyfromFPである。getPosは
EssentialParticleJクラスの位置情報をFDPSに読み込ませるための関数で、
copyFromFPはFullParticleクラスの情報をEssentialParticleJクラスに書きこ
む関数である。これらのメンバ関数の記述例と解説を以下に示す。

\subsubsubsection{getPos}

\begin{screen}
\begin{verbatim}
class EssentialParticleJ {
public:
    PS::F64vec pos;
    PS::F64vec getPos() const {
        return this->pos;
    }
};
\end{verbatim}
\end{screen}

\begin{itemize}

\item {\bf 前提}
  
  EssentialParticleJのメンバ変数posはある1つの粒子の位置情報。このposのデー
  タ型はPS::F64vec型。
  
\item {\bf 引数}

  なし
  
\item {\bf 返値}

  PS::F64vec型。EssentialParticleJクラスの位置情報を保持したメンバ変数。
  
\item {\bf 機能}

  EssentialParticleJクラスの位置情報を保持したメンバ変数を返す。
  
\item {\bf 備考}

  EssentialParticleJクラスのメンバ変数posの変数名は変更可能。

\end{itemize}

\subsubsubsection{copyFromFP}

\begin{screen}
\begin{verbatim}
class FullParticle {
public:
    PS::S64    identity;
    PS::F64    mass;
    PS::F64vec position;
    PS::F64vec velocity;
    PS::F64vec acceleration;
    PS::F64    potential;
};
class EssentialParticleJ {
public:
    PS::S64    id;
    PS::F64    m;
    PS::F64vec pos;
    void copyFromFP(const FullParticle & fp) {
        this->id  = fp.identity;
        this->m   = fp.mass;
        this->pos = fp.position;
    }
};
\end{verbatim}
\end{screen}

\begin{itemize}

\item {\bf 前提}

  FullParticleクラスのメンバ変数identity, mass, positionと
  EssentialParticleJクラスのメンバ変数id, m, posはそれぞれ対応する情報
  を持つ。

\item {\bf 引数}

  fp: 入力。const FullParticle \&型。FullParticleクラスの情報を持つ。
  
\item {\bf 返値}

  なし。
  
\item {\bf 機能}

  FullParticleクラスの持つ1粒子の情報の一部をEssnetialParticleJクラス
  に書き込む。
  
\item {\bf 備考}

  Fullparticeクラスのメンバ変数の変数名、EssentialParticleJクラスのメ
  ンバ変数の変数名は変更可能。メンバ関数EssentialParticleJ::copyFromFP
  の引数名は変更可能。EssentialParticleJクラスの粒子情報はFullParticle
  クラスの粒子情報のサブセット。対応する情報を持つメンバ変数同士のデー
  タ型が一致している必要はないが、実数型とベクトル型(または整数型とベ
  クトル型)という違いがある場合に正しく動作する保証はない。

\end{itemize}

\subsubsection{場合によっては必要なメンバ関数}

\subsubsubsection{概要}

本節では、場合によっては必要なメンバ関数について記述する。相互作用ツリー
クラスのSEARCH\_MODE型にSEARCH\_MODE\_LONG以外を用いる場合に必要なメン
バ関数、列挙型のBOUNDARY\_CONDITION型にPS::BOUNDARY\_CONDITION\_OPEN以
外を選んだ場合に必要となるメンバ関数について記述する。

\subsubsubsection{相互作用ツリークラスのSEARCH\_MODE型にSEARCH\_MODE\_LONG以外を用いる場合}

\subsubsubsubsection{getRsearch}

\begin{screen}
\begin{verbatim}
class EssentialParticleJ {
public:
    PS::F64 search_radius;
    PS::F64 getRsearch() const {
        return this->search_radius;
    }
};
\end{verbatim}
\end{screen}

\begin{itemize}

\item {\bf 前提}

  EssentialParticleJクラスのメンバ変数search\_radiusはある1つの粒子の
  近傍粒子を探す半径の大きさ。このsearch\_radiusのデータ型はPS::F32型
  またはPS::F64型。
  
\item {\bf 引数}

  なし
  
\item {\bf 返値}

  PS::F32型またはPS::F64型。 EssentialParticleJクラスの近傍粒子を探す
  半径の大きさを保持したメンバ変数。
  
\item {\bf 機能}

  EssentialParticleJクラスの近傍粒子を探す半径の大きさを保持したメンバ
  変数を返す。

\item {\bf 備考}

  EssentialParticleJクラスのメンバ変数search\_radiusの変数名は変更可能。
  
\end{itemize}

\subsubsubsection{BOUNDARY\_CONDITION型にPS::BOUNDARY\_CONDITION\_OPEN以外を用いる場合}

\subsubsubsubsection{setPos}

\begin{screen}
\begin{verbatim}
class EssentialParticleJ {
public:
    PS::F64vec pos;
    void setPos(const PS::F64vec pos_new) {
        this->pos = pos_new;
    }
};
\end{verbatim}
\end{screen}

\begin{itemize}

\item {\bf 前提}
  
  EssentialParticleJクラスのメンバ変数posは1つの粒子の位置情報。この
  posのデータ型はPS::F32vecまたはPS::F64vec。EssentialParticleJクラス
  のメンバ変数posの元データとなっているのはFullParticleクラスのメンバ
  変数position。このデータ型はPS::F32vecまたはPS::F64vec。

\item {\bf 引数}

  pos\_new: 入力。const PS::F32vecまたはconst PS::F64vec型。FDPS側で修
  正した粒子の位置情報。

\item {\bf 返値}

  なし。
  
\item {\bf 機能}

  FDPSが修正した粒子の位置情報をEssentialParticleJクラスの位置情報に書き込む。

\item {\bf 備考}

  EssentialParticleJクラスのメンバ変数posの変数名は変更可能。メンバ関
  数EssentialParticleJ::setPosの引数名pos\_newは変更可能。posと
  pos\_newのデータ型が異なる場合の動作は保証しない。

\end{itemize}

\subsection{SuperParticleJクラス}
\label{sec:superparticlej}

\subsubsection{概要}

\subsubsection{既存のクラス}

\subsubsection{必要なメンバ関数}

\subsubsection{場合によっては必要なメンバ関数}

\subsubsubsection{getPos}

\subsubsubsection{setPos}

longcutoff

\subsubsubsection{copyFromMoment}

\subsubsubsection{convertToMoment}

\subsubsubsection{clear}


\subsection{Momentクラス}
\label{sec:moment}

\subsubsection{概要}

\subsubsection{既存のクラス}

\subsubsection{必要なメンバ関数}

\subsubsubsection{init}

\subsubsubsection{getPos}

\subsubsubsection{set}

\subsubsubsection{accumulateAtLeaf}

\subsubsubsection{accumulate}

\subsubsection{場合によっては必要なメンバ関数}

\subsubsubsection{accumulateAtLeaf2}

\subsubsubsection{accumulate2}


\subsection{Forceクラス}
\label{sec:force}

\subsubsection{概要}

\subsubsection{必要なメンバ関数}

\subsubsubsection{clear}


\subsection{ヘッダクラス}

\subsubsection{概要}

\subsubsection{必要なメンバ関数}

\subsubsubsection{概要}

\subsubsection{場合によっては必要なメンバ関数}

\subsubsubsection{概要}

\subsubsubsection{readParticleAscii}

\subsubsubsection{readParticleBinary}

\subsubsubsection{writeParticleAscii}

\subsubsubsection{writeParticleBinary}

\subsection{calcForceEpEpファンクタ}

\subsection{calcForceSpEpファンクタ}


\newpage

%%%%%%%%%%%%%%%%%%%%%%%%%%%%%%%%%%%%%%%%%%%%%%%%%%%%%
\section{プログラムの開始と終了}

\input{initfin.tex}

\newpage

%%%%%%%%%%%%%%%%%%%%%%%%%%%%%%%%%%%%%%%%%%%%%%%%%%%%%
\section{モジュール}

本節では、FDPSのモジュールについて記述する。最初にFDPSの標準機能につい
て、次にFDPSの拡張機能について記述する。

\subsection{標準機能}

\subsubsection{概要}

本節では、FDPSの標準機能について記述する。標準機能には4つのモジュール
があり、領域クラス、粒子群クラス、相互作用ツリークラス、通信用データク
ラスがある。この4つのクラスについて順に記述する。

\subsubsection{領域クラス}

本節では、領域クラスについて記述する。このクラスは領域情報の保持や領域
の分割を行うモジュールである。まずオブジェクトの生成方法を記述し、その
後APIを記述する。

%%%%%%%%%%%%%%%%%%%%%%%%%%%%%%%%%%%%%%%%%%%%%%%%%%%%%%%%%%%%%%%%%
\subsubsubsection{オブジェクトの生成}

領域クラスは以下のように宣言されている。
\begin{lstlisting}[caption=DomainInfo0]
namespace ParticleSimulator {
    class DomainInfo;
}
namespace PS = ParticleSimulator;
\end{lstlisting}

領域クラスのオブジェクトの生成は以下のように行う。ここではdinfoという
オブジェクトを生成している。
\begin{screen}
\begin{verbatim}
PS::DomainInfo dinfo;
\end{verbatim}
\end{screen}

%%%%%%%%%%%%%%%%%%%%%%%%%%%%%%%%%%%%%%%%%%%%%%%%%%%%%%%%%%%%%%%%%
\subsubsubsection{API}

領域クラスには初期設定関連のAPI、領域分割関連のAPIがある。以下、各節に
分けて記述する。

%%%%%%%%%%%%%%%%%%%%%%%%%%%%%%%%%%%%%%%%%%%%%%%%%%%%%%
\subsubsubsubsection{初期設定}

初期設定関連のAPIの宣言は以下のようになっている。このあと各APIについて
記述する。
\begin{lstlisting}[caption=DomainInfo1]
namespace ParticleSimulator {
    class DomainInfo{
    public
        DomainInfo();
        void initialize(const F32 coef_ema=1.0);
        void setNumberOfDomainMultiDimension(const S32 nx,
                                             const S32 ny,
                                             const S32 nz=1);
        void setBoundaryCondition(enum BOUNDARY_CONDITION bc);
        void setPosRootDomain(const F32vec & low,
                              const F32vec & high);
    };
}
namespace PS = ParticleSimulator;
\end{lstlisting}

%%%%%%%%%%%%%%%%%%%%%%%%%%%%%%%%%%%%%%%%%%%
\subsubsubsubsubsection{コンストラクタ}

\begin{screen}
\begin{verbatim}
void PS::DomainInfo::DomainInfo();
\end{verbatim}
\end{screen}

\begin{itemize}

\item {\bf 引数}

なし

\item {\bf 返値}

なし

\item {\bf 機能}

領域クラスのオブジェクトを生成する。

\end{itemize}

%%%%%%%%%%%%%%%%%%%%%%%%%%%%%%%%%%%%%%%%%%%
\subsubsubsubsubsection{initialize}

\begin{screen}
\begin{verbatim}
void PS::DomainInfo::initialize(const PS::F32 coef_ema=1.0);
\end{verbatim}
\end{screen}

\begin{itemize}

\item {\bf 引数}

coef\_ema: 入力。 const PS::F32型。指数移動平均の平滑化係数。デフォルト1.0

\item {\bf 返値}

なし

\item {\bf 機能}

領域クラスのオブジェクトを初期化する。

指数移動平均の平滑化係数を設定する。この係数の許される値は0から1である。
大きくなるほど、最新の粒子分布の情報が領域分割に反映されやすい。1の場
合、最新の粒子分布の情報のみ反映される。1度は呼ぶ必要があるが、2度呼
ぶと例外が送出される。

\end{itemize}

%%%%%%%%%%%%%%%%%%%%%%%%%%%%%%%%%%%%%%%%%%%
\subsubsubsubsubsection{setNumberOfDomainMultiDimension}

\begin{screen}
\begin{verbatim}
void PS::DomainInfo::setNumberOfDomainMultiDimension
                 (const PS::S32 nx,
                  const PS::S32 ny,
                  const PS::S32 nz=1);
\end{verbatim}
\end{screen}

\begin{itemize}

\item {\bf 引数}

nx: 入力。 const PS::S32型。x軸方向のルートドメインの分割数。

ny: 入力。 const PS::S32型。y軸方向のルートドメインの分割数。

nz: 入力。 const PS::S32型。z軸方向のルートドメインの分割数。デフォル
ト1。

\item {\bf 返値}

なし

\item {\bf 機能}

ルートドメインの分割する方法を設定する。nx, ny, nzはそれぞれx軸、y軸、
z軸方向のルートドメインの分割数である。呼ばなければ自動的にnx, ny, nz
が決まる。呼んだ場合に入力するnx, ny, nzの総積がMPIプロセス数と等しく
なければ、例外が送出される。

\end{itemize}

%%%%%%%%%%%%%%%%%%%%%%%%%%%%%%%%%%%%%%%%%%%
\subsubsubsubsubsection{setBoundaryCondition}

\begin{screen}
\begin{verbatim}
void PS::DomainInfo::setBoundaryCondition
                 (enum BOUNDARY_CONDITION bc);
\end{verbatim}
\end{screen}

\begin{itemize}

\item {\bf 引数}

bc: 入力。 列挙型。境界条件。

\item {\bf 返値}

なし

\item {\bf 機能}

境界条件の設定をする。許される入力は、\ref{sec:type_enum}で挙げた列挙
型のみ(ただしBOUNDARY\_CONDITION\_SHEARING\_BOX,
BOUNDARY\_CONDITION\_USER\_DEFINEDは未実装)。呼ばない場合は、開放境界
となる。

\end{itemize}

%%%%%%%%%%%%%%%%%%%%%%%%%%%%%%%%%%%%%%%%%%%
\subsubsubsubsubsection{setPosRootDomain}

\begin{screen}
\begin{verbatim}
void PS::DomainInfo::setPosRootDomain
                 (const PS::F32vec & low,
                  const PS::F32vec & high);
\end{verbatim}
\end{screen}

\begin{itemize}

\item {\bf 引数}

low: 入力。 PS::F32vec型。ルートドメインの下限(閉境界)。

high: 入力。 PS::F32vec型。ルートドメインの上限(解境界)。

\item {\bf 返値}

なし

\item {\bf 機能}

ルートドメインの下限と上限を設定する。開放境界条件の場合は呼ぶ必要はな
い。それ以外の境界条件の場合は、呼ばなくても動作するが、その結果が正し
いことは保証できない。

\end{itemize}

%%%%%%%%%%%%%%%%%%%%%%%%%%%%%%%%%%%%%%%%%%%%%%%%%%%%%%
\subsubsubsubsection{領域分割}

領域分割関連のAPIの宣言は以下のようになっている。このあと各APIについて
記述する。
\begin{lstlisting}[caption=DomainInfo2]
namespace ParticleSimulator {
    class DomainInfo{
    public:
        template<class Tpsys>
        void collectSampleParticle(Tpsys & psys,
                                   const F32 weight=1.0,
                                   const bool clear=true);
        void decomposeDomain();
        template<class Tpsys>
        void decomposeDomainAll(Tpsys & psys,
                                const F32 wgh=1.0);
    };
}
namespace PS = ParticleSimulator;
\end{lstlisting}

%%%%%%%%%%%%%%%%%%%%%%%%%%%%%%%%%%%%%%%%%%%
\subsubsubsubsubsection{collectSampleParticle}

\begin{screen}
\begin{verbatim}
template<class Tpsys>
void PS::DomainInfo::collectSampleParticle
                 (Tpsys & psys,
                  const PS::F32 weight=1.0,
                  const bool clear=true);
\end{verbatim}
\end{screen}

\begin{itemize}

\item {\bf 引数}

psys: 入力。 Tpsys型。領域分割のためのサンプル粒子を提供する粒子群クラ
ス。

weight: 入力。 const PS::F32型。領域分割のためのサンプル粒子数を決める
ためのウェイト。デフォルト1.0。

clear: 入力。 bool型。前にサンプルされた粒子情報をクリアするかどうかを
決定するフラグ。trueでクリアする。デフォルトtrue。

\item {\bf 返値}

なし

\item {\bf 機能}

粒子群クラスのオブジェクトpsysから粒子をサンプルする。weightによってそ
のMPIプロセスからサンプルする粒子の量を調整する(weightが大きいほどサン
プル粒子数が多い)。clearによってこれより前にサンプルした粒子の情報を消
すかどうか決める。

\end{itemize}

%%%%%%%%%%%%%%%%%%%%%%%%%%%%%%%%%%%%%%%%%%%
\subsubsubsubsubsection{decomposeDomain}

\begin{screen}
\begin{verbatim}
template<class Tpsys>
void PS::DomainInfo::decomposeDomain();
\end{verbatim}
\end{screen}

\begin{itemize}

\item {\bf 引数}

なし

\item {\bf 返値}

なし

\item {\bf 機能}

ルートドメインの分割を行う。

\end{itemize}

%%%%%%%%%%%%%%%%%%%%%%%%%%%%%%%%%%%%%%%%%%%
\subsubsubsubsubsection{decomposeDomainAll}

\begin{screen}
\begin{verbatim}
template<class Tpsys>
void PS::DomainInfo::decomposeDomainAll
                 (Tpsys & psys,
                  const PS::F32 weight=1.0);
\end{verbatim}
\end{screen}

\begin{itemize}

\item {\bf 引数}

psys: 入力。 Tpsys型。領域分割のためのサンプル粒子を提供する粒子群クラ
ス。

weight: 入力。 const PS::F32型。領域分割のためのサンプル粒子数を決める
ためのウェイト。デフォルト1.0。

\item {\bf 返値}

なし

\item {\bf 機能}

粒子群クラスのオブジェクトpsysから粒子をサンプルし、続けてルートドメイ
ンの分割を行う。PS::DomainInfo::collectSampleParticleと
PS::DomainInfo::decomposeDomainで行われていることが一度に行われる。
weightの意味はPS::DomainInfo::collectSampleParticleと同じ。

\end{itemize}

%%\subsubsubsubsubsubsection{getNumberofDomainOneAxis}

%%\subsubsubsubsubsubsection{getDomain}


\subsubsection{粒子群クラス}

本節では、粒子群クラスについて記述する。このクラスは粒子情報の保持や
MPIプロセス間で粒子情報の交換を行うモジュールである。まずオブジェクト
の生成方法を記述し、その後APIを記述する。

%%%%%%%%%%%%%%%%%%%%%%%%%%%%%%%%%%%%%%%%%%%%%%%%%%%%%%%%%%%%%%%%%
\subsubsubsection{オブジェクトの生成}

粒子群クラスは以下のように宣言されている。
\begin{lstlisting}[caption=ParticleSystem0]
namespace ParticleSimulator {
    template<class Tptcl>
    class ParticleSystem;
}
namespace PS = ParticleSimulator;
\end{lstlisting}
テンプレート引数Tptclはユーザー定義のFullParticle型である。

粒子群クラスのオブジェクトの生成は以下のように行う。ここではsystemとい
うオブジェクトを生成している。
\begin{screen}
\begin{verbatim}
PS::ParticleSystem<Tptcl> system;
\end{verbatim}
\end{screen}
テンプレート引数Tptclはユーザー定義のFullParticle型である。

%%%%%%%%%%%%%%%%%%%%%%%%%%%%%%%%%%%%%%%%%%%%%%%%%%%%%%%%%%%%%%%%%
\subsubsubsection{API}

このモジュールには初期設定関連のAPI、オブジェクト情報取得設定関連のAPI、
ファイル入出力関連のAPI、粒子交換関連のAPIがある。以下、各節に分けて記
述する。

%%%%%%%%%%%%%%%%%%%%%%%%%%%%%%%%%%%%%%%%%%%%%%%%%%%%%%
\subsubsubsubsection{初期設定}

初期設定関連のAPIの宣言は以下のようになっている。このあと各APIについて
記述する。
\begin{lstlisting}[caption=ParticleSystem1]
namespace ParticleSimulator {
    template<class Tptcl>
    class ParticleSystem{
    public:
        ParticleSystem();
        void initialize();
        void setNumberOfDomainMultiDimension(const S32 nx,
                                             const S32 ny,
                                             const S32 nz=1);
        void setAverageTargetNumberOfSampleParticlePerProcess
                        (const S32 & nsampleperprocess);
        void createParticle(const S32 n_limit);
    };
}
namespace PS = ParticleSimulator;
\end{lstlisting}


%%%%%%%%%%%%%%%%%%%%%%%%%%%%%%%%%%%%%%%%%%%
\subsubsubsubsubsection{コンストラクタ}

\begin{screen}
\begin{verbatim}
void PS::ParticleSystem::ParticleSystem();
\end{verbatim}
\end{screen}

\begin{itemize}

\item {\bf 引数}

なし

\item {\bf 返値}

なし

\item {\bf 機能}

粒子群クラスのオブジェクトを生成する。

\end{itemize}

%%%%%%%%%%%%%%%%%%%%%%%%%%%%%%%%%%%%%%%%%%%
\subsubsubsubsubsection{initialize}

\begin{screen}
\begin{verbatim}
void PS::ParticleSystem::initialize();
\end{verbatim}
\end{screen}

\begin{itemize}

\item {\bf 引数}

なし

\item {\bf 返値}

なし

\item {\bf 機能}

粒子群クラスのオブジェクトを初期化する。1度は呼ぶ必要があるが、2度呼
ぶと例外が送出される。

\end{itemize}

%%%%%%%%%%%%%%%%%%%%%%%%%%%%%%%%%%%%%%%%%%%
\subsubsubsubsubsection{createParticle} %% 必要?

\begin{screen}
\begin{verbatim}
void PS::ParticleSystem::createParticle
                 (const PS::S32 n_limit);
\end{verbatim}
\end{screen}

\begin{itemize}

\item {\bf 引数}

n\_limit: 入力。 const PS::S32型。粒子配列の上限。

%%clear: 入力。 bool型。   デフォルトtrue。

\item {\bf 返値}

なし

\item {\bf 機能}

粒子配列のメモリを確保する。n\_limitには1つのMPIプロセスで扱う粒子数
の上限数を入力する。

\end{itemize}

%%%%%%%%%%%%%%%%%%%%%%%%%%%%%%%%%%%%%%%%%%%
\subsubsubsubsubsection{setAverateTargetNumberOfSampleParticlePerProcess}

\begin{screen}
\begin{verbatim}
void PS::ParticleSystem::setAverateTargetNumberOfSampleParticlePerProcess
                 (const PS::S32 & nsampleperprocess);
\end{verbatim}
\end{screen}

\begin{itemize}

\item {\bf 引数}

nsampleperprocess: 入力。const PS::S32 \&型。1つのMPIプロセスでサンプル
する粒子数目標。

\item {\bf 返値}

なし

\item {\bf 機能}

1つのMPIプロセスでサンプルする粒子数の目標を設定する。呼び出さなくて
もよいが、呼び出さないとこの目標数が30となる。

\end{itemize}

%%%%%%%%%%%%%%%%%%%%%%%%%%%%%%%%%%%%%%%%%%%%%%%%%%%%%%
\subsubsubsubsection{オブジェクト情報の取得設定}

オブジェクト情報取得関連のAPIの宣言は以下のようになっている。このあと
各APIについて記述する。
\begin{lstlisting}[caption=ParticleSystem2]
namespace ParticleSimulator {
    template<class Tptcl>
    class ParticleSystem{
    public
        Tptcl & operator [] (const S32 id);
        void setNumberOfParticleLocal(const S32 n);
        const S32 getNumberOfParticleLocal();
        S32 getNumberOfParticleGlobal();
    };
}
namespace PS = ParticleSimulator;
\end{lstlisting}


%%%%%%%%%%%%%%%%%%%%%%%%%%%%%%%%%%%%%%%%%%%
\subsubsubsubsubsection{operator []}

\begin{screen}
\begin{verbatim}
Tptcl & PS::ParticleSystem::operator []
             (const S32 id);
\end{verbatim}
\end{screen}

\begin{itemize}

\item {\bf 引数}

n: 入力。const PS::S32型。粒子配列のインデックス。

\item {\bf 返値}

Tptcl型。Tptcl型のオブジェクト1つ。

\item {\bf 機能}

Tptcl型のオブジェクト1つ返す。

\end{itemize}

%%%%%%%%%%%%%%%%%%%%%%%%%%%%%%%%%%%%%%%%%%%
\subsubsubsubsubsection{setNumberOfParticleLocal}

\begin{screen}
\begin{verbatim}
void PS::ParticleSystem::setNumberOfParticleLocal
                 (const PS::S32 n);
\end{verbatim}
\end{screen}

\begin{itemize}

\item {\bf 引数}

n: 入力。const PS::S32型。粒子数。

\item {\bf 返値}

なし

\item {\bf 機能}

1つのMPIプロセスの持つ粒子数を設定する。

\end{itemize}

%%%%%%%%%%%%%%%%%%%%%%%%%%%%%%%%%%%%%%%%%%%
\subsubsubsubsubsection{getNumberOfParticleLocal}

\begin{screen}
\begin{verbatim}
const PS::S32 PS::ParticleSystem::getNumberOfParticleLocal();
\end{verbatim}
\end{screen}

\begin{itemize}

\item {\bf 引数}

なし

\item {\bf 返値}

const PS::S32型。1つのMPIプロセスの持つ粒子数。

\item {\bf 機能}

1つのMPIプロセスの持つ粒子数を返す。

\end{itemize}

%%%%%%%%%%%%%%%%%%%%%%%%%%%%%%%%%%%%%%%%%%%
\subsubsubsubsubsection{getNumberOfParticleGlobal}

\begin{screen}
\begin{verbatim}
const PS::S32 PS::ParticleSystem::getNumberOfParticleGlobal();
\end{verbatim}
\end{screen}

\begin{itemize}

\item {\bf 引数}

なし

\item {\bf 返値}

const PS::S32型。全MPIプロセスの持つ粒子数。

\item {\bf 機能}

全MPIプロセスの持つ粒子数を返す。

\end{itemize}

%%%%%%%%%%%%%%%%%%%%%%%%%%%%%%%%%%%%%%%%%%%%%%%%%%%%%%
\subsubsubsubsection{ファイル入出力}

ファイル入出力関連のAPIの宣言は以下のようになっている。このあと各APIに
ついて記述する。
\begin{lstlisting}[caption=ParticleSystem3]
namespace ParticleSimulator {
    template<class Tptcl>
    class ParticleSystem{
    public
        template <class Theader>
        void readParticleAscii(const char * const filename,
                               const char * const format,
                               Theader & header);
        void readParticleAscii(const char * const filename,       
                               const char * const format);
        template <class Theader>
        void readParticleAscii(const char * const filename,
                               Theader & header);
        void readParticleAscii(const char * const filename);        
        template <class Theader>
        void writeParticleAscii(const char * const filename,
                                const char * const format,
                                const Theader & header);        
        template <class Theader>
        void writeParticleAscii(const char * const filename,
                                const Theader & header);
        void writeParticleAscii(const char * const filename,
                                const char * format);                                       
        void writeParticleAscii(const char * const filename);
    };
}
namespace PS = ParticleSimulator;
\end{lstlisting}


%%%%%%%%%%%%%%%%%%%%%%%%%%%%%%%%%%%%%%%%%%%
\subsubsubsubsubsection{readParticleAscii}
\label{sec:readParticleAscii}

\begin{screen}
\begin{verbatim}
template <class Theader>
void PS::ParticleSystem::readParticleAscii
                 (const char * const filename,
                  const char * const format,
                  Theader & header);
\end{verbatim}
\end{screen}

\begin{itemize}

\item {\bf 引数}

filename: 入力。const char * const型。

format: 入力。const char * const型。

header: 入力。Theader \&型。

\item {\bf 返値}

なし

\item {\bf 機能}

\redtext{建設中}

\end{itemize}

\begin{screen}
\begin{verbatim}
template <class Theader>
void PS::ParticleSystem::readParticleAscii
                 (const char * const filename,
                  Theader & header);
\end{verbatim}
\end{screen}

\begin{itemize}

\item {\bf 引数}

filename: 入力。const char * const型。

header: 入力。Theader \&型。

\item {\bf 返値}

なし

\item {\bf 機能}

\redtext{建設中}

\end{itemize}

\begin{screen}
\begin{verbatim}
void PS::ParticleSystem::readParticleAscii
                 (const char * const filename,
                  const char * const format);
\end{verbatim}
\end{screen}

\begin{itemize}

\item {\bf 引数}

filename: 入力。const char * const型。

format: 入力。const char * const型。

\item {\bf 返値}

なし

\item {\bf 機能}

\redtext{建設中}

\end{itemize}

\begin{screen}
\begin{verbatim}
void PS::ParticleSystem::readParticleAscii
                 (const char * const filename);
\end{verbatim}
\end{screen}

\begin{itemize}

\item {\bf 引数}

filename: 入力。const char * const型。

\item {\bf 返値}

なし

\item {\bf 機能}

\redtext{建設中}

\end{itemize}

%%%%%%%%%%%%%%%%%%%%%%%%%%%%%%%%%%%%%%%%%%%
\subsubsubsubsubsection{readParticleBinary}
\label{sec:readParticleBinary}

%%%%%%%%%%%%%%%%%%%%%%%%%%%%%%%%%%%%%%%%%%%
\subsubsubsubsubsection{writeParticlAscii}
\label{sec:writeParticleAscii}

\begin{screen}
\begin{verbatim}
template <class Theader>
void PS::ParticleSystem::writeParticleAscii
                 (const char * const filename,
                  const char * const format,
                  const Theader & header);
\end{verbatim}
\end{screen}

\begin{itemize}

\item {\bf 引数}

filename: 入力。const char * const型。

format: 入力。const char * const型。

header: 入力。const Theader \&型。

\item {\bf 返値}

なし

\item {\bf 機能}

\redtext{建設中}

\end{itemize}

\begin{screen}
\begin{verbatim}
template <class Theader>
void PS::ParticleSystem::writeParticleAscii
                 (const char * const filename,
                  const Theader & header);
\end{verbatim}
\end{screen}

\begin{itemize}

\item {\bf 引数}

filename: 入力。const char * const型。

header: 入力。const Theader \&型。

\item {\bf 返値}

なし

\item {\bf 機能}

\redtext{建設中}

\end{itemize}

\begin{screen}
\begin{verbatim}
void PS::ParticleSystem::writeParticleAscii
                 (const char * const filename,
                  const char * format);
\end{verbatim}
\end{screen}

\begin{itemize}

\item {\bf 引数}

filename: 入力。const char * const型。

format: 入力。const char * 型。

\item {\bf 返値}

なし

\item {\bf 機能}

\redtext{建設中}

\end{itemize}

\begin{screen}
\begin{verbatim}
void PS::ParticleSystem::writeParticleAscii
                 (const char * const filename);
\end{verbatim}
\end{screen}

\begin{itemize}

\item {\bf 引数}

filename: 入力。const char * const型。

\item {\bf 返値}

なし

\item {\bf 機能}

\redtext{建設中}

\end{itemize}

%%%%%%%%%%%%%%%%%%%%%%%%%%%%%%%%%%%%%%%%%%%
\subsubsubsubsubsection{writeParticleBinary}
\label{sec:writeParticleBinary}

%%%%%%%%%%%%%%%%%%%%%%%%%%%%%%%%%%%%%%%%%%%%%%%%%%%%%%
\subsubsubsubsection{粒子交換}

粒子交換関連のAPIの宣言は以下のようになっている。このあと各APIについて
記述する。
\begin{lstlisting}[caption=ParticleSystem4]
namespace ParticleSimulator {
    template<class Tptcl>
    class ParticleSystem{
    public
        template<class Tdinfo>
        void exchangeParticle(Tdinfo & dinfo);
    };
}
namespace PS = ParticleSimulator;
\end{lstlisting}


%%%%%%%%%%%%%%%%%%%%%%%%%%%%%%%%%%%%%%%%%%%
\subsubsubsubsubsection{exchangeParticle}

\begin{screen}
\begin{verbatim}
template<class Tdinfo>
void PS::ParticleSystem::exchangeParticle
                 (Tdinfo & dinfo);
\end{verbatim}
\end{screen}

\begin{itemize}

\item {\bf 引数}

dinfo: 入力。Tdinfo型。領域クラスのオブジェクト。

\item {\bf 返値}

なし

\item {\bf 機能}

粒子が適切なドメインに配置されるように、粒子の交換を行う。どのドメイン
にも属さない粒子が現れた場合、例外が送出される。

\end{itemize}


\subsubsection{相互作用ツリークラス}

本節では、相互作用ツリークラスについて記述する。このクラスは粒子間相互
作用の計算を行うモジュールである。まずオブジェクトの生成方法を記述し、
その後APIを記述する。

%%%%%%%%%%%%%%%%%%%%%%%%%%%%%%%%%%%%%%%%%%%%%%%%%%%%%%%%%%%%%%%%%
\subsubsubsection{オブジェクトの生成}

このクラスは以下のように宣言されている。
\begin{lstlisting}[caption=TreeForForce0]
namespace ParticleSimulator {
    template<class TSM,
             class Tforce,
             class Tepi,
             class Tepj,
             class Tmomloc,
             class Tmomglb,
             class Tspj>
    class TreeForForce;
}
namespace PS = ParticleSimulator;
\end{lstlisting}

テンプレート引数は順に、SEARCH\_MODE型、Force型(ユーザー定義)、
EssentialParticleI型(ユーザー定義)、EssentialParticleJ型(ユーザー定義)、
ローカルツリーのMoment型(ユーザー定義)、グローバルツリーのMoment型(ユー
ザー定義)、SuperParticleJ型(ユーザー定義)である。

SEARCH\_MODE型に応じてラッパーを用意した。これらのラッパーを使えば入力
するテンプレート引数の数が減るので、こちらのラッパーを用いることを推奨
する。以下、SEARCH\_MODE型がSEARCH\_MODE\_LONG,
SEARCH\_MODE\_LONG\_CUTOFF, SEARCH\_MODE\_GATHER,
SEARCH\_MODE\_SCATTER, SEARCH\_MODE\_SYMMETRYの場合のオブジェクトの生
成方法を記述する。

\subsubsubsubsection{SEARCH\_MODE\_LONG}

以下のようにオブジェクトsystemを生成する。
\begin{screen}
\begin{verbatim}
PS::TreeForForceLong<Tforce, Tepi, Tepj, Tmom, Tspj>::Normal system;
\end{verbatim}
\end{screen}
テンプレート引数は順に、Force型(ユーザー定義)、EssentialParticleI型(ユー
ザー定義)、EssentialParticleJ型(ユーザー定義)、ローカルツリー及びグロー
バルツリーのMoment型(ユーザー定義)、SuperParticleJ型(ユーザー定義)であ
る。

あらかじめMoment型とSuperParticleJ型を指定した型も用意した。これらはモー
メントの計算方法別に6種類ある。モーメント計算の中心を粒子の重心または
粒子の幾何中心とした場合に、単極子まで、四重極子まで計算するものである。
以下、粒子の重心を中心とした場合の単極子まで、双極子まで、四重極子まで
のモーメント計算、粒子の幾何中心を中心とした場合の単極子まで、双極子ま
で、四重極子までのモーメント計算、のオブジェクト方法をこの順で記述する。
すべてsystemというオブジェクトを生成している。

\begin{screen}
\begin{verbatim}
PS::TreeForForceLong<Tforce, Tepi, Tepj>::Monopole system;
\end{verbatim}
\end{screen}

%%\begin{screen}
%%\begin{verbatim}
%%PS::TreeForForceLong<Tforce, Tepi, Tepj>::Dipole system;
%%\end{verbatim}
%%\end{screen}

\begin{screen}
\begin{verbatim}
PS::TreeForForceLong<Tforce, Tepi, Tepj>::Quadrupole system;
\end{verbatim}
\end{screen}

\begin{screen}
\begin{verbatim}
PS::TreeForForceLong<Tforce, Tepi, Tepj>::MonopoleGeometricCenter system;
\end{verbatim}
\end{screen}
\begin{screen}
\begin{verbatim}
PS::TreeForForceLong<Tforce, Tepi, Tepj>::DipoleGeometricCenter system;
\end{verbatim}
\end{screen}

\begin{screen}
\begin{verbatim}
PS::TreeForForceLong<Tforce, Tepi, Tepj>::QuadrupoleGeometricCenter system;
\end{verbatim}
\end{screen}
すべての型のテンプレート引数は順に、Force型、EssentialParticleI型、
EssentialParticleJ型である。

\subsubsubsubsection{SEARCH\_MODE\_LONG\_CUTOFF}


以下のようにオブジェクトsystemを生成する。
\begin{screen}
\begin{verbatim}
PS::TreeForForceLong<Tforce, Tepi, Tepj, Tmom, Tspj>::WithCutoff system;
\end{verbatim}
\end{screen}
テンプレート引数は順に、Force型、EssentialParticleI型、
EssentialParticleJ型、ローカルツリー及びグローバルツリーのMoment型、
SuperParticleJ型である。

あらかじめMoment型とSuperParticleJ型を指定した型も用意した。モーメント
計算の中心を粒子の重心とした場合に、単極子まで計算するものである。ここ
ではsystemというオブジェクトを生成している。

\begin{screen}
\begin{verbatim}
PS::TreeForForceLong<Tforce, Tepi, Tepj>::MonopoleWithCutoff system;
\end{verbatim}
\end{screen}
テンプレート引数は順に、Force型、EssentialParticleI型、
EssentialParticleJ型である。

\subsubsubsubsection{SEARCH\_MODE\_GATHER}

以下のようにオブジェクトsystemを生成する。
\begin{screen}
\begin{verbatim}
PS::TreeForForceShort<Tforce, Tepi, Tepj>::Gather system;
\end{verbatim}
\end{screen}
テンプレート引数は順に、Force型、EssentialParticleI型、
EssentialParticleJ型である。

\subsubsubsubsection{SEARCH\_MODE\_SCATTER}

以下のようにオブジェクトsystemを生成する。
\begin{screen}
\begin{verbatim}
PS::TreeForForceShort<Tforce, Tepi, Tepj>::Scatter system;
\end{verbatim}
\end{screen}
テンプレート引数は順に、Force型、EssentialParticleI型、
EssentialParticleJ型である。

\subsubsubsubsection{SEARCH\_MODE\_SYMMETRY}

以下のようにオブジェクトsystemを生成する。
\begin{screen}
\begin{verbatim}
PS::TreeForForceShort<Tforce, Tepi, Tepj>::Symmetry system;
\end{verbatim}
\end{screen}
テンプレート引数は順に、Force型、EssentialParticleI型、
EssentialParticleJ型である。

%%%%%%%%%%%%%%%%%%%%%%%%%%%%%%%%%%%%%%%%%%%%%%%%%%%%%%%%%%%%%%%%%
\subsubsubsection{API}

このモジュールには初期設定関連のAPI、相互作用計算関連の低レベルAPI、相
互作用計算関連の高レベルAPI、ネイバーリスト関連のAPIがある。以下、各節
に分けて記述する。

%%%%%%%%%%%%%%%%%%%%%%%%%%%%%%%%%%%%%%%%%%%%%%%%%%%%%%
\subsubsubsubsection{初期設定}

初期設定関連のAPIの宣言は以下のようになっている。このあと各APIについて
記述する。
\begin{lstlisting}[caption=TreeForForce1]
namespace ParticleSimulator {
    template<class TSM,
             class Tforce,
             class Tepi,
             class Tepj,
             class Tmomloc,
             class Tmomglb,
             class Tspj>
    class TreeForForce{
    public:
    void TreeForForce();
    void initialize(const U64 n_glb_tot,
                    const F32 theta=0.7,
                    const U32 n_leaf_limit=8,
                    const U32 n_group_limit=64);
    };
}
namespace PS = ParticleSimulator;
\end{lstlisting}

\subsubsubsubsubsection{コンストラクタ}

\begin{screen}
\begin{verbatim}
void PS::TreeForForce::TreeForForce();
\end{verbatim}
\end{screen}

\begin{itemize}

\item {\bf 引数}

なし

\item {\bf 返値}

なし

\item {\bf 機能}

相互作用ツリークラスのオブジェクトを生成する。

\end{itemize}

\subsubsubsubsubsection{initialize}

\begin{screen}
\begin{verbatim}
void PS::TreeForForce::initialize
            (const PS::U64 n_glb_tot,
             const PS::F32 theta=0.7,
             const PS::U32 n_leaf_limit=8,
             const PS::U32 n_group_limit=64);
\end{verbatim}
\end{screen}

\begin{itemize}

\item {\bf 引数}

n\_glb\_tot: 入力。const PS::U64型。粒子配列の上限。

theta: 入力。const PS::F32型。見こみ角に対する基準。デフォルト0.7。

n\_leaf\_limit。const PS::U32型。ツリーを切るのをやめる粒子数の上限。
デフォルト8。

n\_group\_limit。const PS::U32型。相互作用リストを共有する粒子数の上限。
デフォルト64。

\item {\bf 返値}

なし

\item {\bf 機能}

相互作用ツリークラスのオブジェクトを初期化する。

\end{itemize}

%%%%%%%%%%%%%%%%%%%%%%%%%%%%%%%%%%%%%%%%%%%%%%%%%%%%%%
\subsubsubsubsection{低レベル関数}

相互作用計算関連の低レベルAPIの宣言は以下のようになっている。このあと
各APIについて記述する。
\begin{lstlisting}[caption=TreeForForce1]
namespace ParticleSimulator {
    template<class TSM,
             class Tforce,
             class Tepi,
             class Tepj,
             class Tmomloc,
             class Tmomglb,
             class Tspj>
    class TreeForForce{
    public:
        template<class Tpsys>
        void setParticleLocalTree(const Tpsys & psys,
                                  const bool clear=true);
        template<class Tdinfo>
        void makeLocalTree(const Tdinfo & difno);
        void makeLocalTree(const F32 l,
                           const F32vec & c = F32vec(0.0));
        template<class Tdinfo>
        void makeGlobalTree(const Tdinfo & dinfo);        
        void calcMomentGlobalTree();
        template<class Tfunc_ep_ep>
        void calcForce(Tfunc_ep_ep pfunc_ep_ep,
                       const bool clear=true);
        template<class Tfunc_ep_ep, class Tfunc_ep_sp>
        void calcForce(Tfunc_ep_ep pfunc_ep_ep,
                       Tfunc_ep_sp pfunc_ep_sp,
                       const bool clear=true);
        Tforce getForce(const S32 i);
    };
}
namespace PS = ParticleSimulator;
\end{lstlisting}

\subsubsubsubsubsection{setParticleLocalTree}

\begin{screen}
\begin{verbatim}
template<class Tpsys>
void PS::TreeForForce::setParticleLocalTree
            (const Tpsys & psys,
             const bool clear = true);
\end{verbatim}
\end{screen}

\begin{itemize}

\item {\bf 引数}

psys: 入力。const Tpsys \&型。ローカルツリーを構成する粒子群。

clear: 入力。const bool型。前に読込んだ粒子をクリアするかどうか決定す
るフラグ。trueでクリアする。デフォルトtrue。

\item {\bf 返値}

なし

\item {\bf 機能}

相互作用ツリークラスのオブジェクトに粒子群クラスのオブジェクトの粒子を
読み込む。clearがtrueならば前に読込んだ粒子情報をクリアし、falseならク
リアしない。

\end{itemize}

\subsubsubsubsubsection{makeLocalTree}
%% setRootCell, mortonSortLocalTreeOnly, linkCellLocalTreeOnly
%% setRootCell関係でオーバーロード

\begin{screen}
\begin{verbatim}
template<class Tdinfo>
void PS::TreeForForce::makeLocalTree
            (const Tdinfo & dinfo);
\end{verbatim}
\end{screen}

\begin{itemize}

\item {\bf 引数}

dinfo: 入力。const Tdinfo \&型。領域クラスのオブジェクト。

\item {\bf 返値}

なし

\item {\bf 機能}

ローカルツリーを作る。領域クラスのオブジェクトから扱うべきルートドメイ
ンを読み取り、ツリーのルートセルを決定する。

\end{itemize}

\begin{screen}
\begin{verbatim}
template<class Tdinfo>
void PS::TreeForForce::makeLocalTree
            (const PS::F32 l,
             const PS::F32vec & c = PS::F32vec(0.0));
\end{verbatim}
\end{screen}

\begin{itemize}

\item {\bf 引数}

l: 入力。const PS::F32型。ツリーのルートセルの大きさ。

c: 入力。const PS::F32vec \&型。ツリーの中心の座標。デフォルトは座標原点。

\item {\bf 返値}

なし

\item {\bf 機能}

ローカルツリーを作る。ツリーのルートセルを2つの引数で決定する。ツリー
のルートセルは全プロセスで共通でなければならない。共通でない場合の動作
の正しさは保証しない。

\end{itemize}

\subsubsubsubsubsection{makeGlobalTree}
%% calcMomentLocalTreeOnly, exchangeLocalEssentialTree,
%% setLocalEssentialTreeToGlobalTree, mortonSortGlobalTreeOnly,
%% linkCellGlobalTreeOnly

\begin{screen}
\begin{verbatim}
template<class Tdinfo>
void PS::TreeForForce::makeGlobalTree
            (const Tdinfo & dinfo);        
\end{verbatim}
\end{screen}

\begin{itemize}

\item {\bf 引数}

dinfo: 入力。const Tdinfo \& 型。領域クラスのオブジェクト。

\item {\bf 返値}

なし

\item {\bf 機能}

グローバルツリーを作る。

\end{itemize}

\subsubsubsubsubsection{calcMomentGlobalTree(仮)}
%% calcMomentGlobalTreeOnly, makeIPGroup

\begin{screen}
\begin{verbatim}
template<class Tdinfo>
void PS::TreeForForce::calcMomentGlobalTree();
\end{verbatim}
\end{screen}

\begin{itemize}

\item {\bf 引数}

なし

\item {\bf 返値}

なし

\item {\bf 機能}

グローバルツリーの各々のセルのモーメントを計算する。

\end{itemize}

\subsubsubsubsubsection{calcForce}

\begin{screen}
\begin{verbatim}
template<class Tfunc_ep_ep>
void PS::TreeForForce::calcForce
             (Tfunc_ep_ep pfunc_ep_ep,
              const bool clear=true);
\end{verbatim}
\end{screen}

\begin{itemize}

\item {\bf 引数}

pfunc\_ep\_ep: 入力。返値がvoid型のEssentialParticleIと
EssentialParticleJの間の相互作用計算用関数ポインタ、または関数オブジェ
クト。関数の引数は第1引数から順にconst EssentialParticleI *型、
PS::S32型、const EssentialParticleJ *型、PS::S32型、Force *型。

clear: 入力。const bool型。前に計算された相互作用の結果をクリアするか
どうかを決定するフラグ。trueならばクリアする。デフォルトtrue。

\item {\bf 返値}

なし

\item {\bf 機能}

このオブジェクトに読み込まれた粒子すべての粒子間相互作用を計算する。粒
子間相互作用は短距離力の場合に限る。

\end{itemize}

\begin{screen}
\begin{verbatim}
template<class Tfunc_ep_ep, class Tfunc_ep_sp>
void PS::TreeForForce::calcForce
             (Tfunc_ep_ep pfunc_ep_ep,
              Tfunc_ep_sp pfunc_ep_sp,
              const bool clear=true);
\end{verbatim}
\end{screen}

\begin{itemize}

\item {\bf 引数}

pfunc\_ep\_ep: 入力。返値がvoid型のEssentialParticleIと
EssentialParticleJの間の相互作用計算用関数ポインタ、または関数オブジェ
クト。関数の引数は第1引数から順にconst EssentialParticleI *型、
PS::S32型、const EssentialParticleJ *型、PS::S32型、Force *型。

pfunc\_ep\_sp: 入力。返値がvoid型のEssentialParticleIとSuperParticleJ
の間の相互作用計算用関数ポインタ、または関数オブジェクト。関数の引数は
第1引数から順にconst EssentialParticleI *型、PS::S32型、const
SuperParticleJ *型、PS::S32型、Force *型。

clear: 入力。const bool型。前に計算された相互作用の結果をクリアするか
どうかを決定するフラグ。trueならばクリアする。デフォルトtrue。

\item {\bf 返値}

なし

\item {\bf 機能}

このオブジェクトに読み込まれた粒子すべての粒子間相互作用を計算する。粒
子間相互作用は長距離力の場合に限る。

\end{itemize}

\subsubsubsubsubsection{getForce}

\begin{screen}
\begin{verbatim}
Tforce PS::TreeForForce::getForce(const PS::S32 i);
\end{verbatim}
\end{screen}

\begin{itemize}

\item {\bf 引数}

i: 入力。const PS::S32型。粒子配列のインデックス。

\item {\bf 返値}

Tforce型。setParticleLocalTreeでi番目に読み込まれた粒子の受ける作用。

\item {\bf 機能}

setParticleLocalTreeでi番目に読み込まれた粒子の受ける作用を返す。

\end{itemize}

\subsubsubsubsubsection{copyLocalTreeStructure}

今後、追加する。

\subsubsubsubsubsection{repeatLocalCalcForce}

今後、追加する。

%%%%%%%%%%%%%%%%%%%%%%%%%%%%%%%%%%%%%%%%%%%%%%%%%%%%%%
\subsubsubsubsection{高レベル関数}

相互作用計算関連の高レベルAPIの宣言は以下のようになっている。このあと
各APIについて記述する。
\begin{lstlisting}[caption=TreeForForce1]
namespace ParticleSimulator {
    template<class TSM,
             class Tforce,
             class Tepi,
             class Tepj,
             class Tmomloc,
             class Tmomglb,
             class Tspj>
    class TreeForForce{
    public:
        template<class Tfunc_ep_ep,
                 class Tpsys,
                 class Tdinfo>
        void calcForceAllAndWriteBack(Tfunc_ep_ep pfunc_ep_ep,
                                      Tpsys & psys,
                                      Tdinfo & dinfo,
                                      const bool clear_force = true);
        template<class Tfunc_ep_ep,
                 class Tfunc_ep_sp,
                 class Tpsys,
                 class Tdinfo>
        void calcForceAllAndWriteBack(Tfunc_ep_ep pfunc_ep_ep, 
                                      Tfunc_ep_sp pfunc_ep_sp,  
                                      Tpsys & psys,
                                      TDinfo & dinfo,
                                      const bool clear_force=true);
                                      
        template<class Tfunc_ep_ep,
                 class Tfunc_ep_sp,
                 class Tpsys,
                 class Tdinfo>
        void calcForceAll(Tfunc_ep_ep pfunc_ep_ep,
                          Tfunc_ep_sp pfunc_ep_sp,
                          Tpsys & psys,
                          Tdinfo & dinfo,
                          const bool clear_force=true);
        template<class Tfunc_ep_ep,
                 class Tfunc_ep_sp,
                 class Tpsys,
                 class Tdinfo>
        void calcForceAll(Tfunc_ep_ep pfunc_ep_ep,
                          Tfunc_ep_sp pfunc_ep_sp,
                          Tpsys & psys,
                          Tdinfo & dinfo,
                          const bool clear_force=true);

        template<class Tfunc_ep_ep,
                 class Tdinfo>
        void calcForceMakeingTree(Tfunc_ep_ep pfunc_ep_ep,
                                  Tdinfo & dinfo,
                                  const bool clear_force=true);
        template<class Tfunc_ep_ep,
                 class Tfunc_ep_sp,
                 class Tdinfo>
        void calcForceMakingTree(Tfunc_ep_ep pfunc_ep_ep,
                                Tfunc_ep_sp pfunc_ep_sp,
                                Tdinfo & dinfo,
                                const bool clear_force=true);

        template<class Tfunc_ep_ep,
                 class Tpsys>
        void calcForceAndWriteBack(Tfunc_ep_ep pfunc_ep_ep,
                                   Tpsys & psys,
                                   const bool clear=true);
        template<class Tfunc_ep_ep,
                 class Tfunc_ep_sp,
                 class Tpsys>
        void calcForceAndWriteBack(Tfunc_ep_ep pfunc_ep_ep,
                                   Tfunc_ep_sp pfunc_ep_sp,
                                   Tpsys & psys,
                                   const bool clear=true);
    };
}
namespace PS = ParticleSimulator;
\end{lstlisting}

\subsubsubsubsubsection{calcForceAllAndWriteBack}
%% setParticleLocalTree, makeLocalTree, makeGlobalTree,
%% calcMomentGlobalTree, calcForce, getForce

\begin{screen}
\begin{verbatim}
template<class Tfunc_ep_ep,
         class Tpsys,
         class Tdinfo>
void PS::TreeForForce::calcForceAllandWriteBack
             (Tfunc_ep_ep pfunc_ep_ep,
              Tpsys & psys,
              Tdinfo & dinfo
              const bool clear=true);
\end{verbatim}
\end{screen}

\begin{itemize}

\item {\bf 引数}

pfunc\_ep\_ep: 入力。返値がvoid型のEssentialParticleIと
EssentialParticleJの間の相互作用計算用関数ポインタ、または関数オブジェ
クト。関数の引数は第1引数から順にconst EssentialParticleI *型、
PS::S32型、const EssentialParticleJ *型、PS::S32型、Force *型。

psys: 入力。Tpsys \&型。相互作用を計算したい粒子群クラスのオブジェクト。

dinfo: 入力。Tdinfo \&型。領域クラスのオブジェクト。

clear: 入力。const bool型。前に計算された相互作用の結果をクリアするか
どうかを決定するフラグ。trueならばクリアする。デフォルトtrue。

\item {\bf 返値}

なし

\item {\bf 機能}

粒子群クラスのオブジェクトpsysの粒子すべての相互作用を計算し、その計算
結果をpsysに書き戻す。粒子間相互作用は短距離力の場合に限る。

\end{itemize}

\begin{screen}
\begin{verbatim}
template<class Tfunc_ep_ep,
         class Tfunc_ep_sp,
         class Tpsys,
         class Tdinfo>
void PS::TreeForForce::calcForceAllandWriteBack
             (Tfunc_ep_ep pfunc_ep_ep,
              Tfunc_ep_sp pfunc_ep_sp,
              Tpsys & psys,
              Tdinfo & dinfo
              const bool clear=true);
\end{verbatim}
\end{screen}

\begin{itemize}

\item {\bf 引数}

pfunc\_ep\_ep: 入力。返値がvoid型のEssentialParticleIと
EssentialParticleJの間の相互作用計算用関数ポインタ、または関数オブジェ
クト。関数の引数は第1引数から順にconst EssentialParticleI *型、
PS::S32型、const EssentialParticleJ *型、PS::S32型、Force *型。

pfunc\_ep\_ep: 入力。返値がvoid型のEssentialParticleIとSuperParticleJ
の間の相互作用計算用関数ポインタ、または関数オブジェクト。関数の引数は
第1引数から順にconst EssentialParticleI *型、PS::S32型、const
SuperParticleJ *型、PS::S32型、Force *型。

psys: 入力。Tpsys \&型。相互作用を計算したい粒子群クラスのオブジェクト。

dinfo: 入力。Tdinfo \&型。領域クラスのオブジェクト。

clear: 入力。const bool型。前に計算された相互作用の結果をクリアするか
どうかを決定するフラグ。trueならばクリアする。デフォルトtrue。

\item {\bf 返値}

なし

\item {\bf 機能}

粒子群クラスのオブジェクトpsysの粒子すべての相互作用を計算し、その計算
結果をpsysに書き戻す。粒子間相互作用は長距離力の場合に限る。

\end{itemize}

\subsubsubsubsubsection{calcForceAll}
%% setParticleLocalTree, makeLocalTree, makeGlobalTree,
%% calcMomentGlobalTree, calcForce

\begin{screen}
\begin{verbatim}
template<class Tfunc_ep_ep,
         class Tpsys,
         class Tdinfo>
void PS::TreeForForce::calcForceAll
             (Tfunc_ep_ep pfunc_ep_ep,
              Tpsys & psys,
              Tdinfo & dinfo
              const bool clear=true);
\end{verbatim}
\end{screen}

\begin{itemize}

\item {\bf 引数}

pfunc\_ep\_ep: 入力。返値がvoid型のEssentialParticleIと
EssentialParticleJの間の相互作用計算用関数ポインタ、または関数オブジェ
クト。関数の引数は第1引数から順にconst EssentialParticleI *型、
PS::S32型、const EssentialParticleJ *型、PS::S32型、Force *型。

psys: 入力。Tpsys \&型。相互作用を計算したい粒子群クラスのオブジェクト。

dinfo: 入力。Tdinfo \&型。領域クラスのオブジェクト。

clear: 入力。const bool型。前に計算された相互作用の結果をクリアするか
どうかを決定するフラグ。trueならばクリアする。デフォルトtrue。

\item {\bf 返値}

なし

\item {\bf 機能}

粒子群クラスのオブジェクトpsysの粒子すべての相互作用を計算する。粒子間
相互作用は短距離力の場合に限る。
PS::TreeForForce::calcForceAllAndWriteBackから計算結果の書き戻しがなく
なったもの。

\end{itemize}

\begin{screen}
\begin{verbatim}
template<class Tfunc_ep_ep,
         class Tfunc_ep_sp,
         class Tpsys,
         class Tdinfo>
void PS::TreeForForce::calcForceAll
             (Tfunc_ep_ep pfunc_ep_ep,
              Tfunc_ep_sp pfunc_ep_sp,
              Tpsys & psys,
              Tdinfo & dinfo
              const bool clear=true);
\end{verbatim}
\end{screen}

\begin{itemize}

\item {\bf 引数}

pfunc\_ep\_ep: 入力。返値がvoid型のEssentialParticleIと
EssentialParticleJの間の相互作用計算用関数ポインタ、または関数オブジェ
クト。関数の引数は第1引数から順にconst EssentialParticleI *型、
PS::S32型、const EssentialParticleJ *型、PS::S32型、Force *型。

pfunc\_ep\_ep: 入力。返値がvoid型のEssentialParticleIとSuperParticleJ
の間の相互作用計算用関数ポインタ、または関数オブジェクト。関数の引数は
第1引数から順にconst EssentialParticleI *型、PS::S32型、const
SuperParticleJ *型、PS::S32型、Force *型。

psys: 入力。Tpsys \&型。相互作用を計算したい粒子群クラスのオブジェクト。

dinfo: 入力。Tdinfo \&型。領域クラスのオブジェクト。

clear: 入力。const bool型。前に計算された相互作用の結果をクリアするか
どうかを決定するフラグ。trueならばクリアする。デフォルトtrue。

\item {\bf 返値}

なし

\item {\bf 機能}

粒子群クラスのオブジェクトpsysの粒子すべての相互作用を計算する。粒子間
相互作用は長距離力の場合に限る。
PS::TreeForForce::calcForceAllAndWriteBackから計算結果の書き戻しがなく
なったもの。

\end{itemize}

\subsubsubsubsubsection{calcForceMakingTree}
%% makeLocalTree, makeGlobalTree, calcMomentGlobalTree, calcForce

\begin{screen}
\begin{verbatim}
template<class Tfunc_ep_ep,
         class Tdinfo>
void PS::TreeForForce::calcForceMakingTree
             (Tfunc_ep_ep pfunc_ep_ep,
              Tdinfo & dinfo
              const bool clear=true);
\end{verbatim}
\end{screen}

\begin{itemize}

\item {\bf 引数}

pfunc\_ep\_ep: 入力。返値がvoid型のEssentialParticleIと
EssentialParticleJの間の相互作用計算用関数ポインタ、または関数オブジェ
クト。関数の引数は第1引数から順にconst EssentialParticleI *型、
PS::S32型、const EssentialParticleJ *型、PS::S32型、Force *型。

dinfo: 入力。Tdinfo \&型。領域クラスのオブジェクト。

clear: 入力。const bool型。前に計算された相互作用の結果をクリアするか
どうかを決定するフラグ。trueならばクリアする。デフォルトtrue。

\item {\bf 返値}

なし

\item {\bf 機能}

これより前に相互作用ツリークラスのオブジェクトに読み込まれた粒子群クラ
スのオブジェクトの粒子すべての相互作用を計算する。粒子間相互作用は短距
離力の場合に限る。PS::TreeForForce::calcForceAllAndWriteBackから粒子群
クラスのオブジェクトの読込と計算結果の書き戻しがなくなったもの。

\end{itemize}

\begin{screen}
\begin{verbatim}
template<class Tfunc_ep_ep,
         class Tfunc_ep_sp,
         class Tdinfo>
void PS::TreeForForce::calcForceMakingTree
             (Tfunc_ep_ep pfunc_ep_ep,
              Tfunc_ep_sp pfunc_ep_sp,
              Tdinfo & dinfo
              const bool clear=true);
\end{verbatim}
\end{screen}

\begin{itemize}

\item {\bf 引数}

pfunc\_ep\_ep: 入力。返値がvoid型のEssentialParticleIと
EssentialParticleJの間の相互作用計算用関数ポインタ、または関数オブジェ
クト。関数の引数は第1引数から順にconst EssentialParticleI *型、
PS::S32型、const EssentialParticleJ *型、PS::S32型、Force *型。

pfunc\_ep\_ep: 入力。返値がvoid型のEssentialParticleIとSuperParticleJ
の間の相互作用計算用関数ポインタ、または関数オブジェクト。関数の引数は
第1引数から順にconst EssentialParticleI *型、PS::S32型、const
SuperParticleJ *型、PS::S32型、Force *型。

dinfo: 入力。Tdinfo \&型。領域クラスのオブジェクト。

clear: 入力。const bool型。前に計算された相互作用の結果をクリアするか
どうかを決定するフラグ。trueならばクリアする。デフォルトtrue。

\item {\bf 返値}

なし

\item {\bf 機能}

これより前に相互作用ツリークラスのオブジェクトに読み込まれた粒子群クラ
スのオブジェクトの粒子すべての相互作用を計算する。粒子間相互作用は長距
離力の場合に限る。PS::TreeForForce::calcForceAllAndWriteBackから粒子群
クラスのオブジェクトの読込と計算結果の書き戻しがなくなったもの。

\end{itemize}

\subsubsubsubsubsection{calcForceAndWriteBack}
%% calcForce, getForce
%% いる?

\begin{screen}
\begin{verbatim}
template<class Tfunc_ep_ep,
         class Tpsys>
void PS::TreeForForce::calcForceAndWriteBack
             (Tfunc_ep_ep pfunc_ep_ep,
              Tpsys & psys,
              const bool clear=true);
\end{verbatim}
\end{screen}

\begin{itemize}

\item {\bf 引数}

pfunc\_ep\_ep: 入力。返値がvoid型のEssentialParticleIと
EssentialParticleJの間の相互作用計算用関数ポインタ、または関数オブジェ
クト。関数の引数は第1引数から順にconst EssentialParticleI *型、
PS::S32型、const EssentialParticleJ *型、PS::S32型、Force *型。

psys: 入力。Tpsys \&型。相互作用の計算結果を書き戻したい粒子群クラスの
オブジェクト。

clear: 入力。const bool型。前に計算された相互作用の結果をクリアするか
どうかを決定するフラグ。trueならばクリアする。デフォルトtrue。

\item {\bf 返値}

なし

\item {\bf 機能}

これより前に相互作用ツリークラスのオブジェクトに構築されたグローバルツ
リーとそのモーメントをもとに、相互作用ツリークラスのオブジェクトに属す
る粒子すべての相互作用が計算され、さらにその結果が粒子群クラスのオブジェ
クトpsysに書き戻される。粒子間相互作用は短距離力の場合に限る。
PS::TreeForForce::calcForceAllAndWriteBackから粒子群クラスのオブジェク
トの読込、ローカルツリーの構築、グローバルツリーの構築、グローバルツリー
のモーメントの計算がなくなったもの。

\end{itemize}

\begin{screen}
\begin{verbatim}
template<class Tfunc_ep_ep,
         class Tfunc_ep_sp,
         class Tpsys>
void PS::TreeForForce::calcForceAllandWriteBack
             (Tfunc_ep_ep pfunc_ep_ep,
              Tfunc_ep_sp pfunc_ep_sp,
              Tpsys & psys,
              const bool clear=true);
\end{verbatim}
\end{screen}

\begin{itemize}

\item {\bf 引数}

pfunc\_ep\_ep: 入力。返値がvoid型のEssentialParticleIと
EssentialParticleJの間の相互作用計算用関数ポインタ、または関数オブジェ
クト。関数の引数は第1引数から順にconst EssentialParticleI *型、
PS::S32型、const EssentialParticleJ *型、PS::S32型、Force *型。

pfunc\_ep\_ep: 入力。返値がvoid型のEssentialParticleIとSuperParticleJ
の間の相互作用計算用関数ポインタ、または関数オブジェクト。関数の引数は
第1引数から順にconst EssentialParticleI *型、PS::S32型、const
SuperParticleJ *型、PS::S32型、Force *型。

psys: 入力。Tpsys \&型。相互作用の計算結果を書き戻したい粒子群クラスの
オブジェクト。

clear: 入力。const bool型。前に計算された相互作用の結果をクリアするか
どうかを決定するフラグ。trueならばクリアする。デフォルトtrue。

\item {\bf 返値}

なし

\item {\bf 機能}

これより前に相互作用ツリークラスのオブジェクトに構築されたグローバルツ
リーとそのモーメントをもとに、相互作用ツリークラスのオブジェクトに属す
る粒子すべての相互作用が計算され、さらにその結果が粒子群クラスのオブジェ
クトpsysに書き戻される。粒子間相互作用は長距離力の場合に限る。
PS::TreeForForce::calcForceAllAndWriteBackから粒子群クラスのオブジェク
トの読込、ローカルツリーの構築、グローバルツリーの構築、グローバルツリー
のモーメントの計算がなくなったもの。

\end{itemize}

%%%%%%%%%%%%%%%%%%%%%%%%%%%%%%%%%%%%%%%%%%%%%%%%%%%%%%
\subsubsubsubsection{ネイバーリスト}

今後、追加する。

%%\subsubsubsubsubsection{getNeighborListOneParticle}

%%\subsubsubsubsubsection{getNeighborListOneIPGroup}

%%\subsubsubsubsubsection{getNeighborListOneIPGroupEachParticle}

%%\subsubsubsubsubsection{getNumberOfIPG}


\subsubsection{通信用データクラス}

本節では、通信用データクラスについて記述する。このクラスはノード間通信
のための情報の保持や実際の通信を行うモジュールである。このクラスはシン
グルトンパターンとして管理されており、オブジェクトの生成は必要としない。
ここではこのモジュールのAPIを記述する。

%%%%%%%%%%%%%%%%%%%%%%%%%%%%%%%%%%%%%%%%%%%%%%%%%%%%%%%%%%%%%%%%%
\subsubsubsection{API}

このモジュールのAPIの宣言は以下のようになっている。このあと各APIについ
て記述する。
\begin{lstlisting}[caption=Communication]
namespace ParticleSimulator {
    class Comm{
    public:
        static S32 getRank();
        static S32 getNumberOfProc();
        static S32 getRankMultiDim(const S32 id);
        static S32 getNumberOfProcMultiDim(const S32 id);
        static bool synchronizeConditionalBranchAND(const bool local);
        static bool synchronizeConditionalBranchOR(const bool local);
        template<class T>
        static T getMinValue(const T val);
        template<class Tfloat, class Tint>
        static void getMinValue(const Tfloat f_in,
                                const Tint i_in,
                                Tfloat & f_out,
                                Tint & i_out);
        template<class T>
        static T getMaxValue(const T val);
        template<class Tfloat, class Tint>
        static void getMaxValue(const Tfloat f_in,
                                const Tint i_in,
                                Tfloat & f_out,
                                Tint & i_out );
        template<class T>
        static T getSum(const T val);
    };
}
namespace PS = ParticleSimulator;
\end{lstlisting}

\subsubsubsubsection{getRank}

\begin{screen}
\begin{verbatim}
static PS::S32 getRank();
\end{verbatim}
\end{screen}

\begin{itemize}

\item{{\bf 引数}}

なし。

\item{{\bf 返り値}}

{\tt PS::S32}型。全プロセス中でのランクを返す。

\end{itemize}

\subsubsubsubsection{getNumberOfProc}

\begin{itemize}

\item{{\bf 引数}}

なし。

\item{{\bf 返り値}}

{\tt PS::S32}型。全プロセス数を返す。

\end{itemize}

\subsubsubsubsection{getRankMultiDim}

\begin{screen}
\begin{verbatim}
static PS::S32 PS::Comm::getRankMultiDim(const PS::S32 id);
\end{verbatim}
\end{screen}

\begin{itemize}

\item{{\bf 引数}}

{\tt id}: 入力。{\tt const PS::S32}型。軸の番号。x軸:0, y軸:1, z軸:2。

\item{{\bf 返り値}}

{\tt PS::S32}型。id番目の軸でのランクを返す。2次元の場合、id=2は1を返す。

\end{itemize}

\subsubsubsubsection{getNumberOfProcMultiDim}

\begin{screen}
\begin{verbatim}
static PS::S32 PS::Comm::getNumberOfProcMultiDim(const PS::S32 id);
\end{verbatim}
\end{screen}

\begin{itemize}

\item{{\bf 引数}}

{\tt id}: 入力。{\tt const PS::S32}型。軸の番号。x軸:0, y軸:1, z軸:2。

\item{{\bf 返り値}}

{\tt PS::S32}型。id番目の軸のプロセス数を返す。2次元の場合、id=2は1を返す。

\end{itemize}

\subsubsubsubsection{synchronizeConditionalBranchAND}

\begin{screen}
\begin{verbatim}
static bool PS::Comm::synchronizeConditionalBranchAND(const bool local)
\end{verbatim}
\end{screen}

\begin{itemize}

\item{{\bf 引数}}

{\tt local}: 入力。{\tt const bool}型。

\item{{\bf 返り値}}

{\tt bool}型。全プロセスで{\tt local}の{\tt AND}を取り、結果を返す。

\end{itemize}

\subsubsubsubsection{synchronizeConditionalBranchOR}

\begin{screen}
\begin{verbatim}
static bool PS::Comm::synchronizeConditionalBranchOR(const bool local);
\end{verbatim}
\end{screen}

\begin{itemize}

\item{{\bf 引数}}

{\tt local}: 入力。{\tt const bool}型。

\item{{\bf 返り値}}

{\tt bool}型。全プロセスで{\tt local}の{\tt OR}を取り、結果を返す。

\end{itemize}

\subsubsubsubsection{getMinValue}

\begin{screen}
\begin{verbatim}
template <class T>
static T PS::Comm::getMinValue(const T val);
\end{verbatim}
\end{screen}

\begin{itemize}

\item{{\bf 引数}}

{\tt val}: 入力。{\tt const T}型。

\item{{\bf 返り値}}

{\tt T}型。全プロセスで{\tt val}の最小値を取り、結果を返す。

\end{itemize}

\begin{screen}
\begin{verbatim}
template <class Tfloat, class Tint>
static void PS::Comm::getMinValue(const Tfloat f_in, const Tint i_in,
                                  Tfloat & f_out, Tint & i_out);
\end{verbatim}
\end{screen}

\begin{itemize}

\item{{\bf 引数}}

{\tt f\_in}: 入力。{\tt const Tfloat}型。

{\tt i\_in}: 入力。{\tt const Tint}型。

{\tt f\_out}: 出力。{\tt Tfloat}型。全プロセスで{\tt f\_in}の最小値を取
り、結果を返す。

{\tt i\_out}: 出力。{\tt Tint}型。{\tt f\_out}に伴うIDを返す。

\item{{\bf 返り値}}

なし。

\end{itemize}

\subsubsubsubsection{getMaxValue}

\begin{screen}
\begin{verbatim}
template <class T>
static T PS::Comm::getMaxValue(const T val);
\end{verbatim}
\end{screen}

\begin{itemize}

\item{{\bf 引数}}

{\tt val}: 入力。{\tt const T}型。

\item{{\bf 返り値}}

{\tt T}型。全プロセスで{\tt val}の最大値を取り、結果を返す。

\end{itemize}

\begin{screen}
\begin{verbatim}
template <class Tfloat, class Tint>
static void PS::Comm::getMaxValue(const Tfloat f_in, const Tint i_in,
                                  Tfloat & f_out, Tint & i_out);
\end{verbatim}
\end{screen}

\begin{itemize}

\item{{\bf 引数}}

{\tt f\_in}: 入力。{\tt const Tfloat}型。

{\tt i\_in}: 入力。{\tt const Tint}型。

{\tt f\_out}: 出力。{\tt Tfloat}型。全プロセスで{\tt f\_in}の最大値を取
り、結果を返す。

{\tt i\_out}: 出力。{\tt Tint}型。{\tt f\_out}に伴うIDを返す。

\item{{\bf 返り値}}

なし。

\end{itemize}

\subsubsubsubsection{getSum}

\begin{screen}
\begin{verbatim}
template <class T>
static T PS::Comm::getSum(const T val);
\end{verbatim}
\end{screen}

\begin{itemize}

\item{{\bf 引数}}

{\tt val}: 入力。{\tt const T}型。

\item{{\bf 返り値}}

{\tt T}型。全プロセスで{\tt val}の総和を取り、結果を返す。

\end{itemize}




\subsection{拡張機能}

\subsubsection{概要}

本節では、FDPSの拡張機能について記述する。拡張機能には1つのモジュール
があり、Particle Meshクラスがある。この1つのクラスについて記述する。

\subsubsection{Particle Meshクラス}

本節では、Particle Meshクラスについて記述する。このクラスはParticle
Mesh法を用いて粒子の相互作用を計算するモジュールである。オブジェクトの
生成方法、API、使用済マクロについて記述する。

%%%%%%%%%%%%%%%%%%%%%%%%%%%%%%%%%%%%%%%%%%%%%%%%%%%%%%%%%%%%%%%%%
\subsubsubsection{オブジェクトの生成}

Particle Meshクラスは以下のように宣言されている。
\begin{lstlisting}[caption=ParticleMesh0]
namespace ParticleSimulator {
    namespace ParticleMesh {
        class ParticleMesh;
    }
    namespace PM = ParticleMesh;
}
namespace PS = ParticleSimulator;
\end{lstlisting}

Particle Meshクラスのオブジェクトの生成は以下のように行う。ここではpm
というオブジェクトを生成している。
\begin{screen}
\begin{verbatim}
PS::PM::ParticleMesh pm;
\end{verbatim}
\end{screen}

%%%%%%%%%%%%%%%%%%%%%%%%%%%%%%%%%%%%%%%%%%%%%%%%%%%%%%%%%%%%%%%%%
\subsubsubsection{API}

Particle Meshクラスには初期設定関連のAPI、低レベルAP、高レベルAPIがあ
る。以下、各節に分けて記述する。

%%%%%%%%%%%%%%%%%%%%%%%%%%%%%%%%%%%%%%%%%%%%%%%%%%%%%%
\subsubsubsubsection{初期設定}

初期設定関連のAPIの宣言は以下のようになっている。このあと各APIについて
記述する。
\begin{lstlisting}[caption=ParticleMesh1]
namespace ParticleSimulator {
    namespace ParticleMesh {
        class ParticleMesh{
            ParticleMesh();
        };
    }
    namespace PM = ParticleMesh;
}
namespace PS = ParticleSimulator;
\end{lstlisting}

\subsubsubsubsubsection{コンストラクタ}

\begin{screen}
\begin{verbatim}
void PS::PM::ParticleMesh::ParticleMesh();
\end{verbatim}
\end{screen}

\begin{itemize}

\item {\bf 引数}

なし

\item {\bf 返値}

なし

\item {\bf 機能}

Particle Meshクラスのオブジェクトを生成する。

\end{itemize}

%%%%%%%%%%%%%%%%%%%%%%%%%%%%%%%%%%%%%%%%%%%
\subsubsubsubsection{低レベルAPI}

低レベルAPIの宣言は以下のようになっている。このあと各APIについて記述す
る。
\begin{lstlisting}[caption=ParticleMesh1]
namespace ParticleSimulator {
    namespace ParticleMesh {
        class ParticleMesh{
            template<class Tdinfo>
            void setDomainInfoParticleMesh(const Tdinfo & dinfo);
            template<class Tpsys>
            void setParticleParticleMesh(const Tpsys & psys,
                                         const bool clear=true);
            void calcMeshForceOnly();
            F32vec getForce(F32vec pos);
        };
    }
    namespace PM = ParticleMesh;
}
namespace PS = ParticleSimulator;
\end{lstlisting}

\subsubsubsubsubsection{setDomainInfoParticleMesh}

\begin{screen}
\begin{verbatim}
template<class Tdinfo>
void PS::PM::ParticleMesh::setDomainInfoParticleMesh
            (const Tdinfo & dinfo);
\end{verbatim}
\end{screen}

\begin{itemize}

\item {\bf 引数}

dinfo: 入力。Tdinfo \&型。領域クラスのオブジェクト。

\item {\bf 返値}

なし

\item {\bf 機能}

領域情報を読み込む。

\end{itemize}

\subsubsubsubsubsection{setParticleParticleMesh}

\begin{screen}
\begin{verbatim}
template<class Tpsys>
void PS::PM::ParticleMesh::setParticleParticleMesh
            (const Tpsys & psys,
             const bool clear=true);
\end{verbatim}
\end{screen}

\begin{itemize}

\item {\bf 引数}

  psys: 入力。Tpsys \& 型。粒子群クラスのオブジェクト。

  clear: 入力。const bool型。これまで読込んだ粒子情報をクリアするかど
  うか決定するフラグ。trueならばクリアする。デフォルトはtrue。

\item {\bf 返値}

  なし

\item {\bf 機能}

  粒子情報を粒子群クラスのオブジェクトから読み込む。

\end{itemize}

\subsubsubsubsubsection{calcMeshForceOnly}

\begin{screen}
\begin{verbatim}
void PS::PM::ParticleMesh::calcMeshForceOnly();
\end{verbatim}
\end{screen}

\begin{itemize}

\item {\bf 引数}

  なし

\item {\bf 返値}

  なし

\item {\bf 機能}

  メッシュ上の力を計算する。

\end{itemize}

\subsubsubsubsubsection{getForce}

\begin{screen}
\begin{verbatim}
PS::F32vec PS::PM::ParticleMesh::getForce
             (F32vec pos);
\end{verbatim}
\end{screen}

\begin{itemize}

\item {\bf 引数}

  pos: 入力。PS::F32vec型。メッシュに課された粒子からの力を計算したい
  位置。

\item {\bf 返値}

  PS::F32vec型。メッシュに課された粒子からの力。

\item {\bf 機能}

  位置posでのメッシュに課された粒子からの力を返す。

\end{itemize}

%%%%%%%%%%%%%%%%%%%%%%%%%%%%%%%%%%%%%%%%%%%
\subsubsubsubsection{高レベルAPI}

高レベルAPIの宣言は以下のようになっている。このあと各APIについて記述す
る。
\begin{lstlisting}[caption=ParticleMesh1]
namespace ParticleSimulator {
    namespace ParticleMesh {
        class ParticleMesh{
            template<class Tpsys,
                     class Tdinfo>
            void calcForceAllAndWriteBack(Tpsys & psys,
                                          const Tdinfo & dinfo);
        };
    }
    namespace PM = ParticleMesh;
}
namespace PS = ParticleSimulator;
\end{lstlisting}

\subsubsubsubsubsection{calcForceAllAndWriteBack}

\begin{screen}
\begin{verbatim}
template<class Tpsys,
         class Tdinfo>
void PS::PM::ParticleMesh::calcForceAllAndWriteBack
            (Tpsys & psys,
             const Tdinfo & dinfo);
\end{verbatim}
\end{screen}

\begin{itemize}

\item {\bf 引数}

  psys: 入力であり出力。Tpsys \& 型。粒子群クラスのオブジェクト。

  dinfo: 入力。const Tdinfo \&型。領域クラスのオブジェクト。

\item {\bf 返値}

  なし

\item {\bf 機能}

  粒子群クラスのオブジェクトpsysに含まれる粒子間のメッシュ力を計算し、
  その結果をpsysに返す。

\end{itemize}

%%%%%%%%%%%%%%%%%%%%%%%%%%%%%%%%%%%%%%%%%%%%%%%%%%%%%%%%%%%%%%%%%
\subsubsubsection{使用済マクロ}
%%BINARY_BOUNDARY, BOUNDARY_COMM_NONBLOCKING, BOUNDARY_SMOOTHING,
%%BUFFER_FOR_TREE, CALCPOT, CLEAN_BOUNDARY_PARTICLE,
%%CONSTANT_TIMESTEP, EXCHANGE_COMM_NONBLOCKING, FFT3D, FFTW3_PARALLEL,
%%FFTW_DOUBLE, FIX_FFTNODE, GADGET_IO, GRAPE_OFF, KCOMPUTER, LONG_ID,
%%MAKE_LIST_PROF, MERGE_SNAPSHOT, MULTI_TIMESTEP, MY_MPI_BARRIER,
%%N128_2H, N256_2H, N256_H, N32_2H, N512_2H, NEW_DECOMPOSITION, NOACC,
%%NPART_DIFFERENT_DUMP, OMP_SCHDULE_DISABLE, PRINT_TANIKAWA,
%%REVERSE_ENDIAN_INPUT, REVERSE_ENDIAN_OUTPUT, RMM_PM,
%%SHIFT_INITIAL_BOUNDARY, STATIC_ARRAY, TREE2,
%%TREECONSTRUCTION_PARALLEL, TREE_PARTICLE_CACHE, UNIFORM, UNSTABLE,
%%USING_MPI_PARTICLE, VERBOSE_MODE, VERBOSE_MODE2

このモジュールでは多くのマクロを使っている。これらを別のマクロとして使
用した場合にプログラムが正しく動作する保証はない。ここでは使用されてい
るマクロをアルファベティカルに列挙する。
\begin{itemize}
\item BINARY\_BOUNDARY
\item BOUNDARY\_COMM\_NONBLOCKING
\item BOUNDARY\_SMOOTHING
\item BUFFER\_FOR\_TREE
\item CALCPOT
\item CLEAN\_BOUNDARY\_PARTICLE
\item CONSTANT\_TIMESTEP
\item EXCHANGE\_COMM\_NONBLOCKING
\item FFT3D
\item FFTW3\_PARALLEL
\item FFTW\_DOUBLE
\item FIX\_FFTNODE
\item GADGET\_IO
\item GRAPE\_OFF
\item KCOMPUTER
\item LONG\_ID
\item MAKE\_LIST\_PROF
\item MERGE\_SNAPSHOT
\item MULTI\_TIMESTEP
\item MY\_MPI\_BARRIER
\item N128\_2H
\item N256\_2H
\item N256\_H
\item N32\_2H
\item N512\_2H
\item NEW\_DECOMPOSITION
\item NOACC
\item NPART\_DIFFERENT\_DUMP
\item OMP\_SCHDULE\_DISABLE
\item PRINT\_TANIKAWA
\item REVERSE\_ENDIAN\_INPUT
\item REVERSE\_ENDIAN\_OUTPUT
\item RMM\_PM
\item SHIFT\_INITIAL\_BOUNDARY
\item STATIC\_ARRAY
\item TREE2
\item TREECONSTRUCTION\_PARALLEL
\item TREE\_PARTICLE\_CACHE
\item UNIFORM
\item UNSTABLE
\item USING\_MPI\_PARTICLE
\item VERBOSE\_MODE
\item VERBOSE\_MODE2。
\end{itemize}






\newpage

%%%%%%%%%%%%%%%%%%%%%%%%%%%%%%%%%%%%%%%%%%%%%%%%%%%%%
\section{エラーメッセージ}

\subsection{概要}

%%\subsection{入力ファイルがない場合}

%%\subsection{メモリ確保に失敗した場合}

%%\subsection{規定より大きな配列確保をしようとした場合}

%%\subsection{粒子がツリーのルートセルからはみでている場合}

%%\subsection{不適切な初期設定}

\newpage

%%%%%%%%%%%%%%%%%%%%%%%%%%%%%%%%%%%%%%%%%%%%%%%%%%%%%
\section{よくしこむバグ(仮)}

\subsection{概要}

\subsection{ユーザー定義クラス}

\subsubsection{概要}

\subsubsection{FullParticle型}

\subsubsection{EssentialParticleI型}

\subsubsection{EssentialParticleJ型}

\subsubsection{SuperParticleJ型}

\subsubsection{Moment型}

\subsubsection{Force型}

\subsubsection{calcForceEpEp型}

\subsubsection{calcForceSpEp型}

\subsubsection{ヘッダ型}

\subsection{プログラム本体}

\subsubsection{概要}

\subsubsection{オブジェクトの生成}

\newpage

%%%%%%%%%%%%%%%%%%%%%%%%%%%%%%%%%%%%%%%%%%%%%%%%%%%%%
\section{よく知られているバグ}

% 周期境界での安易なシフト
% 例: 領域0<=x<1, xi=1e-20, xi+=1. -> xi=1.(はみ出る)

\newpage

%%%%%%%%%%%%%%%%%%%%%%%%%%%%%%%%%%%%%%%%%%%%%%%%%%%%%
\section{限界}

\newpage

%%%%%%%%%%%%%%%%%%%%%%%%%%%%%%%%%%%%%%%%%%%%%%%%%%%%%
\section{ユーザーサポート}

\subsection{概要}

\subsection{ユーザーへのお願い}

\newpage

%%%%%%%%%%%%%%%%%%%%%%%%%%%%%%%%%%%%%%%%%%%%%%%%%%%%%
\section{ライセンス}

MITライセンスに準ずる。標準機能のみ使用する場合は、Iwasawa et al.(2015
in prep)の引用を義務とする。拡張機能のうちParticle Meshクラスを使用する
場合は、上記に加え、Ishiyama, Fukushige \& Makino (2009), Ishiyama,
Nitadori \& Makino (2012)の引用を義務とする。

\end{document}
