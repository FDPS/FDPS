\subsection{概要}

FDPSでは独自の整数型、実数型、ベクトル型、行列型、SEARCH\_MODE型、列挙
型が定義されている。整数型、実数型、ベクトル型、行列型に関しては必ずし
もここに挙げるものを用いる必要はないが、これらを用いることを推奨する。
SEARCH\_MODE型、列挙型は必ず用いる必要がある。以下、整数型、実数型、ベ
クトル型、行列型、SEARCH\_MODE型、列挙型の順に記述する。

\subsection{整数型}

\subsubsection{概要}

整数型にはPS::S32, PS::S64, PS::U32, PS::U64がある。以下、順にこれらを
記述する。

\subsubsection{PS::S32}

PS::S32は以下のように定義されている。すなわち32bitの符号付き整数である。
\begin{lstlisting}[caption=S32]
namespace ParticleSimulator {
    typedef int S32;
}
namespace PS = ParticleSimulator;
\end{lstlisting}

ただし、GCCコンパイラとKコンパイラでのみ32bitであることが保証されている。

\subsubsection{PS::S64}

PS::S64は以下のように定義されている。すなわち64bitの符号付き整数である。
\begin{lstlisting}[caption=S64]
namespace ParticleSimulator {
    typedef long S64;
}
namespace PS = ParticleSimulator;
\end{lstlisting}

ただし、GCCコンパイラとKコンパイラでのみ64bitであることが保証されている。

\subsubsection{PS::U32}

PS::U32は以下のように定義されている。すなわち32bitの符号なし整数である。
\begin{lstlisting}[caption=U32]
namespace ParticleSimulator {
    typedef unsinged U32;
}
namespace PS = ParticleSimulator;
\end{lstlisting}

ただし、GCCコンパイラとKコンパイラでのみ32bitであることが保証されている。

\subsubsection{PS::U64}

PS::U64は以下のように定義されている。すなわち64bitの符号なし整数である。
\begin{lstlisting}[caption=U64]
namespace ParticleSimulator {
    typedef unsinged U64;
}
namespace PS = ParticleSimulator;
\end{lstlisting}

ただし、GCCコンパイラとKコンパイラでのみ64bitであることが保証されている。



\subsection{実数型}

\subsubsection{概要}

実数型にはPS::F32, PS::F64がある。以下、順にこれらを記述する。

\subsubsection{PS::F32}

PS::F32は以下のように定義されている。すなわち32bitの浮動小数点数である。
\begin{lstlisting}[caption=F32]
namespace ParticleSimulator {
    typedef float F32;
}
namespace PS = ParticleSimulator;
\end{lstlisting}

\subsubsection{PS::F64}

PS::F64は以下のように定義されている。すなわち64bitの浮動小数点数である。
\begin{lstlisting}[caption=F64]
namespace ParticleSimulator {
    typedef double F64;
}
namespace PS = ParticleSimulator;
\end{lstlisting}


\subsection{ベクトル型}

\subsubsection{概要}

ベクトル型には2次元ベクトル型PS::Vector2と3次元ベクトル型PS::Vector3が
ある。まずこれら2つを記述する。最後にこれらベクトル型のラッパーについ
て記述する。

\subsubsection{PS::Vector2}

PS::Vector2はx, yの2要素を持つ。これらに対する様々なAPIや演算子を定義
した。それらの宣言を以下に記述する。この節ではこれらについて詳しく記述
する。
\begin{lstlisting}[caption=Vector2]
namespace ParticleSimulator{
    template <typename T>
    class Vector2{
    public:
        //メンバ変数2要素
        T x, y;

        //コンストラクタ
        Vector2();
        Vector2(const T _x, const T _y) : x(_x), y(_y) {}
        Vector2(const T s) : x(s), y(s) {}
        Vector2(const Vector2 & src) : x(src.x), y(src.y) {}

        //代入演算子
        const Vector2 & operator = (const Vector2 & rhs);

        //加減算
        Vector2 operator + (const Vector2 & rhs) const;
        const Vector2 & operator += (const Vector2 & rhs);
        Vector2 operator - (const Vector2 & rhs) const;
        const Vector2 & operator -= (const Vector2 & rhs);

        //ベクトルスカラ積
        Vector2 operator * (const T s) const;
        const Vector2 & operator *= (const T s);
        friend Vector2 operator * (const T s, const Vector2 & v);
        Vector2 operator / (const T s) const;
        const Vector2 & operator /= (const T s);

        //内積
        T operator * (const Vector2 & rhs) const;

        //外積(返り値はスカラ!!)
        T operator ^ (const Vector2 & rhs) const;

        //Vector2<U>への型変換
        template <typename U>
        operator Vector2<U> () const;
    };
}
namespace PS = ParticleSimulator;
\end{lstlisting}

\subsubsubsection{コンストラクタ}

\begin{screen}
\begin{verbatim}
template<typename T>
PS::Vector2<T>()
\end{verbatim}
\end{screen}

\begin{itemize}

\item{{\bf 引数}}

なし。

\item{{\bf 機能}}

デフォルトコンストラクタ。メンバx,yは0で初期化される。

\end{itemize}

\begin{screen}
\begin{verbatim}
template<typename T>
PS::Vector2<T>(const T _x, const T _y)
\end{verbatim}
\end{screen}

\begin{itemize}

\item{{\bf 引数}}

{\tt \_x}: 入力。{\tt const T}型。

{\tt \_y}: 入力。{\tt const T}型。

\item{{\bf 機能}}

メンバ{\tt x}、{\tt y}をそれぞれ{\tt \_x}、{\tt \_y}で初期化する。

\end{itemize}

\begin{screen}
\begin{verbatim}
template<typename T>
PS::Vector2<T>(const T s);
\end{verbatim}
\end{screen}

\begin{itemize}

\item{{\bf 引数}}

{\tt s}: 入力。{\tt const T}型。

\item{{\bf 機能}}

メンバ{\tt x}、{\tt y}を両方とも{\tt s}の値で初期化する。

\end{itemize}

\begin{screen}
\begin{verbatim}
template<typename T>
PS::Vector2<T>(const PS::Vector2<T> & src)
\end{verbatim}
\end{screen}

\begin{itemize}

\item{{\bf 引数}}

{\tt src}: 入力。{\tt const PS::Vector2<T> \&}型。

\item{{\bf 機能}}

コピーコンストラクタ。{\tt src}で初期化する。

\end{itemize}

\subsubsubsection{代入演算子}

\begin{screen}
\begin{verbatim}
template<typename T>
const PS::Vector2<T> & PS::Vector2<T>::operator = 
                       (const PS::Vector2<T> & rhs);
\end{verbatim}
\end{screen}

\begin{itemize}

\item{{\bf 引数}}

{\tt rhs}: 入力。{\tt const PS::Vector2<T> \&}型。

\item{{\bf 返り値}}

{\tt const PS::Vector2<T> \&}型。{\tt rhs}のx,yの値を自身のメンバx,yに
代入し自身の参照を返す。代入演算子。

\end{itemize}

\subsubsubsection{加減算}

\begin{screen}
\begin{verbatim}
template<typename T>
PS::Vector2<T> PS::Vector2<T>::operator + 
               (const PS::Vector2<T> & rhs) const;
\end{verbatim}
\end{screen}

\begin{itemize}

\item{{\bf 引数}}

{\tt rhs}: 入力。{\tt const PS::Vector2<T> \&}型。

\item{{\bf 返り値}}

{\tt PS::Vector2<T> }型。{\tt rhs}のx,yの値と自身のメンバx,yの値の和を
取った値を返す。

\end{itemize}

\begin{screen}
\begin{verbatim}
template<typename T>
const PS::Vector2<T> & PS::Vector2<T>::operator += 
                       (const PS::Vector2<T> & rhs);
\end{verbatim}
\end{screen}

\begin{itemize}

\item{{\bf 引数}}

{\tt rhs}: 入力。{\tt const PS::Vector2<T> \&}型。

\item{{\bf 返り値}}

{\tt const PS::Vector2<T> \&}型。{\tt rhs}のx,yの値を自身のメンバx,yに足し、自
身を返す。

\end{itemize}

\begin{screen}
\begin{verbatim}
template<typename T>
PS::Vector2<T> PS::Vector2<T>::operator - 
               (const PS::Vector2<T> & rhs) const;
\end{verbatim}
\end{screen}

\begin{itemize}

\item{{\bf 引数}}

{\tt rhs}: 入力。{\tt const PS::Vector2<T> \&}型。

\item{{\bf 返り値}}

{\tt PS::Vector2<T> }型。{\tt rhs}のx,yの値と自身のメンバx,yの値の差を
取った値を返す。

\end{itemize}

\begin{screen}
\begin{verbatim}
template<typename T>
const PS::Vector2<T> & PS::Vector2<T>::operator -= 
                       (const PS::Vector2<T> & rhs);
\end{verbatim}
\end{screen}

\begin{itemize}

\item{{\bf 引数}}

{\tt rhs}: 入力。{\tt const PS::Vector2<T> \&}型。

\item{{\bf 返り値}}

{\tt const PS::Vector2<T> \&}型。自身のメンバx,yから{\tt rhs}のx,yを引
き自身を返す。

\end{itemize}

\subsubsubsection{ベクトルスカラ積}

\begin{screen}
\begin{verbatim}
template<typename T>
PS::Vector2<T> PS::Vector2<T>::operator * (const T s) const;
\end{verbatim}
\end{screen}

\begin{itemize}

\item{{\bf 引数}}

{\tt s}: 入力。{\tt const T}型。

\item{{\bf 返り値}}

{\tt PS::Vector2<T>}型。自身のメンバx,yそれぞれに{\tt s}をかけた値を返
す。

\end{itemize}

\begin{screen}
\begin{verbatim}
template<typename T>
const PS::Vector2<T> & PS::Vector2<T>::operator *= (const T s);
\end{verbatim}
\end{screen}

\begin{itemize}

\item{{\bf 引数}}

{\tt rhs}: 入力。{\tt const T}型。

\item{{\bf 返り値}}

{\tt const PS::Vector2<T> \&}型。自身のメンバx,yそれぞれに{\tt s}をかけ
自身を返す。

\end{itemize}

\begin{screen}
\begin{verbatim}
template<typename T>
PS::Vector2<T> PS::Vector2<T>::operator / (const T s) const;
\end{verbatim}
\end{screen}

\begin{itemize}

\item{{\bf 引数}}

{\tt s}: 入力。{\tt const T}型。

\item{{\bf 返り値}}

{\tt PS::Vector2<T>}型。自身のメンバx,yそれぞれを{\tt s}で割った値を返
す。

\end{itemize}

\begin{screen}
\begin{verbatim}
template<typename T>
const PS::Vector2<T> & PS::Vector2<T>::operator /= (const T s);
\end{verbatim}
\end{screen}

\begin{itemize}

\item{{\bf 引数}}

{\tt rhs}: 入力。{\tt const T}型。

\item{{\bf 返り値}}

{\tt const PS::Vector2<T> \&}型。自身のメンバx,yそれぞれを{\tt s}で割り
自身を返す。

\end{itemize}

\subsubsubsection{内積、外積}

\begin{screen}
\begin{verbatim}
template<typename T>
T PS::Vector2<T>::operator * (const PS::Vector2<T> & rhs) const;
\end{verbatim}
\end{screen}

\begin{itemize}

\item{{\bf 引数}}

{\tt rhs}: 入力。{\tt const PS::Vector2<T> \&}型。

\item{{\bf 返り値}}

{\tt T}型。自身と{\tt rhs}の内積を取った値を返す。

\end{itemize}

\begin{screen}
\begin{verbatim}
template<typename T>
T PS::Vector2<T>::operator ^ (const PS::Vector2<T> & rhs) const;
\end{verbatim}
\end{screen}

\begin{itemize}

\item{{\bf 引数}}

{\tt rhs}: 入力。{\tt const PS::Vector2<T> \&}型。

\item{{\bf 返り値}}

{\tt T}型。自身と{\tt rhs}の外積を取った値を返す。

\end{itemize}

\subsubsubsection{{\tt Vector2<U>}への型変換}

\begin{screen}
\begin{verbatim}
template<typename T>
template <typename U>
PS::Vector2<T>::operator PS::Vector2<U> () const;
\end{verbatim}
\end{screen}

\begin{itemize}

\item{{\bf 引数}}

  なし。

\item{{\bf 返り値}}

  {\tt const PS::Vector2<U>}型。

\item{{\bf 機能}}

  {\tt const PS::Vector2<T>}型を{\tt const PS::Vector2<U>}型にキャ
  ストする。

\end{itemize}




\subsubsection{PS::Vector3}

PS::Vecotr3はx, y, zの2要素を持つ。これらに対する様々なAPIや演算子を定
義した。それらの宣言を以下に記述する。この節ではこれらについて詳しく記
述する。
\begin{lstlisting}[caption=Vector3]
namespace ParticleSimulator{
    template <typename T>
    class Vector3{
    public:
        //メンバ変数は以下の二つのみ。
        T x, y, z;

        //コンストラクタ
        Vector3() : x(T(0)), y(T(0)), z(T(0)) {}
        Vector3(const T _x, const T _y, const T _z) : x(_x), y(_y), z(_z) {}
        Vector3(const T s) : x(s), y(s), z(s) {}
        Vector3(const Vector3 & src) : x(src.x), y(src.y), z(src.z) {}

        //代入演算子
        const Vector3 & operator = (const Vector3 & rhs);

        //加減算
        Vector3 operator + (const Vector3 & rhs) const;
        const Vector3 & operator += (const Vector3 & rhs);
        Vector3 operator - (const Vector3 & rhs) const;
        const Vector3 & operator -= (const Vector3 & rhs);

        //ベクトルスカラ積
        Vector3 operator * (const T s) const;
        const Vector3 & operator *= (const T s);
        friend Vector3 operator * (const T s, const Vector3 & v);
        Vector3 operator / (const T s) const;
        const Vector3 & operator /= (const T s);

        //内積
        T operator * (const Vector3 & rhs) const;

        //外積(返り値はスカラ!!)
        T operator ^ (const Vector3 & rhs) const;

        //Vector3<U>への型変換
        template <typename U>
        operator Vector3<U> () const;
    };
}
\end{lstlisting}
%%%%%%%%%%%%%%%%%%%%%%%%%%%%%
\subsubsubsection{コンストラクタ}
\mbox{}
%%%%%%%%%%%%%%%%%%%%%%%%%%%%%
%%%%%%%%%%%%%%%%%%%%%%%%%%%%%
\begin{screen}
\begin{verbatim}
template<typename T>
PS::Vector3<T>()
\end{verbatim}
\end{screen}

\begin{itemize}

\item{{\bf 引数}}

なし。

\item{{\bf 機能}}

デフォルトコンストラクタ。メンバx,yは0で初期化される。

\end{itemize}

%%%%%%%%%%%%%%%%%%%%%%%%%%%%%
\begin{screen}
\begin{verbatim}
template<typename T>
PS::Vector3<T>(const T _x, const T _y)
\end{verbatim}
\end{screen}

\begin{itemize}

\item{{\bf 引数}}

{\tt \_x}: 入力。{\tt const T}型。

{\tt \_y}: 入力。{\tt const T}型。

\item{{\bf 機能}}

メンバ{\tt x}、{\tt y}をそれぞれ{\tt \_x}、{\tt \_y}で初期化する。

\end{itemize}

%%%%%%%%%%%%%%%%%%%%%%%%%%%%%
\begin{screen}
\begin{verbatim}
template<typename T>
PS::Vector3<T>(const T s);
\end{verbatim}
\end{screen}

\begin{itemize}

\item{{\bf 引数}}

{\tt s}: 入力。{\tt const T}型。

\item{{\bf 機能}}

メンバ{\tt x}、{\tt y}を両方とも{\tt s}の値で初期化する。

\end{itemize}

%%%%%%%%%%%%%%%%%%%%%%%%%%%%%
\begin{screen}
\begin{verbatim}
template<typename T>
PS::Vector3<T>(const PS::Vector3<T> & src)
\end{verbatim}
\end{screen}

\begin{itemize}

\item{{\bf 引数}}

{\tt src}: 入力。{\tt const PS::Vector3<T> \&}型。

\item{{\bf 機能}}

コピーコンストラクタ。{\tt src}で初期化する。

\end{itemize}

%%%%%%%%%%%%%%%%%%%%%%%%%%%%%
\subsubsubsection{代入演算子}
\mbox{}
%%%%%%%%%%%%%%%%%%%%%%%%%%%%%

%%%%%%%%%%%%%%%%%%%%%%%%%%%%%
\begin{screen}
\begin{verbatim}
template<typename T>
const PS::Vector3<T> & PS::Vector3<T>::operator = 
                       (const PS::Vector3<T> & rhs);
\end{verbatim}
\end{screen}

\begin{itemize}

\item{{\bf 引数}}

{\tt rhs}: 入力。{\tt const PS::Vector3<T> \&}型。

\item{{\bf 返り値}}

{\tt const PS::Vector3<T> \&}型。{\tt rhs}のx,yの値を自身のメンバx,yに
代入し自身の参照を返す。代入演算子。

\end{itemize}


%%%%%%%%%%%%%%%%%%%%%%%%%%%%%
\subsubsubsection{加減算}
\mbox{}
%%%%%%%%%%%%%%%%%%%%%%%%%%%%%

%%%%%%%%%%%%%%%%%%%%%%%%%%%%%
\begin{screen}
\begin{verbatim}
template<typename T>
PS::Vector3<T> PS::Vector3<T>::operator + 
               (const PS::Vector3<T> & rhs) const;
\end{verbatim}
\end{screen}

\begin{itemize}

\item{{\bf 引数}}

{\tt rhs}: 入力。{\tt const PS::Vector3<T> \&}型。

\item{{\bf 返り値}}

{\tt PS::Vector3<T> }型。{\tt rhs}のx,yの値と自身のメンバx,yの値の和を
取った値を返す。

\end{itemize}


%%%%%%%%%%%%%%%%%%%%%%%%%%%%%
\begin{screen}
\begin{verbatim}
template<typename T>
const PS::Vector3<T> & PS::Vector3<T>::operator += 
                       (const PS::Vector3<T> & rhs);
\end{verbatim}
\end{screen}

\begin{itemize}

\item{{\bf 引数}}

{\tt rhs}: 入力。{\tt const PS::Vector3<T> \&}型。

\item{{\bf 返り値}}

{\tt const PS::Vector3<T> \&}型。{\tt rhs}のx,yの値を自身のメンバx,yに足し、自
身を返す。

\end{itemize}


%%%%%%%%%%%%%%%%%%%%%%%%%%%%%
\begin{screen}
\begin{verbatim}
template<typename T>
PS::Vector3<T> PS::Vector3<T>::operator - 
               (const PS::Vector3<T> & rhs) const;
\end{verbatim}
\end{screen}

\begin{itemize}

\item{{\bf 引数}}

{\tt rhs}: 入力。{\tt const PS::Vector3<T> \&}型。

\item{{\bf 返り値}}

{\tt PS::Vector3<T> }型。{\tt rhs}のx,yの値と自身のメンバx,yの値の差を
取った値を返す。

\end{itemize}


%%%%%%%%%%%%%%%%%%%%%%%%%%%%%
\begin{screen}
\begin{verbatim}
template<typename T>
const PS::Vector3<T> & PS::Vector3<T>::operator -= 
                       (const PS::Vector3<T> & rhs);
\end{verbatim}
\end{screen}

\begin{itemize}

\item{{\bf 引数}}

{\tt rhs}: 入力。{\tt const PS::Vector3<T> \&}型。

\item{{\bf 返り値}}

{\tt const PS::Vector3<T> \&}型。自身のメンバx,yから{\tt rhs}のx,yを引
き自身を返す。

\end{itemize}

%%%%%%%%%%%%%%%%%%%%%%%%%%%%%
\subsubsubsection{ベクトルスカラ積}
\mbox{}
%%%%%%%%%%%%%%%%%%%%%%%%%%%%%

%%%%%%%%%%%%%%%%%%%%%%%%%%%%%
\begin{screen}
\begin{verbatim}
template<typename T>
PS::Vector3<T> PS::Vector3<T>::operator * (const T s) const;
\end{verbatim}
\end{screen}

\begin{itemize}

\item{{\bf 引数}}

{\tt s}: 入力。{\tt const T}型。

\item{{\bf 返り値}}

{\tt PS::Vector3<T>}型。自身のメンバx,yそれぞれに{\tt s}をかけた値を返
す。

\end{itemize}


%%%%%%%%%%%%%%%%%%%%%%%%%%%%%
\begin{screen}
\begin{verbatim}
template<typename T>
const PS::Vector3<T> & PS::Vector3<T>::operator *= (const T s);
\end{verbatim}
\end{screen}

\begin{itemize}

\item{{\bf 引数}}

{\tt rhs}: 入力。{\tt const T}型。

\item{{\bf 返り値}}

{\tt const PS::Vector3<T> \&}型。自身のメンバx,yそれぞれに{\tt s}をかけ
自身を返す。

\end{itemize}


%%%%%%%%%%%%%%%%%%%%%%%%%%%%%
\begin{screen}
\begin{verbatim}
template<typename T>
PS::Vector3<T> PS::Vector3<T>::operator / (const T s) const;
\end{verbatim}
\end{screen}

\begin{itemize}

\item{{\bf 引数}}

{\tt s}: 入力。{\tt const T}型。

\item{{\bf 返り値}}

{\tt PS::Vector3<T>}型。自身のメンバx,yそれぞれを{\tt s}で割った値を返
す。

\end{itemize}


%%%%%%%%%%%%%%%%%%%%%%%%%%%%%
\begin{screen}
\begin{verbatim}
template<typename T>
const PS::Vector3<T> & PS::Vector3<T>::operator /= (const T s);
\end{verbatim}
\end{screen}

\begin{itemize}

\item{{\bf 引数}}

{\tt rhs}: 入力。{\tt const T}型。

\item{{\bf 返り値}}

{\tt const PS::Vector3<T> \&}型。自身のメンバx,yそれぞれを{\tt s}で割り
自身を返す。

\end{itemize}


%%%%%%%%%%%%%%%%%%%%%%%%%%%%%
\subsubsubsection{内積、外積}
\mbox{}
%%%%%%%%%%%%%%%%%%%%%%%%%%%%%

%%%%%%%%%%%%%%%%%%%%%%%%%%%%%
\begin{screen}
\begin{verbatim}
template<typename T>
T PS::Vector3<T>::operator * (const PS::Vector3<T> & rhs) const;
\end{verbatim}
\end{screen}

\begin{itemize}

\item{{\bf 引数}}

{\tt rhs}: 入力。{\tt const PS::Vector3<T> \&}型。

\item{{\bf 返り値}}

{\tt T}型。自身と{\tt rhs}の内積を取った値を返す。

\end{itemize}

%%%%%%%%%%%%%%%%%%%%%%%%%%%%%
\begin{screen}
\begin{verbatim}
template<typename T>
T PS::Vector3<T>::operator ^ (const PS::Vector3<T> & rhs) const;
\end{verbatim}
\end{screen}

\begin{itemize}

\item{{\bf 引数}}

{\tt rhs}: 入力。{\tt const PS::Vector3<T> \&}型。

\item{{\bf 返り値}}

{\tt T}型。自身と{\tt rhs}の外積を取った値を返す。

\end{itemize}


%%%%%%%%%%%%%%%%%%%%%%%%%%%%%
\subsubsubsection{{\tt Vector3<U>}への型変換}
\mbox{}
%%%%%%%%%%%%%%%%%%%%%%%%%%%%%

%%%%%%%%%%%%%%%%%%%%%%%%%%%%%
\begin{screen}
\begin{verbatim}
template<typename T>
template <typename U>
PS::Vector3<T>::operator PS::Vector3<U> () const;
\end{verbatim}
\end{screen}

\begin{itemize}

\item{{\bf 引数}}

  なし

\item{{\bf 返り値}}

  {\tt const PS::Vector3<U>}型。

\item{{\bf 機能}}

  {\tt const PS::Vector3<T>}型を{\tt const PS::Vector3<U>}型にキャ
  ストする。

\end{itemize}


\subsubsection{ベクトル型のラッパー}

ベクトル型のラッパーの定義を以下に示す。
\begin{lstlisting}[caption=vectorwrapper]
namespace ParticleSimulator{
    typedef Vector2<F32> F32vec2;
    typedef Vector3<F32> F32vec3;
    typedef Vector2<F64> F64vec2;
    typedef Vector3<F64> F64vec3;
#ifdef PARTICLE_SIMULATOR_TOW_DIMENSION
    typedef F32vec2 F32vec;
    typedef F64vec2 F64vec;
#else
    typedef F32vec3 F32vec;
    typedef F64vec3 F64vec;
#endif
}
namespace PS = ParticleSimulator;
\end{lstlisting}

すなわちPS::F32vec2, PS::F32vec3, PS::F64vec2, PS::F64vec3はそれぞれ単
精度2次元ベクトル、倍精度2次元ベクトル、単精度3次元ベクトル、倍精度3次
元ベクトルである。FDPSで扱う空間座標系を2次元とした場合、PS::F32vecと
PS::F64vecはそれぞれ単精度2次元ベクトル、倍精度2次元ベクトルとなる。一
方、FDPSで扱う空間座標系を3次元とした場合、PS::F32vecとPS::F64vecはそ
れぞれ単精度3次元ベクトル、倍精度3次元ベクトルとなる。





\subsection{対称行列型}

\subsubsection{概要}

対称行列型には2x2対称行列型PS::MatrixSym2と3x3対称行列型PS::MatrixSym3
がある。まずこれら2つを記述する。最後にこれら対称行列型のラッパーにつ
いて記述する。

\subsubsection{PS::MatrixSym2}

PS::MatrixSym2はxx, yy, xyの3要素を持つ。これらに対する様々なAPIや演算
子を定義した。それらの宣言を以下に記述する。この節ではこれらについて詳
しく記述する。
\begin{lstlisting}[caption=MatrixSym2]
namespace ParticleSimulator{
    template<class T>
    class MatrixSym2{
    public:
        // メンバ変数3要素
        T xx, yy, xy;

        // コンストラクタ
        MatrixSym2() : xx(T(0)), yy(T(0)), xy(T(0)) {}
        MatrixSym2(const T _xx, const T _yy, const T _xy)
            : xx(_xx), yy(_yy), xy(_xy) {}
        MatrixSym2(const T s) : xx(s), yy(s), xy(s){}
        MatrixSym2(const MatrixSym2 & src) : xx(src.xx), yy(src.yy), xy(src.xy) {}

        // 代入演算子
        const MatrixSym2 & operator = (const MatrixSym2 & rhs);

        // 加減算
        MatrixSym2 operator + (const MatrixSym2 & rhs) const;
        const MatrixSym2 & operator += (const MatrixSym2 & rhs) const;
        MatrixSym2 operator - (const MatrixSym2 & rhs) const;
        const MatrixSym2 & operator -= (const MatrixSym2 & rhs) const;

        // トレースの計算
        T getTrace() const;

        // MatrixSym2<U>への型変換
        template <typename U>
        operator MatrixSym2<U> () const;
    }
}
namespace PS = ParticleSimulator;
\end{lstlisting}

%%%%%%%%%%%%%%%%%%%%%%%%%%%%%%%%%%%%%%%%%%%%%%%%%%%%%
\subsubsubsection{コンストラクタ}

\begin{screen}
\begin{verbatim}
template<typename T>
PS::MatrixSym2<T>();
\end{verbatim}
\end{screen}

\begin{itemize}

\item{{\bf 引数}}

なし。

\item{{\bf 機能}}

デフォルトコンストラクタ。メンバxx,yy,xyは0で初期化される。

\end{itemize}

\begin{screen}
\begin{verbatim}
template<typename T>
PS::MatrixSym2<T>(const T _xx,
                  const T _yy,
                  const T _xy);
\end{verbatim}
\end{screen}

\begin{itemize}

\item{{\bf 引数}}

{\tt \_xx}: 入力。{\tt const T}型。

{\tt \_yy}: 入力。{\tt const T}型。

{\tt \_xy}: 入力。{\tt const T}型。

\item{{\bf 機能}}

メンバ{\tt xx}、{\tt yy}、{\tt xy}をそれぞれ{\tt \_xx}、{\tt \_yy}、
{\tt \_xy}で初期化する。

\end{itemize}

\begin{screen}
\begin{verbatim}
template<typename T>
PS::MatrixSym2<T>(const T s);
\end{verbatim}
\end{screen}

\begin{itemize}

\item{{\bf 引数}}

{\tt s}: 入力。{\tt const T}型。

\item{{\bf 機能}}

メンバ{\tt xx}、{\tt yy}、{\tt xy}すべてを{\tt s}の値で初期化する。

\end{itemize}

\begin{screen}
\begin{verbatim}
template<typename T>
PS::MatrixSym2<T>(const PS::MatrixSym2<T> & src)
\end{verbatim}
\end{screen}

\begin{itemize}

\item{{\bf 引数}}

{\tt src}: 入力。{\tt const PS::MatrixSym2<T> \&}型。

\item{{\bf 機能}}

コピーコンストラクタ。{\tt src}で初期化する。

\end{itemize}

%%%%%%%%%%%%%%%%%%%%%%%%%%%%%%%%%%%%%%%%%%%%%%%%%%%%%
\subsubsubsection{代入演算子}

\begin{screen}
\begin{verbatim}
template<typename T>
const PS::MatrixSym2<T> & PS::MatrixSym2<T>::operator = 
                       (const PS::MatrixSym2<T> & rhs);
\end{verbatim}
\end{screen}

\begin{itemize}

\item{{\bf 引数}}

{\tt rhs}: 入力。{\tt const PS::MatrixSym2<T> \&}型。

\item{{\bf 返り値}}

{\tt const PS::MatrixSym2<T> \&}型。{\tt rhs}のxx,yy,xyの値を自身のメ
ンバxx,yy,xyに代入し自身の参照を返す。代入演算子。

\end{itemize}

%%%%%%%%%%%%%%%%%%%%%%%%%%%%%%%%%%%%%%%%%%%%%%%%%%%%%
\subsubsubsection{加減算}

\begin{screen}
\begin{verbatim}
template<typename T>
PS::MatrixSym2<T> PS::MatrixSym2<T>::operator + 
               (const PS::MatrixSym2<T> & rhs) const;
\end{verbatim}
\end{screen}

\begin{itemize}

\item{{\bf 引数}}

{\tt rhs}: 入力。{\tt const PS::MatrixSym2<T> \&}型。

\item{{\bf 返り値}}

{\tt PS::MatrixSym2<T> }型。{\tt rhs}のxx,yy,xyの値と自身のメンバ
xx,yy,xyの値の和を取った値を返す。

\end{itemize}

\begin{screen}
\begin{verbatim}
template<typename T>
const PS::MatrixSym2<T> & PS::MatrixSym2<T>::operator += 
                       (const PS::MatrixSym2<T> & rhs);
\end{verbatim}
\end{screen}

\begin{itemize}

\item{{\bf 引数}}

{\tt rhs}: 入力。{\tt const PS::MatrixSym2<T> \&}型。

\item{{\bf 返り値}}

{\tt const PS::MatrixSym2<T> \&}型。{\tt rhs}のxx,yy,xyの値を自身のメ
ンバxx,yy,xyに足し、自身を返す。

\end{itemize}

\begin{screen}
\begin{verbatim}
template<typename T>
PS::MatrixSym2<T> PS::MatrixSym2<T>::operator - 
               (const PS::MatrixSym2<T> & rhs) const;
\end{verbatim}
\end{screen}

\begin{itemize}

\item{{\bf 引数}}

{\tt rhs}: 入力。{\tt const PS::MatrixSym2<T> \&}型。

\item{{\bf 返り値}}

{\tt PS::MatrixSym2<T> }型。{\tt rhs}のxx,yy,xyの値と自身のメンバ
xx,yy,xyの値の差を取った値を返す。

\end{itemize}

\begin{screen}
\begin{verbatim}
template<typename T>
const PS::MatrixSym2<T> & PS::MatrixSym2<T>::operator -= 
                       (const PS::MatrixSym2<T> & rhs);
\end{verbatim}
\end{screen}

\begin{itemize}

\item{{\bf 引数}}

{\tt rhs}: 入力。{\tt const PS::MatrixSym2<T> \&}型。

\item{{\bf 返り値}}

{\tt const PS::MatrixSym2<T> \&}型。自身のメンバxx,yy,xyから{\tt rhs}
のxx,yy,xyを引き自身を返す。

\end{itemize}

%%%%%%%%%%%%%%%%%%%%%%%%%%%%%%%%%%%%%%%%%%%%%%%%%%%%%
\subsubsubsection{トレースの計算}

\begin{screen}
\begin{verbatim}
template<typename T>
T PS::MatrixSym2<T>::getTrace() const;
\end{verbatim}
\end{screen}

\begin{itemize}

\item{{\bf 引数}}

なし

\item{{\bf 返り値}}

{\tt T}型。

\item{{\bf 機能}}

  トレースを計算し、その結果を返す。

\end{itemize}

%%%%%%%%%%%%%%%%%%%%%%%%%%%%%%%%%%%%%%%%%%%%%%%%%%%%%
\subsubsubsection{{\tt MatrixSym2<U>}への型変換}

\begin{screen}
\begin{verbatim}
template<typename T>
template<typename U>
PS::MatrixSym2<T>::operator PS::MatrixSym2<U> () const;
\end{verbatim}
\end{screen}

\begin{itemize}

\item{{\bf 引数}}

  なし。

\item{{\bf 返り値}}

{\tt const PS::MatrixSym2<U>}型。

\item{{\bf 機能}}

\redtext{{\tt const PS::MatrixSym2<T>}型を{\tt const
    PS::MatrixSym2<U>}型にキャストする}

\end{itemize}



\subsubsection{PS::MatrixSym3}

PS::MatrixSym3はxx, yy, zz, xy, xz, yzの6要素を持つ。これらに対する様々
なAPIや演算子を定義した。それらの宣言を以下に記述する。この節ではこれ
らについて詳しく記述する。
\begin{lstlisting}[caption=MatrixSym3]
namespace ParticleSimulator{
    template<class T>
    class MatrixSym3{
    public:
        // メンバ変数6要素
        T xx, yy, zz, xy, xz, yz;

        // コンストラクタ
        MatrixSym3() : xx(T(0)), yy(T(0)), zz(T(0)),
                       xy(T(0)), xz(T(0)), yz(T(0)) {}
        MatrixSym3(const T _xx, const T _yy, const T _zz,
                   const T _xy, const T _xz, const T _yz )
                       : xx(_xx), yy(_yy), zz(_zz),
                       xy(_xy), xz(_xz), yz(_yz) {}
        MatrixSym3(const T s) : xx(s), yy(s), zz(s),
                                xy(s), xz(s), yz(s) {}
        MatrixSym3(const MatrixSym3 & src) :
            xx(src.xx), yy(src.yy), zz(src.zz),
            xy(src.xy), xz(src.xz), yz(src.yz) {}

        // 代入演算子
        const MatrixSym3 & operator = (const MatrixSym3 & rhs);

        // 加減算
        MatrixSym3 operator + (const MatrixSym3 & rhs) const;
        const MatrixSym3 & operator += (const MatrixSym3 & rhs) const;
        MatrixSym3 operator - (const MatrixSym3 & rhs) const;
        const MatrixSym3 & operator -= (const MatrixSym3 & rhs) const;

        // トレースを取る
        T getTrace() const;

        // MatrixSym3<U>への型変換
        template <typename U>
        operator MatrixSym3<U> () const;
    }
}
namespace PS = ParticleSimulator;
\end{lstlisting}

%%%%%%%%%%%%%%%%%%%%%%%%%%%%%%%%%%%%%%%%%%%%%%%%%%%%%
\subsubsubsection{コンストラクタ}

\begin{screen}
\begin{verbatim}
template<typename T>
PS::MatrixSym3<T>();
\end{verbatim}
\end{screen}

\begin{itemize}

\item{{\bf 引数}}

なし。

\item{{\bf 機能}}

デフォルトコンストラクタ。6要素は0で初期化される。

\end{itemize}

\begin{screen}
\begin{verbatim}
template<typename T>
PS::MatrixSym3<T>(const T _xx,
                  const T _yy,
                  const T _zz,
                  const T _xy,
                  const T _xz,
                  const T _yz);
\end{verbatim}
\end{screen}

\begin{itemize}

\item{{\bf 引数}}

{\tt \_xx}: 入力。{\tt const T}型。

{\tt \_yy}: 入力。{\tt const T}型。

{\tt \_zz}: 入力。{\tt const T}型。

{\tt \_xy}: 入力。{\tt const T}型。

{\tt \_xz}: 入力。{\tt const T}型。

{\tt \_yz}: 入力。{\tt const T}型。

\item{{\bf 機能}}

メンバ{\tt xx}、{\tt yy}、{\tt zz}、{\tt xy}、{\tt xz}、{\tt yz}をそれ
ぞれ{\tt \_xx}、{\tt \_yy}、{\tt \_zz}、{\tt \_xy}、{\tt \_xz}、{\tt
  \_yz}で初期化する。

\end{itemize}

\begin{screen}
\begin{verbatim}
template<typename T>
PS::MatrixSym3<T>(const T s);
\end{verbatim}
\end{screen}

\begin{itemize}

\item{{\bf 引数}}

{\tt s}: 入力。{\tt const T}型。

\item{{\bf 機能}}

6要素すべてを{\tt s}の値で初期化する。

\end{itemize}

\begin{screen}
\begin{verbatim}
template<typename T>
PS::MatrixSym3<T>(const PS::MatrixSym3<T> & src)
\end{verbatim}
\end{screen}

\begin{itemize}

\item{{\bf 引数}}

{\tt src}: 入力。{\tt const PS::MatrixSym3<T> \&}型。

\item{{\bf 機能}}

コピーコンストラクタ。{\tt src}で初期化する。

\end{itemize}

%%%%%%%%%%%%%%%%%%%%%%%%%%%%%%%%%%%%%%%%%%%%%%%%%%%%%
\subsubsubsection{代入演算子}

\begin{screen}
\begin{verbatim}
template<typename T>
const PS::MatrixSym3<T> & PS::MatrixSym3<T>::operator = 
                       (const PS::MatrixSym3<T> & rhs);
\end{verbatim}
\end{screen}

\begin{itemize}

\item{{\bf 引数}}

{\tt rhs}: 入力。{\tt const PS::MatrixSym3<T> \&}型。

\item{{\bf 返り値}}

{\tt const PS::MatrixSym3<T> \&}型。{\tt rhs}の6要素それぞれの値を自
身の6要素それぞれに代入し自身の参照を返す。代入演算子。

\end{itemize}

%%%%%%%%%%%%%%%%%%%%%%%%%%%%%%%%%%%%%%%%%%%%%%%%%%%%%
\subsubsubsection{加減算}

\begin{screen}
\begin{verbatim}
template<typename T>
PS::MatrixSym3<T> PS::MatrixSym3<T>::operator + 
               (const PS::MatrixSym3<T> & rhs) const;
\end{verbatim}
\end{screen}

\begin{itemize}

\item{{\bf 引数}}

{\tt rhs}: 入力。{\tt const PS::MatrixSym3<T> \&}型。

\item{{\bf 返り値}}

{\tt PS::MatrixSym3<T> }型。{\tt rhs}の6要素それぞれの値と自身の6要
素の値の和を取った値を返す。

\end{itemize}

\begin{screen}
\begin{verbatim}
template<typename T>
const PS::MatrixSym3<T> & PS::MatrixSym3<T>::operator += 
                       (const PS::MatrixSym3<T> & rhs);
\end{verbatim}
\end{screen}

\begin{itemize}

\item{{\bf 引数}}

{\tt rhs}: 入力。{\tt const PS::MatrixSym3<T> \&}型。

\item{{\bf 返り値}}

{\tt const PS::MatrixSym3<T> \&}型。{\tt rhs}の6要素それぞれの値を自
身の6要素それぞれに足し、自身を返す。

\end{itemize}

\begin{screen}
\begin{verbatim}
template<typename T>
PS::MatrixSym3<T> PS::MatrixSym3<T>::operator - 
               (const PS::MatrixSym3<T> & rhs) const;
\end{verbatim}
\end{screen}

\begin{itemize}

\item{{\bf 引数}}

{\tt rhs}: 入力。{\tt const PS::MatrixSym3<T> \&}型。

\item{{\bf 返り値}}

{\tt PS::MatrixSym3<T> }型。{\tt rhs}の6要素それぞれの値と自身の6要
素それぞれの値の差を取った値を返す。

\end{itemize}

\begin{screen}
\begin{verbatim}
template<typename T>
const PS::MatrixSym3<T> & PS::MatrixSym3<T>::operator -= 
                       (const PS::MatrixSym3<T> & rhs);
\end{verbatim}
\end{screen}

\begin{itemize}

\item{{\bf 引数}}

{\tt rhs}: 入力。{\tt const PS::MatrixSym3<T> \&}型。

\item{{\bf 返り値}}

{\tt const PS::MatrixSym3<T> \&}型。自身の6要素それぞれから{\tt rhs}
の6要素それぞれを引き自身を返す。

\end{itemize}

%%%%%%%%%%%%%%%%%%%%%%%%%%%%%%%%%%%%%%%%%%%%%%%%%%%%%
\subsubsubsection{トレースの計算}

\begin{screen}
\begin{verbatim}
template<typename T>
T PS::MatrixSym3<T>::getTrace() const;
\end{verbatim}
\end{screen}

\begin{itemize}

\item{{\bf 引数}}

なし

\item{{\bf 返り値}}

{\tt T}型。

\item{{\bf 機能}}

  トレースを計算し、その結果を返す。

\end{itemize}

%%%%%%%%%%%%%%%%%%%%%%%%%%%%%%%%%%%%%%%%%%%%%%%%%%%%%
\subsubsubsection{{\tt MatrixSym3<U>}への型変換}

\begin{screen}
\begin{verbatim}
template<typename T>
template<typename U>
PS::MatrixSym3<T>::operator PS::MatrixSym3<U> () const;
\end{verbatim}
\end{screen}

\begin{itemize}

\item{{\bf 引数}}

  なし。

\item{{\bf 返り値}}

{\tt const PS::MatrixSym3<U>}型。

\item{{\bf 機能}}

\redtext{{\tt const PS::MatrixSym3<T>}型を{\tt const
    PS::MatrixSym3<U>}型にキャストする}

\end{itemize}



\subsubsection{行列型のラッパー}

対称行列型のラッパーの定義を以下に示す。
\begin{lstlisting}[caption=matrixsymwrapper]
namespace ParticleSimulator{
    typedef MatrixSym2<F32> F32mat2;
    typedef MatrixSym3<F32> F32mat3;
    typedef MatrixSym2<F64> F64mat2;
    typedef MatrixSym3<F64> F64mat3;
#ifdef PARTICLE_SIMULATOR_TOW_DIMENSION
    typedef F32mat2 F32mat;
    typedef F64mat2 F64mat;
#else
    typedef F32mat3 F32mat;
    typedef F64mat3 F64mat;
#endif
}
namespace PS = ParticleSimulator;
\end{lstlisting}

すなわちPS::F32mat2, PS::F32mat3, PS::F64mat2, PS::F64mat3はそれぞれ単
精度2x2対称行列、倍精度2x2対称行列、単精度3x3対称行列、倍精度3x3対称行
列である。FDPSで扱う空間座標系を2次元とした場合、PS::F32matと
PS::F64matはそれぞれ単精度2x2対称行列、倍精度2x2対称行列となる。一方、
FDPSで扱う空間座標系を3次元とした場合、PS::F32matとPS::F64matはそれぞ
れ単精度3x3対称行列、倍精度3x3対称行列となる。





\subsection{SEARCH\_MODE型}

\subsubsection{概要}

本節では、SEARCH\_MODE型について記述する。SEARCH\_MODE型は相互作用ツリー
クラスのテンプレート引数としてのみ使用されるものである。この型によって、
相互作用ツリークラスで計算する相互作用のモードを決定する。SEARCH\_MODE
型にはSEARCH\_MODE\_LONG, SEARCH\_MODE\_LONG\_CUTOFF,
SEARCH\_MODE\_GATHER, SEARCH\_MODE\_SCATTER, SEARCH\_MODE\_SYMMETRYが
ある。以下に、それぞれが対応する相互作用のモードについて記述する。

\subsubsection{SEARCH\_MODE\_LONG}

この型を使用するのは、遠くの粒子からの寄与を複数の粒子にまとめた超粒子
からの寄与として計算する場合である。開放境界条件における重力やクーロン
力に適用できる。

\subsubsection{SEARCH\_MODE\_LONG\_CUTOFF}

この型を使用するのは、遠くの粒子からの寄与を複数の粒子にまとめた超粒子
からの寄与として計算し、かつ有限の距離までの寄与しか計算しない場合であ
る。周期境界条件における重力やクーロン力(Particle Mesh法を並用)などに
適用できる。

\subsubsection{SEARCH\_MODE\_GATHER}

この型を使用するのは、相互作用の到達距離が有限でかつ、その到達距離がi
粒子の大きさで決まる場合である。

\subsubsection{SEARCH\_MODE\_SCATTER}

この型を使用するのは、相互作用の到達距離が有限でかつ、その到達距離がj
粒子の大きさで決まる場合である。

\subsubsection{SEARCH\_MODE\_SYMMETRY}

この型を使用するのは、相互作用の到達距離が有限でかつ、その到達距離がi,
j粒子両方の大きさで決まる場合である。



\subsection{列挙型}

\subsubsection{概要}

本節ではFDPSで定義されている列挙型について記述する。列挙型には
BOUNDARY\_CONDITION型が存在する。以下、各列挙型について記述する。

\subsubsection{BOUNDARY\_CONDITION型}
\label{sec:type_enum}

\subsubsubsection{概要}

BOUNDARY\_CONDITION型は境界条件を指定するためのデータ型である。これは
以下のように定義されている。
\begin{lstlisting}[caption=boundarycondition]
namespace ParticleSimulator{
    enum BOUNDARY_CONDITION{
        BOUNDARY_CONDITION_OPEN,
        BOUNDARY_CONDITION_PERIODIC_X,
        BOUNDARY_CONDITION_PERIODIC_Y,
        BOUNDARY_CONDITION_PERIODIC_Z,
        BOUNDARY_CONDITION_PERIODIC_XY,
        BOUNDARY_CONDITION_PERIODIC_XZ,
        BOUNDARY_CONDITION_PERIODIC_YZ,
        BOUNDARY_CONDITION_PERIODIC_XYZ,
        BOUNDARY_CONDITION_SHEARING_BOX,
        BOUNDARY_CONDITION_USER_DEFINED,
    };
}
namespace PS = ParticleSimulator;
\end{lstlisting}

以下にどの変数がどの境界条件に対応するかを記述する。

\subsubsubsection{PS::BOUNDARY\_CONDITION\_OPEN}

開放境界となる。

\subsubsubsection{PS::BOUNDARY\_CONDITION\_PERIODIC\_X}

x軸方向のみ周期境界、その他の軸方向は開放境界となる。周期の境界の下限
は閉境界、上限は開境界となっている。この境界の規定はすべての軸方向にあ
てはまる。

\subsubsubsection{PS::BOUNDARY\_CONDITION\_PERIODIC\_Y}

y軸方向のみ周期境界、その他の軸方向は開放境界となる。

\subsubsubsection{PS::BOUNDARY\_CONDITION\_PERIODIC\_Z}

z軸方向のみ周期境界、その他の軸方向は開放境界となる。

\subsubsubsection{PS::BOUNDARY\_CONDITION\_PERIODIC\_XY}

x, y軸方向のみ周期境界、その他の軸方向は開放境界となる。

\subsubsubsection{PS::BOUNDARY\_CONDITION\_PERIODIC\_XZ}

x, z軸方向のみ周期境界、その他の軸方向は開放境界となる。

\subsubsubsection{PS::BOUNDARY\_CONDITION\_PERIODIC\_YZ}

y, z軸方向のみ周期境界、その他の軸方向は開放境界となる。

\subsubsubsection{PS::BOUNDARY\_CONDITION\_PERIODIC\_XYZ}

x, y, z軸方向すべてが周期境界となる。

\subsubsubsection{PS::BOUNDARY\_CONDITION\_SHEARING\_BOX}

未実装。

\subsubsubsection{PS::BOUNDARY\_CONDITION\_USER\_DEFINED}

未実装。



%%\subsection{MPIデータ型}

%%\subsubsection{概要}

本節ではFDPSで定義されているMPIデータ型について記述する。

\subsubsection{PS::GetDataType$<$S32$>$()}

PS::S32に対応するMPIデータ型である。

\subsubsection{PS::GetDataType$<$S64$>$()}

PS::S64に対応するMPIデータ型である。

\subsubsection{PS::GetDataType$<$U32$>$()}

PS::U32に対応するMPIデータ型である。

\subsubsection{PS::GetDataType$<$U64$>$()}

PS::U64に対応するMPIデータ型である。

\subsubsection{PS::GetDataType$<$F32$>$()}

PS::F32に対応するMPIデータ型である。

\subsubsection{PS::GetDataType$<$F64$>$()}

PS::F64に対応するMPIデータ型である。

\subsubsection{PS::MPI\_F32VEC}

PS::F32vecに対応するMPIデータ型である。PS::F32vecは、2次元直交座標系を
扱っている場合には2次元ベクトル、3次元直交座標系を扱っている場合には3
次元ベクトルである。

\subsubsection{PS::MPI\_F64VEC}

PS::F64vecに対応するMPIデータ型である。PS::F64vecは、2次元直交座標系を
扱っている場合には2次元ベクトル、3次元直交座標系を扱っている場合には3
次元ベクトルである。


